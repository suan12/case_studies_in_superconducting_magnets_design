\chapter{电磁场}
\section{引言}
本章我们以麦克斯韦方程组为主线回顾电磁理论。这个回顾是为了``唤起"读者对电磁理论的理解,以便本书的主题---超导磁体---可以量化的展开。
随后,专题研究部分给出了几个在大量磁体应用中抽象出来的可用解析法分析的特例。
\section{Maxwell方程}
麦克斯韦方程组包括四个基本方程:1)Gauss定律;2)Ampere定律;3)Faraday定律;4)磁通连续定律。此外,我们还会常用到电荷守恒方程及其他本构关系。

本书中如非特别指明,电磁量都采用SI单位(表\ref{emquantity})。磁体界常混用磁场强度$\vec{H}$和磁通密度(或磁感应强度)$\vec{B}$。尽管这个做法并无大碍,也基本不会导致混淆,但我们应对此提起警惕,比如从$\vec{M}$ vs. $\vec{H}$图计算能量的时候。

自由空间的磁导率$\mu_0=4\pi \times 10^{-7}$ H/m;自由空间介电系数$\epsilon_0=\frac{1}{\mu_0c^2}$,近似值为$8.85\times 10^{-12}$ F/m。
附录IA给出了其他物理常数及部分常用非SI单位到SI单位的转换因子。

超导磁体磁场$\vec{H}$的主要产生源是电流密度,故相对较小的时变$\vec{D}$场对$\vec{H}$的贡献在本章的麦克斯韦方程中并未包括进来。

\begin{table}[htbp]\small
  \centering
  \caption{电磁量} \label{emquantity}
\begin{tabular}{|c|l|l|}
  \hline
  % after \\: \hline or \cline{col1-col2} \cline{col3-col4} ...
  符号 & 名称 & SI单位 \\ \hline
  $E$&电场&[V/m] \\ \hline
  $H$&磁场&[A/m] \\ \hline
  $B$&磁感应强度&[T]\\ \hline
  $J$&电流密度&[$\mathrm{A/m^2}$] \\ \hline
  $K$&面电流密度&[A/m]\\ \hline
  $\rho_c$&电荷密度&[$\mathrm{C/m^3}$]\\ \hline
  $\sigma_c$&面电荷密度&[$\mathrm{C/m^2}]$\\ \hline
  $\rho_e$&电阻&[$\Omega$m]\\
  \hline
\end{tabular}
\end{table}

\subsection{电场Gauss定律}
自由空间中电场Gauss定律的积分形式为:
\begin{equation}
\oint_S \epsilon_o\vec{E}\cdot d\vec{A}=\int_V\rho_c dV
\end{equation}
$\epsilon_0\vec{E}$场的面积分等于表面S围成的体积V内的净电荷量。式2.1中,$d\vec{A} =\vec{n}dS$,其中$\vec{n}$是表面元上指向外侧的单位法向量。
从2.1得其微分形式为:
\begin{equation}
  \epsilon_0 \nabla \cdot \vec{E}=\rho_c
\end{equation}

\textbf{边界条件}:在电荷密度为$\sigma_c[\mathrm{C/m^2}]$的面上,从区域1到区域2的电场法向分量不连续:
\begin{equation}
  \vec{n}_{12}\cdot (\vec{E}_2-\vec{E}_1)=\sigma_c/\epsilon_0
\end{equation}
式中,单位矢量$\vec{n}_{12}$从区域1指向区域2。

\subsection{Ampere定律}
Ampere定律的积分形式为:
\begin{equation}
\oint_C \vec{H}\cdot d\vec{S}=\int_S \vec{J}_f d\vec{A}
\end{equation}
方程表明,$\vec{H}$场的线积分等于C围成的面S内的总的``自由"电流,即不含有磁化电流。对应的微分形式为:
\begin{equation}
   \nabla \times \vec{H}=\vec{J}_f
\end{equation}

上述两个方程都没有计入$\vec{H}$的另一个源:$\frac{\partial{\vec{E}}}{\partial{t}}$。如前所述,该源可忽略。

\textbf{边界条件}:如果存在``自由"面电流密度$\vec{K}_f$[A/m],则通过区域1到区域2的磁场在切向不连续,满足:
\begin{equation}
  \vec{n}_{12}\times (\vec{H}_2-\vec{H}_1)=\vec{K}_f
\end{equation}

\subsection{Faraday定律}
Faraday定律的积分形式为:
\begin{equation}\label{eqn:faradaylaw}
\oint_C \vec{E}\cdot d\vec{S}=-\frac{d}{dt}\int_S \vec{B}\cdot d\vec{A}
\end{equation}

方程表明,$\vec{E}$场的线积分等于由C围成的面S内的总磁通对时间的变化率的负值。对应的微分形式为:
\begin{equation}
   \nabla \times \vec{E}=-\frac{\partial{\vec{B}}}{\partial{t}}
\end{equation}

\textbf{边界条件}:通过区域1到区域2的$\vec{E}$场的切向分量总是连续的:
\begin{equation}\label{eqn:faraday bc}
  \vec{n}_{12}\times (\vec{E}_2-\vec{E}_1)=0
\end{equation}

\subsection{磁通连续性}
磁通连续性方程的积分形式为:
\begin{equation}
\oint_S \vec{B}\cdot d\vec{A}=0
\end{equation}
方程表明,$\vec{B}$场在面S上的面积分为0,即$\vec{B}$场无源。对应的微分形式为:
\begin{equation}
  \nabla \cdot \vec{B}=0
\end{equation}

\textbf{边界条件}:通过区域1到区域2的$\vec{B}$场的法向分量是连续的,即:
\begin{equation}
  \vec{n}_{12}\cdot (\vec{B}_2-\vec{B}_1)=0
\end{equation}

下面将会看到,在磁介质中$\vec{B}$是磁场强度$\vec{H}$和磁化强度$\vec{M}$之叠加。这意味着不论两种介质的磁化是多么不同,$\vec{B}$场在两种介质中的法向分量连续性都可以保持。

\subsection{电荷守恒}
``自由"电流密度$\vec{J}_f$与``自由"电荷密度$\rho_{cf}$的时变率有关,积分形式为:
\begin{equation}
\oint_S \vec{J}_f\cdot d\vec{A}=-\frac{d}{dt}\int_V \rho_{cf}d\vec{V}
\end{equation}

微分形式为:
\begin{equation}
   \nabla \cdot \vec{J}_f=-\frac{\partial{\rho_{cf}}}{\partial{t}}
\end{equation}

\subsection{本构关系}
磁感应强度$\vec{B}$,磁场强度$\vec{H}$和磁化强度$\vec{M}$的关系为:
\begin{equation}
\vec{B}=\mu_0(\vec{H}+\vec{M})
\end{equation}

在同质、各向同性、线性介质(本书通常以此设定为前提)中,$\vec{B}=\mu \vec{H}=\mu_0(1+\chi)\vec{H}$。式中的磁导率$\mu$和磁化系数$\chi$
一般假定与磁场无关。铁磁材料如``高$\mu$"屏蔽材料的$\chi$可高达$10^6$。典型的顺磁材料,例如氧,$\chi=10^{-6}$;抗磁材料如单原子气体、多数液体的
磁化系数是负值。第I类超导体具有完全抗磁性,有$\chi=-1, \mu=0$。

金属等导体材料中,电场$\vec{E}$会激发出电流密度$\vec{J}$,两者关系为:
\begin{equation}
  \vec{J}=\frac{\vec{E}}{\rho_e}
\end{equation}
式中,$\rho_e$是金属的电阻率[$\Omega$m]。

\section{准静态}
电场$\vec{E}$和磁场$\vec{B}$通过法拉第定律耦合在一起。自由空间里,必须求解如下的完整方程组:
\begin{subequations}
	\begin{align}
\nabla \cdot (\epsilon_0\vec{E})=&\rho_c \\
\nabla \times \vec{E} =&-\frac{\partial{B}}{\partial{t}} \\
\nabla \times \vec{H} =&\vec{J}_f+\epsilon_0 \frac{\partial{E}}{\partial{t}}  \\
\nabla \cdot \vec{B} =&0  \\
\nabla \cdot \vec{J}_f =&-\frac{\partial{\rho_{c}}}{\partial{t}}
	\end{align}
\end{subequations}

如果电场$\vec{E}$和磁场$\vec{B}$能够解耦,将大大简化解方程组2.17的难度。
``准静态"分析就是一种可以满足很多重要实际应用的近似方法。
最简单的做法就以静态方程替代式2.17。于是,在$0^{th}$阶近似下,我们有:
\begin{subequations}
	\begin{align}
\nabla \cdot (\epsilon_0\vec{E}) &=\rho_c \\
\nabla \times \vec{E} &=0  \\
\nabla \times \vec{H} &=\vec{J}_{f0}  \\
\nabla \cdot \vec{B} &=0  \\
\nabla \cdot \vec{J}_f &=0
	\end{align}
\end{subequations}

以$\mathrm{0^{th}}$阶近似电场$\vec{E}$为例,它可以独立于$\vec{H}$解出。
稍复杂的情况下,感生场相比于初始的时变场若可以忽略,则可取准静态的$\mathrm{1^{st}}$阶近似,我们有:
\begin{subequations}
	\begin{align}
\nabla \cdot (\epsilon_0\vec{E}) &=\rho_{c1} \\
\nabla \times \vec{E} &=-\frac{\partial{B_0}}{\partial{t}} \\
\nabla \times \vec{H} &=\vec{J}_{f1}+\epsilon_0 \frac{\partial{E_0}}{\partial{t}}  \\
\nabla \cdot \vec{B} &=0 \\
\nabla \cdot \vec{J}_f &=-\frac{\partial{\rho_{c0}}}{\partial{t}}
	\end{align}
\end{subequations}

注意到此时的$\vec{E_1}$仍是和$\vec{H_1}$无关的。一般而言,电源的$\vec{J_f}$仅有$\vec{J_{f0}}$分量;在金属中,有$\vec{J_{f1}}=\vec{E_1}/\rho_e$。

显然,上述的近似过程可以无限的进行下去,但对于本章专题中所关系的``低频"情况,解出$\mathrm{0^{th}}$阶和$\mathrm{1^{st}}$阶场就够了。

\section{Poynting矢量}
Poynting定理可用下式表示:
\begin{equation}\label{eqn:poynting}
-\nabla\cdot \vec{S}=p+\frac{dw}{dt}
\end{equation}
式中,$\vec{S}[\mathrm{W/m^2}]$是Poynting矢量,定义为$\vec{P}=\vec{E}\times \vec{H}$。
$p$是功率耗散密度,$w$是以电磁能存储的能量密度。

方程表明,S矢量的散度的负值等于$p$(能量耗散密度与产生密度之差)与$dw/dt$(能量存储敏度的变化率)之和。如果$\nabla \cdot \vec{S}=0$,表明
系统内能量平衡,即流入和流出相等;如果$\nabla \cdot \vec{S}<0$,表明有能量流入系统,在系统内要么被耗散,要么被存储。

\subsubsection{简谐情况}
处理简谐时变电场$\vec{E}$时,常用复数,即有$\vec{E}=\vec{E_0}e^{j\omega t}$。从而,$\vec{J}=(\vec{E}/\rho_e)e^{j\omega t}$。此时,时均功率耗散密度$<p>$写成:
\begin{equation}\label{eqn:poynting sincase}
  <p>=\frac{1}{2}\vec{E}\cdot \vec{J^*}=\frac{1}{2\rho_e}|E|^2=\frac{\rho_e}{2}|J|^2
\end{equation}
式中,$\vec{J^*}$是$\vec{J}$的复共轭量。

简谐条件下,S矢量写成如下形式:
\begin{subequations}\label{eqn:poynting s-vector sin}
	\begin{align}
\vec{S}&=\frac{1}{2}(\vec{E}\times \vec{H^*}) \\
-\oint_S \vec{S}\cdot d\vec{A}&=<P>+j2\omega (<E_m>-<E_e>)
	\end{align}
\end{subequations}
式中,$<P>$[W],$<E_m>$[J],$<E_e>$[J]分别是总能耗、磁场能和电场能。
每一个时均量都是通过对系统体积分得到的:
\begin{subequations}
	\begin{align}
<P>&= \frac{1}{2\rho_e}\int_V|E|^2dV\\
<E_m>&= \frac{\mu_0}{4}\int_V|H|^2dV \\
<E_e>&= \frac{\epsilon_0}{4}\int_V|E|^2dV
	\end{align}
\end{subequations}

复数形式的Poynting矢量$\vec{S}$展开到$\mathrm{1^{st}}$阶场的形式为:
\begin{equation}\label{eqn:1st poynting}
\vec{S}=\frac{1}{2}\left(\vec{E}_0\times \vec{H}_0^*+\vec{E}_0\times \vec{H}_1^*+\vec{E}_1\times \vec{H}_0^*\right)
\end{equation}

\section{场的标量势解法}
静电场因其旋度为零($\nabla \times \vec{E}=0$),是保守场。故存在一个标量势,满足:
\begin{equation}
  \vec{E}=-\nabla \phi
\end{equation}

于是,$\nabla\cdot\vec{E}$可以写为:
\begin{equation}
  \nabla\cdot\vec{E}=-\nabla\cdot\nabla\phi=-\nabla^2\phi
\end{equation}

若无电荷密度存在,即$\rho_c=0$,方程2.2可约化为:
\begin{equation}
  \nabla\cdot\vec{E}=0
\end{equation}

进而,可以得到:
\begin{equation}
\nabla^2\phi=0
\end{equation}

方程2.28即所谓的Laplace方程,它给出了使用标量势求解矢量$\vec{E}$的方法。

类似的,在直流条件下,若无自由电流存在,因为$\nabla\times \vec{H}=0$,则磁场$\vec{H}$也可以由满足$\vec{H}=-\nabla \phi$的标量势方程导出。
在电磁领域之外,Laplace方程在工程中还有其他用于求解时间无关变量的应用,比如:
无源、各向同性传导介质中的温度($T$);体积膨胀;无力、无动量存在,各向同性弹性介质中x-,y-和z-向的线性应力和。

下面将给出二维圆柱坐标和三维球坐标下的Laplace方程的解。
\subsection{二维柱坐标}
二维柱坐标系$(r,\theta)$下的电势$\nabla^2\phi$形式为:
\begin{equation}
  \nabla^2\phi = \frac{1}{r}\frac{\partial}{\partial r}\left(r\frac{\partial \phi}{\partial r}\right)+\frac{1}{r^2}\frac{\partial^2\phi}{\partial\theta^2}=0
\end{equation}

解方程2.29的标准技术是将$\phi$表示为两个分别仅与一个坐标有关的函数之积:
\begin{equation}
  \phi=R(r)\Theta(\theta)
\end{equation}

方程2.29的解的一般形式如下:
\begin{subequations}
	\begin{align}
\mbox{对于n=0,}\phi_{0} &= (A_1 \ln r+A_2)(C_1 \theta+C_2) \\
\mbox{对于n>0,} \phi_{n} &= (A_1 r^n+A_2 r^{-n})(C_1 \sin n\theta +C_2 \cos n\theta)
	\end{align}
\end{subequations}
其中,$A_1, A_2, C_1, C_2$都是常数。

\subsubsection{特例}
\begin{description}
  \item[$n=0$] 最简单的情况。该条件下,场量在空间上仅依赖于$1/r$。实例包括线电荷($\lambda=2\pi \epsilon_0$)产生的电场以及  线电流($I=2\pi$)产生的磁场。
  此时,由$[\phi_0]_E=\ln r$可得$\vec{E}=(1/r)\vec{i}_r $;由$[\phi_0]_H=\theta$可得$\vec{E}=(1/r)\vec{i}_{\theta}$。可见,远离源时,场强以$1/r$衰减。
\item[$n=1$] 电势$\phi_1=\sin\theta/r$ 和$\phi_{1^\prime}=\cos \theta/r$与二维电/磁偶极子的电场/磁场有关。注意到两式在原点处($r$=0)都有奇点,故它们通常仅用于表示不含原点的偶极子场。
究竟是选$\sin\theta$还是$\cos\theta$,取决于场在坐标系中的取向。
此外,$\phi_1^\prime=r \sin \phi$和$\phi_{1^\prime}^\prime=r \cos \phi$与均匀矢量场有关。在第2章
和第3章,我们将研究几个二维偶极子场的案例。
 \item[$n=2$] 电势$\phi_2=\cos 2\theta/r^2$和$\phi_2^\prime= r^2 \cos 2\theta$与二维四极子场有关。前者因在原点有奇点,仅用于不含原点的空间;后者可用于有限远内的全部空间。第3章将研究一个理想四极磁体。
\end{description}

\subsection{球坐标}
球坐标$(r, \theta, \varphi)$下的势方程:
\begin{equation}\label{eqn:laplace sph1}
\begin{split}
  \nabla\cdot\nabla\phi=&\nabla^2\phi=\frac{1}{r^2}\frac{\partial}{\partial r}\left(r^2\frac{\partial \phi}{\partial r}\right)
  +\frac{1}{r^2\sin \theta}\frac{\partial}{\partial \theta}\left(\sin\theta\frac{\partial \phi}{\partial \theta}\right)\\
  &+\frac{1}{r^2\sin^2 \theta}\frac{\partial^2 \phi}{\partial \varphi^2}
\end{split}
\end{equation}

类似的,$\nabla^2\phi$的解也可以写成三个各仅含一个坐标的函数的乘积形式:
\begin{subequations}
	\begin{align}
  \phi&=R(r)\Theta(\theta)\Phi(\varphi)\\
  R(r)&=A_1 r^n+A_2 r^{-(n+1)} \\
  \Theta(\theta)&=C P_n^m(\cos \theta), (m \le n) \\
  \Phi(\varphi)&=D_1 \sin m\varphi +D_2 \cos m\varphi
  	\end{align}
\end{subequations}
其中,$A_1, A_2, C, D_1, D_2$都是常数。$P_n^0(\cos \theta)$,或其简写$P_n(\cos\theta)$,就是所谓的\textbf{勒让德函数}(Legendre Functions)。
它在\textit{设计}空间均匀螺管磁体时很有用,此类磁体在设计阶段常假定其磁场关于$z$轴($\theta=0$)对称。$P_n^m(\cos\theta)$,被称为\textbf{伴随勒让德函数}(Associated Legendre Functions)。它在最小化一个\textit{实际}螺管磁体的设计``误差"时很有用。
第3章将更详细的讨论由方程2.33导出的磁场表达式。
表2.2给出了$n=0-8$时的$P_n(\cos\theta)$以及$n=1-4(0<m\le n)$时的$P_n^m(\cos\theta)$。
表2.3给出了特定$n$和$m$组合下的$P_n^m(0)$。
表2.4给出了方程2.33在笛卡尔坐标系下的解;
这些表达式在设计和分析匀场电磁体和铁磁器件时,非常重要。

\subsubsection{特例}
\begin{description}
  \item[$n=m=0$] 此时给出最简单的解,$\phi_0\propto 1/r$。一个常见的例子是电量为$4\pi\epsilon_0$的点电荷产生的电场$\vec{E}=1/r^2 \vec{i_r}$($r>0$)。
  \item[$n=1, m=0$] 有两个解,分别是$\phi_1 \propto \cos\theta /r^2$和$\phi_1^\prime\propto r\cos\theta$。前者表示球外的偶极场,该场由球面上的电荷产生;
  后者是球内的均匀场。磁矩的偶极场亦可由$\phi_1$导出。
\end{description}

\subsection{正交坐标系下的微分算符}
下面将给出四个矢量微分算符---grad($\nabla$)、div($\nabla\cdot$)、curl($\nabla\times$)、div grad($\nabla^2$)---在正交坐标系下的表达式。
\subsubsection{笛卡尔坐标}
三个正交坐标是:$x, y, z$。
\begin{subequations}\label{eqn:field cart}
	\begin{align}
\nabla U=&\frac{\partial U}{\partial x}\vec{i_x} +\frac{\partial U}{\partial y}\vec{i_y}+\frac{\partial U}{\partial z}\vec{i_z}  \\
\nabla\cdot \vec{A}=& \frac{\partial{A_x}}{\partial x} +\frac{\partial{A_y}}{\partial y}+\frac{\partial{A_z}}{\partial z}\\
\nabla\times \vec{A}=& \left(\frac{\partial{A_z}}{\partial y} -\frac{\partial{A_y}}{\partial z}\right)\vec{i_x} + \left(\frac{\partial{A_x}}{\partial z} -\frac{\partial{A_z}}{\partial x}\right) \vec{i_y}
+ \left(\frac{\partial{A_y}}{\partial x} -\frac{\partial{A_x}}{\partial y}\right)\vec{i_z} \\
\nabla^2 U=&\frac{\partial^2 U}{\partial x^2}+\frac{\partial^2 U}{\partial y^2}+\frac{\partial^2 U}{\partial z^2}
  	\end{align}
\end{subequations}

\subsubsection{柱坐标}
三个正交坐标是:$r, \theta, z$。
\begin{subequations}\label{eqn:field cyl}
	\begin{align}
\nabla U=&\frac{\partial U}{\partial r}\vec{i_r} +\frac{1}{r}\frac{\partial U}{\partial \theta}\vec{i_{\theta}}+\frac{\partial U}{\partial z}\vec{i_z} \\
\nabla\cdot \vec{A} =& \frac{1}{r}\frac{\partial{(r A_r)}}{\partial r} +\frac{1}{r}\frac{\partial{A_\theta}}{\partial \theta}+\frac{\partial{A_z}}{\partial z} \\
\nabla\times \vec{A}=& \left(\frac{1}{r}\frac{\partial{A_z}}{\partial \theta} -\frac{\partial{A_\theta}}{\partial z}\right)\vec{i_r} + \left(\frac{\partial{A_r}}{\partial z} -\frac{\partial{A_z}}{\partial r}\right) \vec{i_\theta}+ \left(\frac{1}{r}\frac{\partial{(r A_\theta)}}{\partial r} -\frac{1}{r}\frac{\partial{A_r}}{\partial \theta}\right)\vec{i_z}\\
\nabla^2 U=&\frac{1}{r}\frac{\partial}{\partial r}\left(r\frac{\partial U}{\partial r}\right)+\frac{1}{r^2}\frac{\partial^2 U}{\partial \theta^2}+\frac{\partial^2 U}{\partial z^2}
  	\end{align}
\end{subequations}

\subsubsection{球坐标}
三个正交坐标是:$r, \theta, \varphi$。
\begin{subequations}\label{eqn:field sphl}
	\begin{align}
\nabla U=&\frac{\partial U}{\partial r}\vec{i_r} +\frac{1}{r}\frac{\partial U}{\partial \theta}\vec{i_{\theta}}+\frac{1}{r\sin\theta}\frac{\partial U}{\partial \varphi}\vec{i_\varphi}\\
\nabla\cdot \vec{A} =& \frac{1}{r^2}\frac{\partial{(r^2 A_r)}}{\partial r} +\frac{1}{r\sin\theta}\frac{\partial{(A_\theta \sin\theta)}}{\partial\theta}+\frac{1}{r\sin\theta}\frac{\partial{A_\varphi}}{\partial \varphi} \\
\nabla\times \vec{A}=& \left(\frac{1}{r\sin\theta}\frac{\partial{(A_\varphi \sin\theta)}}{\partial \theta} -\frac{1}{r\sin\theta}\frac{\partial{A_\theta}}{\partial \varphi}\right)\vec{i_r} \notag \\
& + \left(\frac{1}{r\sin\theta}\frac{\partial{A_r}}{\partial \varphi} -\frac{1}{r}\frac{\partial{(r A_\varphi)}}{\partial r}\right) \vec{i_\theta}
+ \left(\frac{1}{r}\frac{\partial{(r A_\theta)}}{\partial r} -\frac{1}{r}\frac{\partial{A_r}}{\partial \theta}\right)\vec{i_\varphi}  \\
\nabla^2 U=&\frac{1}{r^2}\frac{\partial}{\partial r}\left(r^2\frac{\partial U}{\partial r}\right)+
\frac{1}{r^2\sin\theta}\frac{\partial}{\partial\theta}\left(\sin\theta\frac{\partial U}{\partial \theta}\right)+\frac{1}{r^2\sin^2\theta}\frac{\partial^2 U}{\partial \varphi^2}
  	\end{align}
\end{subequations}


\subsection{Legendre函数}
方程2.33c中的$P_n^m(\cos\theta)$在$m=0$时被称为$n$阶Legendre函数,即$P_n^0(\cos\theta)\equiv P_n(\cos\theta)\equiv P_n(u)$(此处
令$u=\cos\theta$)。在$1\le m \le n$时,被称为伴随Legendre函数。$P_n^m(\cos\theta)\equiv P_n^m(u)$满足下面的关于$v$微分方程:
\begin{equation}\label{eqn:legendre diff}
  \frac{d}{du}\left[(1-u^2)\frac{dv}{du}\right]+\left[n(n+1)-\frac{m^2}{1-u^2}\right]v=0
\end{equation}

如前所述,$P_n(u)$在解具有\textbf{旋转对称性}的磁场(比如\textit{理想}螺管)时特别有用;
$P_n^m(u)$则在\textit{实际}螺管(非理想、缺少旋转对称性)的磁场分析中十分有用。
\begin{subequations}\label{eqn:legendre function1}
	\begin{align}
P_n(u)=&\left(\frac{1}{2^n n!}\right)\frac{d^n(u^2-1)^n}{du^n} \\
P_n^m(u)=&(1-u^2)^{m/2}\frac{d^mP_n(u)}{du^m}\\
=&\left[\frac{(1-u^2)^{m/2}}{2^n n!}\right]\frac{d^{m+n}(u^2-1)^n}{du^{m+n}}
  	\end{align}
\end{subequations}

几个有用的Legendre函数以及伴随Legendre函数的递归形式如下:
\begin{subequations}\label{eqn:legendre function2}
	\begin{align}
(n+1)P_{n+1}(u)-(2n+1)uP_n(u)+nP_{n-1}(u)=&0  \\
(n+1-m)P^m_{n+1}(u)-(2n+1)uP^m_n(u)+(n+m)P^m_{n-1}(u)=&0  \\
P_n^{m+2}(u)-\frac{2(m+1)u}{\sqrt{(1-u^2)}}P_n^{m+1}+(n-m)(n+m-1)P_n^m(u)=&0
  	\end{align}
\end{subequations}

表2.2列出了$n$取至8的Legendre函数$P_n(u)$以及$m(1\le m\le n)$取至4的的伴随Legendre函数$P_n^m(u)$。
表2.3给出了$n$=2,4,6,8,10和$m$=0,2,4,6,8,10组合下的$P_n^m(0)$。
表2.4给出了方程2.32在笛卡尔坐标系下的解。

\begin{quotation}
\textbf{勒让德小传}\ \kaishu {勒让德(Adrien-Marie Legendre,1752年9月18日-1833年1月10日)的主要贡献在统计学、数论、抽象代数、数学分析、初等几何与天体力学。勒让德是椭圆积分理论奠基人之一,是解析数论的先驱者之一,对数论的主要贡献是二次互反律,归纳出了素数分布律。其他贡献包括:椭圆函数论、最小二乘法、测地线理论等。拉格朗日(Lagrange)、拉普拉斯(Laplace)和勒让德(Legendre)三位的姓氏的第一个字母为“L”,又生活在同一时代,所以人们称他们为“三L”。Legendre的大部分工作与比他小二十岁的Gauss重合,两者分别独立的得到了结果,但Gauss的天才光环,让后世的人更喜欢把功劳归功到Gauss头上。正可谓``既生瑜,何生亮''。}
\end{quotation}

\begin{table}[htbp]\small
  \centering
  \caption{Legendre函数和伴随Legendre函数} 
\begin{tabular}{|l|l|l|}
\hline
\multicolumn{3}{|c|}{Legendre Functions(m=0)$\left[P_n(0) \ \mathrm{for} \ n \ \mathrm{odd}\right]$} \\ \hline
n & $P_n(\cos\theta)\equiv P_n(u)$ & Miltiple-Angle Form \\ \hline
0 & 1 & 1 \\ \hline
1 & u & $\cos\theta$ \\ \hline
2 & $\frac{1}{2}(3u^2-1)$ & $\frac{1}{4}(3\cos2\theta+1)$ \\ \hline
3 & $\frac{1}{2}(5u^3-3u)$ & $\frac{1}{8}(5\cos3\theta+3\cos\theta)$ \\ \hline
4 & $\frac{1}{8}(35u^4-30u^2+3)$ & $\frac{1}{68}(35\cos4\theta+20\cos2\theta+9)$ \\ \hline
5 & $\frac{1}{8}(63u^5-70u^3+15u)$ & $\frac{1}{128}(63\cos5\theta+35\cos3\theta+30\cos\theta)$ \\ \hline
6 & $\frac{1}{16}(231u^6-315u^4+105u^2-5)$ & $\frac{1}{512}(231\cos6\theta+126\cos4\theta+105\cos2\theta+50)$ \\ \hline
7 & $\frac{1}{16}(429u^7-639u^5+315u^3-35u)$ &\begin{tabular}[c]{@{}l@{}}$\frac{1}{1024}(429\cos7\theta+231\cos5\theta+189\cos3\theta$\\ $+175\cos\theta)$\end{tabular} \\ \hline
8 & $\frac{1}{128}(6435u^8-12012u^6+6930u^4-1260u^2+35)$ &\begin{tabular}[c]{@{}l@{}}$\frac{1}{16384}(6435\cos8\theta+3432\cos6\theta
+2772\cos4\theta$\\ $+2520\cos2\theta+1225)$\end{tabular} \\ \hline
\multicolumn{3}{|c|}{Associated Legendre Functions$(1\leq m\leq n)\left[P_{n}^{m}(1)=0\right]$} \\ \hline
m,n & $P_{n}^{m}(\cos\theta)\equiv P_{n}^{m}(u)$ & Multiple-Angle Form \\ \hline
\multicolumn{1}{|r|}{\multirow{4}{*}{\begin{tabular}[c]{@{}r@{}}1,1\\ 2\\ 3\\ 4\end{tabular}}} & $(1-u^2)^{1/2}$ & $\sin\theta$ \\ \cline{2-3} 
\multicolumn{1}{|r|}{} & $3u(1-u^2)^{1/2}$ & $\frac{3}{2}\sin2\theta$ \\ \cline{2-3} 
\multicolumn{1}{|r|}{} & $\frac{3}{2}(1-u^2)^{1/2}(5u^2-1)$ & $\frac{3}{8}(\sin\theta+5\sin3\theta)$ \\ \cline{2-3} 
\multicolumn{1}{|r|}{} & $\frac{5}{2}(1-u^2)^{1/2}(7u^3-3u)$ & $\frac{5}{16}(2\sin2\theta+7\sin4\theta)$ \\ \hline
\multicolumn{1}{|r|}{\multirow{3}{*}{\begin{tabular}[c]{@{}r@{}}2,2\\ 3\\ 4\end{tabular}}} & $3(1-u^2)$ & $\frac{3}{2}(1-\cos2\theta)$ \\ \cline{2-3} 
\multicolumn{1}{|r|}{} & $15(1-u^2)u$ & $\frac{15}{4}(\cos\theta-\cos3\theta)$ \\ \cline{2-3} 
\multicolumn{1}{|r|}{} & $\frac{15}{2}(1-u^2)(7u^2-1)$ & $\frac{15}{16}(3+4\cos2\theta-7\cos4\theta)$ \\ \hline
\multicolumn{1}{|r|}{\multirow{2}{*}{\begin{tabular}[c]{@{}r@{}}3,3\\ 4\end{tabular}}} & $15(1-u^2)^{3/2}$ & $\frac{15}{4}(3\sin\theta-\sin3\theta)$ \\ \cline{2-3} 
\multicolumn{1}{|r|}{} & $105(1-u^2)^{3/2}u$ & $\frac{105}{8}(2\sin2\theta-\sin4\theta)$ \\ \hline
4,4 & $105(1-u^2)^2$ & $\frac{105}{8}(3-4\cos2\theta+\cos4\theta)$ \\ \hline
\end{tabular}
\end{table}

\begin{table}[htbp]\small
  \centering
  \caption{$m=0,2,4,6,8,10$对应的$P_n^m(0)$值} 
\begin{tabular}{|c|l|}
\hline
m & $P_{n}^{m}(0)$ \\ \hline
0 & $P_2(0)=-\frac{1}{2};\ P_4(0)=\frac{3}{8};\ P_6(0)=-\frac{5}{16};\ P_8(0)=\frac{35}{128};\ P_10(0)=-\frac{63}{256}$ \\ \hline
2 & $P_{2}^{2}(0)=3;\ P_{4}^{2}(0)=-\frac{15}{2};\ P_{6}^{2}(0)=\frac{105}{8};\ P_{8}^{2}(0)=-\frac{315}{16};\ P_{10}^{2}(0)=\frac{3465}{128}$ \\ \hline
4 & $P_{4}^{4}(0)=3\cdot5\cdot7=105;\ P_{4}^{6}(0)=-\frac{945}{2};\ P_{4}^{8}(0)=\frac{10395}{8};\ P_{4}^{10}(0)=-\frac{45045}{16} $ \\ \hline
6 & $P_{6}^{6}(0)=3\cdot5\cdot7\cdot9\cdot11=10395;\ P_{6}^{8}(0)=-\frac{135135}{2};\ P_{6}^{10}(0)=\frac{2027025}{8} $ \\ \hline
8 & $P_{8}^{8}(0)=3\cdot5\cdot7\cdot9\cdot11\cdot13\cdot15=2027025;\ P_{8}^{10}(0)=-\frac{3\cdot5\cdot7\cdot9\cdot11\cdot13\cdot15\cdot17}{2}=-\frac{34459425}{2}$ \\ \hline
10 & $P_{10}^{10}(0)=3\cdot5\cdot7\cdot9\cdot11\cdot13\cdot15\cdot17\cdot19=654729075$ \\ \hline
\end{tabular}
\end{table}


\begin{table}[htbp]\small
  \centering
  \caption{方程2.32在笛卡尔坐标系下的解} 
\begin{tabular}{|c|l|l|}
\hline
\multicolumn{3}{|c|}{Legendre Functions(m=0)} \\ \hline
n & \multicolumn{2}{l|}{$r^nP_n(u)$} \\ \hline
0 & \multicolumn{2}{l|}{1} \\ \hline
1 & \multicolumn{2}{l|}{z} \\ \hline
2 & \multicolumn{2}{l|}{$z^2-\frac{1}{2}(x^2+y^2)$} \\ \hline
3 & \multicolumn{2}{l|}{$z^3-\frac{3}{2}(x^2+y^2)z$} \\ \hline
4 & \multicolumn{2}{l|}{$z^4-3(x^2+y^2)\left[z^2-\frac{1}{8}(x^2+y^2)\right]$} \\ \hline
5 & \multicolumn{2}{l|}{$z^5-5(x^2+y^2)\left[z^2-\frac{3}{8}(x^2+y^2)\right]z$} \\ \hline
6 & \multicolumn{2}{l|}{$z^6-\frac{5}{2}(x^2+y^2)\left[3z^4-\frac{9}{4}z^2(x^2+y^2)+\frac{1}{8}(x^2+y^2)^2\right]$} \\ \hline
7 & \multicolumn{2}{l|}{$z^7-\frac{7}{16}(x^2+y^2)\left[24z^4-30z^2(x^2+y^2)+5(x^2+y^2)^2\right]z$} \\ \hline
8 & \multicolumn{2}{l|}{$z^8-\frac{7}{128}(x^2+y^2)\left[256z^6-480z^4(x^2+y^2)+160z^2(x^2+y^2)-5(x^2+y^2)^3\right]$} \\ \hline
\multicolumn{3}{|c|}{Associated Legendre Functions($1\leq m\leq n$)} \\ \hline
\multicolumn{1}{|l|}{m,n} & $r^nP_{n}^{m}(u)\cos(m\varphi)$ &$r^nP_{n}^{m}(u)\sin(m\varphi)$ \\ \hline
\multicolumn{1}{|r|}{\multirow{4}{*}{\begin{tabular}[c]{@{}r@{}}1,1\\ 2\\ 3\\ 4\end{tabular}}} & x & y \\ \cline{2-3} 
\multicolumn{1}{|r|}{} & 3zx & 3zy \\ \cline{2-3} 
\multicolumn{1}{|r|}{} & $6x\left[z^2-\frac{1}{4}(x^2+y^2)\right]$ & $6y\left[z^2-\frac{1}{4}(x^2+y^2)\right]$ \\ \cline{2-3} 
\multicolumn{1}{|r|}{} & $10zx\left[z^2-\frac{3}{4}(x^2+y^2)\right]$ & $10zy\left[z^2-\frac{3}{4}(x^2+y^2)\right]$ \\ \hline
\multicolumn{1}{|r|}{\multirow{3}{*}{\begin{tabular}[c]{@{}r@{}}2,2\\ 3\\ 4\end{tabular}}} & $3(x^2-y^2)$ & 3(2xy) \\ \cline{2-3} 
\multicolumn{1}{|r|}{} & $15z(x^2-y^2)$ & 15z(2xy) \\ \cline{2-3} 
\multicolumn{1}{|r|}{} & $45(x^2-y^2)\left[z^2-\frac{1}{6}(x^2+y^2)\right]$ &$45(2xy)\left[z^2-\frac{1}{6}(x^2+y^2)\right]$ \\ \hline
\multicolumn{1}{|r|}{\multirow{2}{*}{\begin{tabular}[c]{@{}r@{}}3,3\\ 4\end{tabular}}} & $15x(x^2-3y^2)$ & $15y(3x^2-y^2)$ \\ \cline{2-3} 
\multicolumn{1}{|r|}{} & $105zx(x^2-3y^2)$ & $105zy(3x^2-y^2)$ \\ \hline
\multicolumn{1}{|r|}{4,4} & $105(x^3-6x^2y^2+y^4)$ & $210(2xy)(x^2-y^2)$ \\ \hline
\end{tabular}
\end{table}


\section{专题}
\subsection{问题2.1:均匀场中的磁球}
本题处理一个置于均匀外磁场中的磁球。由于背景场是均匀的,故球的净受力为0。
球内的场表达式可以用来估算磁体的边缘场对附近铁磁物体的作用力。
磁体边缘场对铁制物体的作用力是第3章讨论3.12的主题。

如图2.1所示的一个磁球,半径是$R$,磁导率是$\mu$,置于均匀外磁场中:
\begin{equation}\label{eqn:2.40}
  \vec{H}_\infty = H_0 (-\cos \theta\vec{i_r}+\sin\theta\vec{i_\theta})
\end{equation}

a) 证明磁感应强度在磁球内($B_2$)、外($B_1$)分别为:
\begin{subequations}
	\begin{align}
  \vec{B_1} =& \mu_0 H_0 (-\cos\theta\vec{i_r}+\sin\theta\vec{i_\theta})\notag\\ 
          &+\mu_0 \left(\frac{\mu_0-\mu}{2\mu_0+\mu}\right)H_0 \left(\frac{R}{r}\right)^3 (2\cos\theta\vec{i_r}+\sin\theta\vec{i_\theta})\\ 
  \vec{B_2}=& \frac{3\mu_0\mu H_0}{2\mu_0+\mu} (-\cos\theta_{\vec{i_r}}+\sin\theta_{\vec{i_\theta}})
  	\end{align}
\end{subequations}

考虑以下三种极限情况:$\mu/\mu_0=0, \mu/\mu_0=1$和$\mu/\mu_0=\infty$。
确认结果表达式在球内是有物理意义的。

b) 在头脑中想象出$\mu=0.1\mu_0$和$\mu=100\mu_0$时$\vec{B}$的场分布,然后和图2.2对照。

\begin{figure}[htbp]
  \centering
 \includegraphics[scale=0.8]{chpt2/figs/fig2.1.eps}
  \caption{均匀磁场中的磁化球}
\end{figure}

\subsubsection*{问题2.1之解答}
a) 本题用标量势来解是最简单的。磁标量势表示为
$$\vec{H}=-\nabla \phi \eqno (S1.1)$$

对线性、各向同性介质,磁场和磁感应强度的关系为
$$\vec{B}=\mu \vec{H} \eqno (S1.2)$$

问题可以分为两个区域:区域1($r\ge R$)和区域2($r\le R$)。因为在$r=\infty$时,$H_\infty=H_0$,
在$r=0$时,$H\neq \pm \infty$,两个区域的标量势可分别选为:
	\begin{align}
\phi_1&=H_0 r\cos\theta+\frac{A}{r^2}\cos\theta &(r\ge R)\tag{S1.3a}\\
\phi_2&=C r \cos\theta &(r\le R)\tag{S1.3b}
  	\end{align}

注意到,在$r\rightarrow \infty$时,$\phi_1\rightarrow H_0 r \cos\theta$;在$r=0$时,$\phi_2$是有限的。

应用球坐标系下的$\nabla$算符(式2.36a),我们有:
	\begin{align}
\vec{H}_1&= H_0(−\cos\theta\vec{i_r} + \sin\theta\vec{i}_\theta) + \frac{A}{r^3}(2 \cos\theta\vec{i_r} + \sin\theta\vec{i_\theta})\tag{S1.4a}\\
\vec{H}_2&=C(-\cos\theta\vec{i_r} + \sin\theta\vec{i}_\theta) \tag{S1.4b}
  	\end{align}

\textbf{边界条件}

1)在$r=R$处,因无表面电流,故$\vec{H}$的切向分量($\vec{i_\theta}$)连续。这等价于在$r=R$时,有$\phi_1=\phi_2$:
$$H_0 + \frac{A}{R^3} = C \eqno (S1.5)$$

2)在$r=R$处,$\vec{B}$的法向分量($\vec{i_r}$)连续:
$$\mu_0\left(-H_0 + 2\frac{A}{R^3} \right)= -\mu C \eqno (S1.6)$$

从式S1.5和式S1.6,解出$A$和$C$两个常数:
$$C =\frac{3H_0\mu_0}{2\mu_0+\mu}\eqno (S1.7)$$
$$A=(C-H_0)R^3=H_0\left(\frac{\mu_0-\mu}{2\mu_0+\mu}\right)R^3 \eqno (S1.8)$$

于是,$\vec{B}_1$和$\vec{B}_2$可写为:
	\begin{align}
\vec{B}_1=& \mu_0 H_0(−\cos\theta\vec{i_r} + \sin\theta\vec{i_\theta}) +\mu_0\left(\frac{\mu_0-\mu}{2\mu_0+\mu}\right)H_0\left(\frac{R}{r}\right)^3(2 \cos\theta\vec{i_r} + \sin\theta\vec{i_\theta})\tag{2.41a}\\
\vec{B}_2=&\frac{3\mu_0\mu H_0}{2\mu_0+\mu}(-\cos\theta\vec{i_r} + \sin\theta\vec{i_\theta}))\tag{2.41b}
  	\end{align}

下面,我们考虑三种$\mu/\mu_0$的三种特例。

\textbf{特例1:$\mu/\mu_0=0$}

将$\mu=0$代入式2.41,可得

	\begin{align}
\vec{B}_1&= \mu_0 H_0(−\cos\theta\vec{i_r} + \sin\theta\vec{i_\theta}) +
\frac{\mu_0 H_0}{2}\left(\frac{R}{r}\right)^3 (2 \cos\theta\vec{i_r} + \sin\theta\vec{i_\theta})\tag{S1.9a}\\
\vec{B}_2&=0 \tag{S1.9b}
  	\end{align}

这个球像第I类超导体;球内不允许有磁通密度存在---Meissner效应。
磁通和图1.1b中超导体的点C是一致的。
问题2.2将会讨论到,$\vec{H}$在
$r=R$处的$\theta$分量不连续将要求存在表面电流(被限制在一个薄层内)。
因为这个电流一旦建立起来,就必须一直流下去,这就表明了球对电流是无电阻的。
如第1章所言的,存在Meissner效应的材料必须同时是理想导体:那这种材料其实就是超导体。

\textbf{特例2:$\mu/\mu_0=1$}

这时问题退化为平凡解,即等价于没有球的存在。代入式2.41,两个方程将退化为一个方程。

\textbf{特例3:$\mu/\mu_0=\infty$}

这时表示的是一个理想铁磁材料;软铁的性质与此近似。
磁场被``吸"入球内。将$\mu=\infty$代入式2.41,有:
	\begin{align}
\vec{B}_1&= \mu_0 H_0(−\cos\theta\vec{i_r} + \sin\theta\vec{i_\theta}) -
\mu_0 H_0\left(\frac{R}{r}\right)^3 (2 \cos\theta\vec{i_r} + \sin\theta\vec{i_\theta})\tag{S1.10a}\\
\vec{B}_2&=3\mu_0 H_0(−\cos\theta\vec{i_r} + \sin\theta\vec{i_\theta})\tag{S1.10b}
  	\end{align}

我们发现,此时球内的$\vec{B}$是外施磁场$\vec{B}$的3倍。
如果球存在磁饱和,则$\mu$不再是$\infty$。
所有磁性材料都存在的磁饱和效应将在问题2.3中讨论。

b) 图2.2给出了$\mu/\mu_0=0.1$和$\mu/\mu_0=100$两种情况下的磁场线。

\begin{figure}[htbp]
  \centering
 \includegraphics[scale=0.6]{chpt2/figs/fig2.2.eps}
  \caption{$\mu/\mu_0=0.1$(上)和$\mu/\mu_0=100$(下)两种情况下球内和球附近的磁场线。上球在$r=\infty$处的磁场线密度是球内的$\sim 2.6(=\sqrt{2.1/0.3})$倍;下球的球内磁场比无穷远处密$\sim 1.7(=\sqrt{300/101})$倍。这两个比值不仅在纸平面成立,在垂直纸面方向也成立。$\mu/\mu_0=100$的圆柱体的比值为200/101,
  但仅在纸平面成立。我们看到,在$\mu/\mu_0=100$时,进出球的场线\textit{几乎}与球表面垂直---
  $\mu/\mu_0=\infty$时完全垂直。}
\end{figure}

\subsection{问题2.2:均匀场中的第I类超导棒}
本题研究第I代超导体的Meissner效应,并用London超导理论对结果进行解释。

\begin{figure}[htbp]
	\centering
	\includegraphics[scale=0.4]{chpt2/figs/fig2.3.eps}
	\caption{无限长、圆截面超导棒置于均匀磁场中}
\end{figure}

如图2.3,无限长、圆截面(半径$R$)的Pb棒置于垂直于其轴的均匀外磁场中。
\begin{equation*}\label{eqn:2.40}
  \vec{H}_\infty = H_0 (-\cos \theta\vec{i_r}+\sin\theta\vec{i_\theta}) \tag{2.40}
\end{equation*}
式中,$\mu_0 H_0=0.08\ \mathrm{T}$。初始棒处于\textit{正常}态,磁场在棒内外处处相同。
然后,将Pb棒逐步冷却直至超导。

a)证明磁场暂态效应消失后,棒外磁场为:
\begin{equation}
  \vec{H}_1=H_0(-\cos\theta\vec{i_r}+\sin\theta\vec{i_\theta})+H_0\left(\frac{R}{r}\right)^2 (\cos\theta\vec{i_r}+\sin\theta\vec{i_\theta})
\end{equation}

b) 证明在穿透深度$\lambda\ll R$内的面(自由)电流密度$K_f [\mathrm{A/m}]$为:
\begin{equation}
  \vec{K}_f=2H_0 \sin\theta\vec{i_z}
\end{equation}

c)将面电流密度幅值折算为(体)电流密度$J_f [\mathrm{A/m^2}]$。
证明这个值与用London超导理论的计算值一致。

\subsubsection*{问题2.2之解答}
a) 问题分为两个区域考虑:区域1($r\ge R$)和区域2($r\le R$)。
因为我们处理的是第I类超导体,故圆棒超导时有$\vec{B}_2=0 (\phi_2=0)$。
区域1的磁场可由势函数导出:
$$  \phi_1=H_0 r \cos \theta +\frac{A}{r} \cos \theta \eqno(S2.1)$$
注意到,当$r\rightarrow \infty $时,$\phi_1\rightarrow H_0 r \cos\theta$。

使用圆柱坐标系下的$\nabla$算符(式2.35a),可以得到$\vec{H_1}$:
\begin{align}
\vec{H}_1=&-\frac{\partial}{\partial r}(H_0 r \cos\theta +\frac{A}{r}\cos\theta)\vec{i}_r-\frac{1}{r}\frac{\partial}{\partial \theta}(H_0 r \cos\theta +\frac{A}{r}\cos\theta)\vec{i}_\theta \tag{S2.2}\\
=&-(H_0 \cos\theta +\frac{A}{r^2}\cos\theta)\vec{i}_r-(-H_0 \sin\theta -\frac{A}{r^2}\sin\theta)\vec{i}_\theta \tag{S2.3}
\end{align}

整理式S2.3,可得
\begin{equation*}
\vec{H}_1=H_0(-\cos \theta \vec{i}_r+\sin \theta \vec{i}_\theta) +\frac{A}{r^2}(\cos\theta \vec{i}_r+\sin\theta \vec{i}_\theta )\tag{S2.4}
\end{equation*}

$B$法向分量的连续性要求式S2.3中$\vec{i}_r$的系数在$r=R$处为零:
\begin{equation*}
-H_0+\frac{A}{R^2}=0  \tag{S2.5}
\end{equation*}

解出$A$,得:
\begin{equation*}
A=R^2 H_0 \tag{S2.6}
\end{equation*}

于是,超导圆棒外的磁场(区域1)为:
\begin{equation*}
\vec{H}_1=H_0(-\cos \theta \vec{i}_r+\sin \theta \vec{i}_\theta) +H_0 \left(\frac{R}{r}\right)^2(\cos\theta \vec{i}_r+\sin\theta \vec{i}_\theta ) \tag{2.42}
\end{equation*}

注意到,在$r=R,\theta=90^\circ$时,$|\vec{H}_1|=2H_0$;也即此处的幅值是远场的两倍。物理上看,圆棒正常态时其内部本来存在磁场,进入超导态后,磁场被``推"出,
``堆积"在$\theta=90^\circ$附近。


b) 因为$2H_0\sin \theta$中的$\vec{H}$在$r=R$处的切向分量不连续,所以肯定存在表面电流密度$\vec{K}_f$流过超导棒(式2.6)。我们有:

\begin{equation*}
  \vec{K}_f=\vec{i}_r \times 2H_0\sin\theta \vec{i}_\theta=2H_0\sin\theta\vec{i}_z\tag{2.43}
\end{equation*}

该正弦电流分布(余弦更常用)是高能物理设备``核粒子加速器"内所用的大多数偶极磁体的基础。
问题3.8中将研究一个\textit{理想}偶极磁体。


c) 根据超导电性的London理论(第1章的1.2.2部分),第I类超导体的超导电流密度$J_s=en_{se}v$。其中,$e$是电子电荷量($e=1.6\times 10^{-19}\ \mathrm{C}$),$n_{se}$是超导电子的密度,$v$是超导电子的漂移速度。
超导电子的密度\textit{大致}上等于自由电子的密度:
\begin{equation*}
n_{se}\simeq n_{fe}=\frac{\rho N_A}{W_A} \tag{1.2}
\end{equation*}

对于Pb,$\rho=11.4\ \mathrm{g/cm^3};N_A=6.023\times 10^{23}\ \mathrm{particle/mole};W_A=207.2\ \mathrm{g/mole}$,我们得:
\begin{equation*}
n_{se}=\frac{11.4\times 6.023\times 10^{23}}{207.2} \simeq 3.3\times 10^{28}\ \mathrm{electron/m^3} \tag{S2.7}
\end{equation*}

超导电子的漂移速度$v\sim 200\ \mathrm{m/s}$,我们可以确定$J_s$的\textit{量级}:
\begin{equation*}
J_s=e n_{se} v\sim 1.6\times 10^{-19} \times 3.3\times 10^{28} \times 200 \sim 1\times 10^{12}\ \mathrm{A/m^2} 
\end{equation*}

在上述铅圆柱棒所要求的表面电流密度($K_f=J_f \lambda$)下,$J_f$应\textit{大致}上与$J_s$相等,也即与上面所计算的$J_s$在同一个量级上。
London理论给出了超导电流流过超导体表面的穿透深度$\lambda$:
\begin{equation*}
\lambda=\sqrt{\frac{m}{\mu_0 e^2 n_{se}}} \tag{1.1}
\end{equation*}
其中,$n_{se}$如式S2.7,$m=9.1\times 10^{-31}\ \mathrm{kg}$,则:
\begin{equation*}
\lambda=\sqrt{\frac{9.1\times 10^{-31}}{4\pi \times 10^{-7}\times 1.6\times 10^{-19}\times 3.3\times 10^{28}}}\simeq 3\times 10^{-8}\ \mathrm{m}
\end{equation*}

由于$K_f=J_f\lambda$,则:
\begin{equation*}
J_f=\frac{2H_0}{\lambda}=\frac{2\mu_0 H_0}{\mu_0 \lambda}\simeq 4\times 10^{12} \ \mathrm{A/m^2}  \tag{S2.9}
\end{equation*}

也即,$J_f$与$J_s$\textit{大致}上相等。



\subsection{讨论2.1:均匀场中的理想导体球}
本题我们推导理想导体($\rho=0$)球置于磁场中的磁场定量表达式。
特别是针对图 1.1c中的$C\Rightarrow D$转变过程。
点C处,包括球内的整个空间处于均匀外场中(问题2.1的式2.40)。
我们使用和问题2.1中图2.1相同的球坐标系。

当均匀外场减小至零(图1.1c中的D点),由于球内磁感应强度无法改变,其磁场强度$\vec{H_2}$必保持不变。
远离球中心($r\rightarrow \infty$)处的外磁场$\vec{H_1}$为零。
靠近球处的磁场可由标量势$\phi_1=(A/r^2)\cos\theta$导出。
在$\phi_1$上应用$\nabla$算符,得到就是球坐标下的偶极场。
\begin{subequations}
	\begin{align}
\vec{H}_2=&H_0 (-\cos\theta \vec{i_r}+\sin\theta \vec{i_\theta})  \qquad(r\le R) \\
\vec{H}_=&\frac{A}{r^3}(2\cos\theta \vec{i_r}+\sin\theta \vec{i_\theta})\qquad  (r\ge R)
	\end{align}
\end{subequations}

在$r=R$处,$B$的法向分量连续。由于在$\theta=0$处球内外的磁场均仅有法向分量,故:
\begin{equation*}
-H_0=2\frac{A}{R^3}
\end{equation*}

于是,$A=-H_0 R^3/2$。进而:
\begin{equation*}
\vec{H}_1=-H_0\left(\frac{R}{r}\right)^3 \left(\cos\theta \vec{i_r}+\frac{1}{2}\sin\theta \vec{i_\theta}\right) \tag{2.44c}
\end{equation*}


式2.44c即磁场分布。不同的是,图1.1c的外场是自下向上的,而图2.1的磁场是自右向左的。

磁场切向分量在$r=R$处的不连续等于在$r=R$处的球面上存在表面电流密度$\vec{K}_f$。
联立式2.6、2.44a和2.44c,有:
\begin{equation}
  \vec{K}_f=\vec{i_r}\times \left[H_0 \sin\theta \vec{i_\theta}-\left(-\frac{1}{2}H_0 \sin\theta \vec{i_\theta}\right)\right]=\frac{3}{2}H_0\sin\theta\vec{i_\theta}
\end{equation}

此$\sin\theta$式分布可以用一个绕在球面上的\textit{在$z$轴方向具有均匀匝密度}的``薄"线圈来近似。




\subsection{问题2.3:球壳的磁屏蔽}
本问题处理被动磁屏蔽,该问题在MRI、磁悬浮列车、人和磁敏感设备可能会暴露于其外场中的高场系统中是非常重要的。美国食药局(FDA)规定,MRI中的最大外缘磁场为5 gauss($0.5\ \mathrm{mT}$)。

在空间中的球形区域内,需要被屏蔽均匀磁场$\vec{H}_\infty$为:
\begin{equation*}
\vec{H}_\infty=H_0 (-\cos\theta \vec{i_r}+\sin\theta \vec{i_\theta})\tag{2.40}
\end{equation*}

为了实现屏蔽,可以使用外径$2R$、壁厚$d/R\ll 1$、高磁导率($\mu/\mu_0 \gg 1$)材料的球壳。
示意如图2.4。

a) 将本问题视为均匀外磁场中的磁性球壳。记$H_{ss}$为球壳内($r\le R-d$)的磁场幅值,
证明$H_{ss}/H_0$为:
\begin{equation}
\frac{H_{ss}}{H_0}=\frac{9\mu_0 \mu}{9\mu_0 \mu+2(\mu-\mu_0)^2\left[1-\left(1-\frac{d}{R}\right)^3\right]}
\end{equation}

b) 证明,在$\mu/\mu_0\gg 1$和$d/R\ll 1$的极限情况下,式2.46给出的$H_{ss}/H_0$可约化为:
\begin{equation}
\frac{H_{ss}}{H_0}\simeq \frac{3}{2}\left(\frac{\mu_0}{\mu}\right)\left(\frac{R}{d}\right)
\end{equation}

c) 接下来,用微扰法导出上式。首先,在$\mu=\infty$条件下解出壳体中($R-d\le r\le R$)的磁场。
然后,用微扰法在$\mu/\mu_0 \gg 1$条件下得到式2.47。

d) 实践中,屏蔽材料中的磁通必须控制在材料的饱和磁通($\mu_0 M_{sa}$)之下。
证明壳体不饱和的$d/R$条件为:
\begin{equation}
\frac{d}{R} \ge \frac{3H_0}{2M_{sa}}
\end{equation}

e)画出$\mu/\mu_0 \gg 1$情况下的磁力线。

\begin{figure}[htbp]
  \centering
 \includegraphics[scale=0.9]{chpt2/figs/fig2.4.eps}
  \caption{均匀磁场中的磁性球壳}
\end{figure}

\subsubsection*{问题2.3之解答}
a) 将问题分为三个区域:区域1($r\ge R$);区域2(球壳);区域3($r\le R-d$)。相应的势函数分别为:
\begin{align}
  \phi_1 =& H_0 r\cos\theta+\frac{A}{r^2}\cos\theta \tag{S3.1a} \\
  \phi_2 =& C r \cos\theta+\frac{D}{r^2}\cos\theta\tag{S3.1b} \\
  \phi_3 =& H_{ss} r\cos\theta  \tag{S3.1c}
\end{align}

可见,在$r\rightarrow \infty$时,$\phi_1 \rightarrow H_0 r\cos\theta$;
$r\rightarrow 0$时,$\phi_3$为有限值。

应用球坐标下的$\nabla$算符,得:
\begin{align}
  \vec{H}_1 =& H_0 (-\cos\theta\vec{i_r}+\sin\theta\vec{i}_\theta)+\frac{A}{r^3} (2\cos\theta\vec{i_r}+\sin\theta\vec{i}_\theta) \tag{S3.2a}\\
  \vec{H}_2 =& C(-\cos\theta\vec{i_r}+\sin\theta\vec{i}_\theta)+\frac{D}{r^3} (2\cos\theta\vec{i_r}+\sin\theta\vec{i}_\theta)  \tag{S3.2b}\\
   \vec{H}_3 =& H_{ss}  (-\cos\theta\vec{i_r}+\sin\theta\vec{i}_\theta)  \tag{S3.2c}
\end{align}

\noindent\textbf{边界条件:}

\begin{enumerate}
  \item $r=R$处,$\vec{H}$的切向分量$H_\theta$连续:$\phi_1=\phi_2$。
  \item 类似的,在$r=R-d$处,$H_\theta$连续:$\phi_2=\phi_3  $。
  \item 在$r=R$处,$\vec{B}$的法向分量$B_r$连续。
  \item 类似的,在$r=R-d$处,$B_r$连续。
\end{enumerate}

上述边界条件给出下面的四个等式:
\begin{align}
  &H_0+\frac{A}{R^3}= C+\frac{D}{R^3} \nonumber \tag{S3.3a}\\
  &C+\frac{D}{\left(R-d\right)^3}= H_{ss} \nonumber \tag{S3.3b}\\
  &\mu_0\left(-H_0+\frac{2A}{R^3}\right)= \mu\left(-C+\frac{2D}{R^3}\right)\nonumber \tag{S3.3c} \\
  &\mu\left[-C+\frac{2D}{(R-d)^3}\right]= -\mu_0 H_{ss} \nonumber\tag{S3.3d}
\end{align}

联立式S3.3a和式S3.3b,消去$C$:
\begin{equation*}
\frac{A}{R^3}+D\left[\frac{1}{(R-d)^3}-\frac{1}{R^3}\right]-H_{ss} =-H_0 \tag{S3.4}
\end{equation*}


联立式S3.3b和式S3.3d,解出用$H_{ss}$表示的$D$:
\begin{equation*}
D=\frac{\mu-\mu_0}{2\mu} \left(R-d\right)^3 H_{ss}  \tag{S3.5}
\end{equation*}

联立式S3.4和式S3.5,得到以$H_{ss}$表示的$A/R^3$:
\begin{equation*}
\frac{A}{R^3}=H_{ss}\left\{1-\frac{\mu-\mu_0}{3\mu}\left[1-\left(1-\frac{d}{R}\right)^3\right]\right\}-H_0   \tag{S3.6}
\end{equation*}

由式S3.3c和式S3.3d,得到:
\begin{equation*}
\frac{2A}{R^3}+2\frac{\mu}{\mu_0}D\left[\frac{1}{(R-d)^3}-\frac{1}{R^3}\right]+H_{ss} =H_0  \tag{S3.7}
\end{equation*}

联立式S3.3-S3.7,用$H_0$表达$H_{ss}$,可得:
\begin{equation*}
\frac{H_{ss}}{H_0}=\frac{9\mu_0 \mu}{9\mu_0 \mu+2(\mu-\mu_0)^2\left[1-\left(1-\frac{d}{R}\right)^3\right]}  \tag{2.46}
\end{equation*}

b) 式2.46右侧的分子分母同除$\mu_0^2$,应用极限$\mu/\mu_0\gg 1$和$d/R\ll 1$:
\begin{align}
\frac{H_{ss}}{H_0}\simeq& \frac{9\mu/\mu_0}{9\left(\frac{\mu}{\mu_0}\right)+2\left(\frac{\mu}{\mu_0}\right)^2\left[1-\left(1-3\left(\frac{d}{R}\right)\right)\right]} \nonumber\tag{S3.8}\\
\simeq&\frac{3}{3+2\left(\frac{\mu}{\mu_0}\right)\left(\frac{d}{R}\right)}\nonumber \tag{S3.9}
\end{align}

在$\mu d/\mu_0 R \gg 1$的特殊情况下,式S3.9可进一步简化:
\begin{equation*}
\frac{H_{ss}}{H_0}\simeq \frac{2}{3}\left(\frac{\mu_0}{\mu}\right)\left(\frac{R}{d}\right)  \tag{2.47}
\end{equation*}

c) 式2.47的相同结果可以直接在极限$\mu/\mu_0\gg 1$和$d/R\ll 1$下由微扰法得到。

我们首先假设球壳材料的$\mu$无限大。这要求B线在$r=R$处垂直于球壳。
这是因为在$r=R$处,$\vec{H}_1$仅有径向分量。由于壳内$H=0$,故在$r=R$处$H_\theta$必须连续。
(当$\mu=\infty$时,$C=D=0$。)
由式S3.3a,$A=-R^3 H_0$,于是在$r=R$处:
\begin{equation*}
\vec{H}_1=-3 H_0 \cos\theta \vec{i_r} \tag{S3.10}
\end{equation*}

\begin{figure}[htbp]
	\centering
	\includegraphics[scale=0.8]{chpt2/figs/fig2.5.eps}
	\caption{通过球壳$\pm \theta$边界内区域进入壳的磁通。}
\end{figure}

$B$线局限在壳内,不会``逸"至区域3;也即壳内的$B$仅有$\vec{i}_\theta$分量。
应用磁通连续性($\nabla \cdot \vec{B}=0$)并在$\mu=\infty$条件下解$\vec{B}_2$。
解出$B_2$后,可以得到$\mu \neq \infty$但$\mu/\mu_0 \gg 1$时$\vec{H}_3$的近似表达式。

首先,计算通过表面区域$\pm \theta$范围内进入壳的总磁通(图2.5)。
壳表面积可由如图所示的微元(圆环)从$0$到$\theta$积分得到:
\begin{align}
\Phi=&\mu_0\int_{0}^{\theta} \vec{H}_1 \cdot d\vec{A}=\mu_0\int_{0}^{\theta} 3H_0\cos\theta 2\pi R^2 \sin\theta d\theta\nonumber\\
=&3\pi\mu_0 R^2 H_0 \sin^2 \theta\nonumber \tag{S3.11}
\end{align}

上式给出的$\Phi$等于在球壳$\theta$处$\theta$方向的磁通量。
因为在$d\ll R$条件下,壳在$\theta$处的截面区域面积$A_2$可由下式给出:
\begin{equation*}
A_2\simeq d\cdot 2\pi R\sin\theta \tag{S3.12}
\end{equation*}

于是,我们有:
\begin{align}
\Phi=&3\pi\mu_0 R^2 H_0 \sin^2 \theta\nonumber\\
\simeq& B_2 A_2=B_2 d 2 \pi R \sin\theta \tag{S3.13}
\end{align}

在式S3.13中解出$B_2$,得:
\begin{equation*}
\vec{B}_2\simeq \frac{3}{2}\mu_0 \left(\frac{R}{d}\right) H_0\sin\theta\vec{i}_\theta \tag{S3.14}
\end{equation*}

上式的$\vec{B}_2$是在$\mu=\infty$条件下得到的;由于$\vec{H}$的$\vec{i}_\theta$分量在$r=R-d$处必须连续,
于是我们可以导出$\vec{H}_3$的近似表达:
\begin{equation*}
H_{\theta 3}\simeq \frac{B_{\theta 2}}{\mu}=\frac{3}{2}\left(\frac{\mu_0}{\mu}\right)\left(\frac{R}{d}\right)H_0\sin\theta \tag{S3.15}
\end{equation*}

得到了$H_{\theta 3}$,我们就可以给出$\vec{H}_3$的完整表达式了:
\begin{align}
\vec{H}_3\simeq &\frac{3}{2}\left(\frac{\mu_0}{\mu}\right)\left(\frac{R}{d}\right)H_0(-\cos\theta\vec{i_r}+\sin\theta\vec{i}_\theta) \nonumber\tag{S3.16a}\\
\left|\frac{\vec{H}_3}{H_0}\right|\simeq& \frac{3}{2}\left(\frac{\mu_0}{\mu}\right)\left(\frac{R}{d}\right) \nonumber\tag{S3.16b}
\end{align}

式S3.16b给出的$\left|{\vec{H}_3}/{H_0}\right|$比值与式2.47给出的$H_{ss}/H_0$是一致的。
注意:此处的微扰法需要$\mu=\infty$和$d\ll R$条件,但从式S3.9到式2.47无需$\mu d/\mu_0 R \gg 1$条件。

d)应当明确,不能为了满足$d/R \ll 1$而将$d$选的特别小。事实上,下面的分析仅在以下条件满足时有效:
\begin{equation*}
\frac{\mu_0}{\mu} \ll \frac{d}{R} \ll 1 \tag{S3.17}
\end{equation*}

实际的$\mu$不可能无限大,屏蔽材料会随着外场的增大最终饱和。
因此,壳内的最大磁通($\theta=90^\circ$时取得)必须小于屏蔽壳材料的饱和磁通$\mu_0 M_{sa}$。
于是:
\begin{equation*}
\frac{3}{2}\left(\frac{R}{d}\right)\mu_0 H_0 \le \mu_0 M_{sa}  \tag{S3.18}
\end{equation*}

在式S3.18中解出$d/R$,有
\begin{equation*}
\frac{d}{R}\ge \frac{3H_0}{2M_{sa}}  \tag{2.48}
\end{equation*}

表2.5 给出了三种材料的\textit{微分}$\mu/\mu_0$(定义为$(\mu/\mu_0)_{dif}\equiv \Delta M/ \Delta H_0 |_{\mu_0 H_0}$)和$\mu_0M(H_0)$($\mu_0 H_0$的范围为$0\sim 1000$ gauss)的近似值。
这几种材料常用来屏蔽$\sim 100$ gauss下的磁场。三种材料的饱和磁通$\mu_0 M_{sa}$大概为2.1 T,2.0 T和2.2 T。

\begin{table}[htbp]
\centering
\caption{铁合金的$\mu/\mu_0$、$\mu_0M(H_0)$和$\mu_0 M_{sa}$值}
\begin{threeparttable}
\begin{tabular}{|c|c|c|c|c|c|c|}
\hline
\multirow{2}{*}{$\mu_0 H_0$[gauss]} & \multicolumn{2}{c|}{Annealed Ingot Iron} & \multicolumn{2}{c|}{As-Cast Steel} & \multicolumn{2}{c|}{Vanadium Permendur} \\ \cline{2-7}
&$ (\mu/\mu_0)_{dif} $ \tnote{*}& $\mu_0 $M[T] & $(\mu/\mu_0)_{dif} $\tnote{*} &$ \mu_0 M$[T] &$ (\mu/\mu_0)_{dif} $\tnote{*} & $\mu_0 M$[T] \\ \hline
1 & 7710 & 0.375 & na & na & na & na \\ \hline
3 & 3850 & 0.91 & 1660 & 0.25 & 4845 & 0.65 \\ \hline
5 & 500 & 1.42 & 1155 & 0.51 & 1875 & 1.25 \\ \hline
10 & 115 & 1.54 & 565 & 0.93 & 545 & 1.67 \\ \hline
20 & 47 & 1.60 & 180 & 1.25 & 170 & 1.96 \\ \hline
50 & 23.5 & 1.70 & 50 & 1.52 & 17 & 2.10 \\ \hline
100 & 17.5 & 1.81 & 25 & 1.70 & 4.8 & 2.15 \\ \hline
200 & 8.25 & 1.93 & 10 & 1.85 & 1.3 & 2.17 \\ \hline
500 & 2.0 & 2.05 & 1.0 & 1.92 & 0.26 & 2.18 \\ \hline
1000 & 0.4 & 2.11 & 0.45 & 2.01 & 0.07 & 2.19 \\ \hline
$\mu_0 M_{sa} $ & \multicolumn{2}{c|}{2.13 [T]} & \multicolumn{2}{c|}{2.03 [T]} & \multicolumn{2}{c|}{2.20 [T]} \\ \hline
\end{tabular}
\begin{tablenotes}
        \footnotesize
        \item[*] 数据来自永磁体手册的$M(H)$图(通用电气公司, 1963)。 %此处加入注释*信息
      \end{tablenotes}
\end{threeparttable}
\end{table}

e) $\mu/\mu_0=100$条件下的磁力线分布如图2.6。

\begin{figure}[htbp]
  \centering
 \includegraphics[scale=0.7]{chpt2/figs/fig2.6.eps}
  \caption{$\mu/\mu_0=100$磁性球壳置于均匀场中的磁场分布。注意到进出球壳的磁场线是\textit{近乎}垂直于壳的。}
\end{figure}


\subsection{讨论2.2:用圆柱壳屏蔽}
我们用类似于问题2.3中所用的微扰法推导$H_{cs}/H_0$的表达式。圆柱壳外径o.d.为$2R$,壁厚($d/R\ll 1$),
由高磁导率($\mu/\mu_0 \gg 1$)制成,置于幅值为$H_0$的外场$\vec{H}_\infty$中,圆柱壳内($r\le R-d$)
的磁场记为$H_{cs}$。在二维柱坐标系下,外场的形式为:
\begin{equation*}
\vec{H}_\infty =H_0 (-\cos\theta \vec{i_r}+\sin\theta\vec{i}_\theta) \tag{2.40}
\end{equation*}
其中,$\theta$的定义见图2.3。

我们假定圆柱材料的$\mu$无限大,$B$线在$r=R$处必须垂直于圆柱。于是,在$r=R$处有:
$$\vec{H}_1 =-2 H_0 \cos\theta \vec{i_r}$$

当然,壳内的$B$为$\theta$向。在$d/R\ll 1$条件下,磁通连续性要求$B$满足:
$$B_2 d=\int_{0}^{\theta}2\mu_0 H_0 R\cos\theta d\theta=2\mu_0 R H_0 \sin\theta$$

于是:
$$\vec{B}_2=2\mu_0 \left(\frac{R}{d}\right)H_0 \sin\theta \vec{i}_\theta$$

一旦得到$\mu=\infty$下的$\vec{B}_2$,我们就知道了$\mu/\mu_0 \gg 1$下的$\vec{H}_2$:
$$\vec{H}_2= \frac{\vec{B}_2}{\mu}\simeq 2\left(\frac{\mu_0}{\mu}\right) \left(\frac{R}{d}\right)H_0 \sin\theta \vec{i}_\theta$$

因为在没有表面电流时$H_\theta$是连续的,区域2和区域3一定有$H_{\theta 2}=H_{\theta 3}$。于是,在$r=R-d$处:
$$H_{\theta 3}=H_{\theta 2}\simeq 2\left(\frac{\mu_0}{\mu}\right) \left(\frac{R}{d}\right)H_0 \sin\theta$$

由以上几式可得:
\begin{align}
\vec{H}_3 \simeq& 2\left(\frac{\mu_0}{\mu}\right) \left(\frac{R}{d}\right)H_0 (-\cos\theta \vec{i_r}+\sin\theta\vec{i}_\theta)\nonumber\\
\left|\frac{\vec{H}_3}{H_0}\right|\equiv&\frac{H_{cs}}{H_0}\simeq 2\left(\frac{\mu_0}{\mu}\right) \left(\frac{R}{d}\right)
\end{align}

如问题2.3中的球壳一样,圆柱壳的厚度也不能任意薄;它必须有足够的厚度以防止饱和:
$$\mu H_{cs}=2\mu_0 H_0 \frac{R}{d}\le \mu_0 M_{sa}$$

根据上面的推导,有:
\begin{equation}
\frac{d}{R}\ge \frac{2H_0}{M_{sa}}
\end{equation}


\subsection{问题2.4:四个偶极子簇的远场}
本题讨论如图2.7布置的四个\textit{理想}偶极子簇的远场。
各偶极子的方向由圆圈内的箭头指示。
两个相反偶极子的中心距为$2\delta_d$。
偶极子$j$绕组厚度为零,直径为$2r_d$,$y$向总长度为$\ell_d$,
径向($r_j$)远离偶极子的远场($r_j \gg \ell_d$)可建模为球偶极子场$\vec{B}_j$:
\begin{equation}
\vec{B}_j=\frac{r_d^2 \ell_d B_0}{2r_j^3}(\cos\vartheta_j \vec{i}_{r_j}+\frac{1}{2} \sin\vartheta_j \vec{i}_{\theta_j})
\end{equation}
式中,$r_j$是分别到各偶极子中心的距离;
$\vartheta_j$的定义确保$\vartheta_j=0^\circ$时,各偶极子在绕组内的磁场指向$r_j$方向。
图2.7给出了各偶极子内磁场的方向,还定义了对所有偶极子共用的$r-\theta$坐标系和$z-x$坐标系。
$r\gg \delta_d$时,我们有${\vartheta}_1=\theta+180^\circ,\vartheta_2=\theta-90^\circ,\vartheta_3=\theta,\vartheta_4=\theta + 90^\circ$。

证明:若忽略各偶极子的末端效应,即仅考虑$y=0$平面,本组合系统的远场($r/\delta_d \gg 1$)的近似表达式为:
\begin{equation}
\vec{B}\simeq \frac{3r_d^2 \ell_d B_0 \delta_d}{r^4}(-\sin 2\theta \vec{i}_r+\frac{1}{2}\cos 2\theta \vec{i}_\theta)
\end{equation}

\begin{figure}[htbp]
  \centering
 \includegraphics[scale=0.9]{chpt2/figs/fig2.7.eps}
  \caption{四个偶极子布局的横截面。每个偶极子中的箭头指示线圈内磁场的方向。}
\end{figure}


\subsubsection*{问题2.4之解}
对于$r\gg \delta_d$,每个偶极子的$r_j$可以用$r$和$\theta$表示:
\begin{align}
r_1\simeq& r-\delta_d \sin\theta \nonumber\tag{S4.1a}\\
r_2\simeq& r+\delta_d \cos\theta\nonumber\tag{S4.1b}\\
r_3\simeq& r+\delta_d \sin\theta\nonumber\tag{S4.1c}\\
r_4\simeq& r-\delta_d \cos\theta\nonumber\tag{S4.1d}
\end{align}

将式S4.1分别代入式2.51,使用$\theta$表示$\vartheta$的:
\begin{align}
\vec{B}_1 \simeq& \frac{r_d^2 \ell_d B_0}{2(r-\delta_d\sin\theta)^3}(-\cos\theta \vec{i}_r-\frac{1}{2}\sin\theta \vec{i}_\theta) \nonumber\tag{S4.2a}\\
\vec{B}_2 \simeq& \frac{r_d^2 \ell_d B_0}{2(r+\delta_d\cos\theta)^3}(\sin\theta \vec{i}_r-\frac{1}{2}\cos\theta \vec{i}_\theta)  \nonumber\tag{S4.2b}\\
\vec{B}_3 \simeq& \frac{r_d^2 \ell_d B_0}{2(r+\delta_d\sin\theta)^3}(\cos\theta \vec{i}_r+\frac{1}{2}\sin\theta \vec{i}_\theta) \nonumber\tag{S4.2c}\\
\vec{B}_4 \simeq& \frac{r_d^2 \ell_d B_0}{2(r-\delta_d\cos\theta)^3}(-\sin\theta \vec{i}_r+\frac{1}{2}\cos\theta \vec{i}_\theta)  \nonumber\tag{S4.2d}
\end{align}

在$r\gg \delta_d$时,S4.2各式中的分母可对$\delta_d/r$作一阶展开,成为:
\begin{align}
\vec{B}_1 \simeq& \frac{r_d^2 \ell_d B_0}{2 r^3}\left[1+3\left(\frac{\delta_d}{r}\right)\sin\theta \right](-\cos\theta \vec{i}_r-\frac{1}{2}\sin\theta \vec{i}_\theta)\nonumber\tag{S4.3a}\\
\vec{B}_2 \simeq& \frac{r_d^2 \ell_d B_0}{2 r^3}\left[1-3\left(\frac{\delta_d}{r}\right)\cos\theta \right](\sin\theta \vec{i}_r-\frac{1}{2}\cos\theta \vec{i}_\theta)\nonumber\tag{S4.3b}\\
\vec{B}_3 \simeq& \frac{r_d^2 \ell_d B_0}{2 r^3}\left[1-3\left(\frac{\delta_d}{r}\right)\sin\theta \right](\cos\theta \vec{i}_r+\frac{1}{2}\sin\theta \vec{i}_\theta)\nonumber\tag{S4.3c}\\
\vec{B}_4 \simeq& \frac{r_d^2 \ell_d B_0}{2 r^3}\left[1+3\left(\frac{\delta_d}{r}\right)\sin\theta \right](-\sin\theta \vec{i}_r+\frac{1}{2}\cos\theta \vec{i}_\theta)\nonumber\tag{S4.3d}
\end{align}

将S4.3各式组合,可得:
\begin{equation*}
\vec{B}=\vec{B}_1+\vec{B}_2+\vec{B}_3+\vec{B}_4\simeq \frac{3r_d^2 \ell_d B_0 \delta_d}{r^4}(-\sin 2\theta \vec{i}_r+\frac{1}{2}\cos 2\theta \vec{i}_\theta) \tag{2.52}
\end{equation*}

可见,四个偶极子簇的$|\vec{B}|$按$\propto 1/r^4$衰减,而不像单个偶极子依照$\propto 1/r^3$衰减。


\subsection{问题2.5:铁电磁体的磁极形状}

\begin{figure}[htbp]
  \centering
 \includegraphics[scale=0.4]{chpt2/figs/fig2.8.eps}
  \caption{电极对,倒角为$\theta_{tp}$。图中$\bullet$指示了磁矩$\vec{m}_A\uparrow$的位置。}
\end{figure}

图2.8展示了一个铁制电磁体作用部分的剖面,两个铁磁材料制成的圆柱形磁极相对放置,圆柱末端削成锥形
---图中阴影部分。
磁极上的线圈(图中未画出)通电后,两极间的间隙内将产生一个相对均匀的磁场。
中心场理论上没有上限,因为它$\propto \ln(R_2/R_1)$。其中,$2R_1$和$2R_2$分别是锥体顶部和基部的直径。
由于随着$R_2$增加,磁体的质量会变得很大,故实际中的中心场上限$\sim 7\ \mathrm{T}$(巴黎Bellevue;瑞典Uppsala Uni.)。
锥顶部分的磁矩的磁场对中心位置的$z$向场有\textit{负}贡献,所以圆锥能够加强中心场。

证明:若铁芯在沿系统$z$轴方向励磁,$\theta_{tp}=54^\circ 44^\prime$是这种简单磁极几何的最优角度。
假定中心场是均匀分布于磁极件上的磁矩产生的场的叠加。
(如图所示的四条点线分别投影在定义$\theta_{tp}=54^\circ 44^\prime$的四条线上,相交于中心点。
如果间隙足够大,削成锥形并无好处。)$z$向励磁的磁矩$\vec{m}_A [\mathrm{A\cdot m^2}]$产生的偶极场为:
\begin{equation}
\vec{H}_{m_A}=\frac{\vec{m_A}}{r^3}(\cos\theta \vec{i}_r+\frac{1}{2}\sin\theta\vec{i}_\theta)
\end{equation}

$\vec{H}_{m_A}(r,\theta)$关于$z$轴对称,故可以从标量势$\cos\theta/r^2$导出。
这种推导方法已在问题2.1、2.3、2.4以及讨论2.1中应用。

提示:解出位于磁极倒角基部边缘上(图2.8左下角黑点)的单个磁矩$\vec{m}_A \uparrow$在$z$轴\textit{中心}场。

\subsubsection*{问题2.5的解}
磁矩$\vec{m}_A$在中心产生的磁场的$z$向分量$H_{m_A z}$为:
\begin{equation*}
H_{m_A z}=\frac{\vec{m}_A}{r^3}(\cos^2\theta-\frac{1}{2}\sin^2\theta) \tag{S5.1}
\end{equation*}
式中,$r_A$是$\vec{m}_A$到中心的距离。当$\theta=\theta_{tp}$时,式S5.1右侧为0。于是:
\begin{equation*}
\cos^2\theta_{tp}-\frac{1}{2}\sin^2\theta_{tp}=0 \tag{S5.2}
\end{equation*}

根据式S5.2,有$\cos\theta_{tp}=1/\sqrt{3}$(或$\tan\theta_{tp}=\sqrt{2}$),
于是$\theta_{tp}\simeq 54.736^\circ\simeq54^\circ 44^\prime$。

我们注意到,对阴影区中基线之上的磁矩均有$\theta_{tp}>54^\circ 44^\prime$。
由于随$\theta$值的增加,$\cos\theta$增大而$\sin\theta$减小,
在$\theta_{tp}>54^\circ 44^\prime$时,$H_{m_A z}$为负。
以$\theta_{tp}=54^\circ 44^\prime$倒角磁极可以消除该\textit{负}贡献,从而最大化中心场。

值得一提的是,NMR质谱仪中的``\textit{神奇角}"(magic angle)正是$54^\circ 44^\prime$。
通常,NMR样品放置时以这个``神奇角"对主轴场取向,以减少各向异性的作用。


\subsection{讨论2.3:永磁体}
永磁体是日常生活中使用的大量设备---如汽车、电视、电脑、手机、冰箱等---的关键组件。
实际上,如果没有永磁体,我们当下的现代生活将难以存续。
在低场($<1$) T的MRI中,永磁体还是超导体的竞争对手。由于永磁体MRI既无需制冷也很便宜,很受欢迎。

表2.6给出了90年(1910s-1990s)间永磁体和超导体的发展。
永磁体的指标以最大磁能$BH|_{mx}$表示,而超导体以最高临界温度$T_c|_{mx}$表示。
在这期间,$BH|_{mx}$提高了约30倍,$T_c|_{mx}$提高了约20倍。

永磁体照这个步调发展下去,在不久的将来,永磁体MRI将能达到$\sim 1\ $T。
更高磁场的MRI,还是以超导为主导。

\begin{table}[htbp]
\centering
\caption{永磁体和超导体的发展历程}
\label{磁体和超导体的发展}
\begin{threeparttable}
\begin{tabular}{|c|c|c||c|c|}
\hline
年代        & 永磁体 & $BH|_{mx}$[$\mathrm{kJ/m^3}$] & 超导体 & $T_c|_{mx}$[K]   \\ \hline
1910      &  特种钢   & 11       &  Pb   & 7.2 \\ \hline
1920-1940 &  Alnico\tnote{*} 1-4   & 15       &   NbN  & 16  \\ \hline
1950      &   Alnico 5  & 35       &    $\mathrm{Nb_3 Sn}$ & 18  \\ \hline
1960      &   Alnico 8,9  & 55       &   $\mathrm{Nb_{12}Al_3 Ge}$  &   19  \\ \hline
1970      &   $\mathrm{SmCo_5} $  & 140      &  $\mathrm{Nb_3Ge}$   & 23    \\ \hline
1980      &   Sm(CoCuFeZr)  & 240      &  $\mathrm{Bi_2 Sr_2 Ca_2 Cu_3 O_x}$   & 118 \\ \hline
1990      &  $\mathrm{Nd_2Fe_{14}B}$   & 350      & $\mathrm{(Hg, Pb) Sr_2 Ca_2 Cu_3 O_x}$    & 133 \\ \hline
\end{tabular}
\begin{tablenotes}
        \footnotesize
        \item[*] 主要组分为Al、Ni和Co的铁合金。 %此处加入注释*信息
      \end{tablenotes}
\end{threeparttable}
\end{table}



\subsection{问题2.6:圆柱中的准静态场}

\begin{figure}[htbp]
  \centering
 \includegraphics[scale=0.4]{chpt2/figs/fig2.9.eps}
  \caption{置于$z$向正弦时变磁场中的在$\theta=0$处有$\delta$窄缝的半径为$R$的理想导体制成
  的薄壁长圆柱的轴向视图。}
\end{figure}

半径为$R$的薄壁长圆柱由理想导体($\rho=\infty$)而非超导体薄板制成,侧面开有宽$\delta$的窄缝(如图2.9)。
圆柱置于正弦时变磁场中,磁场为$0^{th}$、均匀、$z$向(垂直纸面),即:
\begin{equation}
\vec{\CMcal{H}}_\infty(t)=\mathrm{Re}[H_0 e^{j\omega t}] \vec{i}_z
\end{equation}
式中,$H_0$是复幅值。忽略端部效应。

a)忽略$\delta/R$阶项。证明跨窄缝短边的$1^{th}$阶复电压幅值$V_{1|0}\equiv V_{1|\theta=0}$为:
\begin{equation}
V_{1|0}\equiv V_{1|\theta=0}=-j\omega \pi R^2 \mu_0 H_0
\end{equation}

b)一个电阻率为$\rho[\Omega\cdot\mathrm{m}]$的正常金属板置于窄缝将圆柱开缝恰好连接起来。
推导通过该板的$1^{th}$阶复电流密度(轴向单位长度)$J_1 [\mathrm{A/m}]$的表达式。
假定驱动频率($\omega/2\pi$)足够小,磁场仍保持方程2.54的准静态形式;电流在板截面上均匀流过。

c)在\textit{无}金属板(或$\rho_s=\infty$)条件下,在圆柱腔内画出6条描绘电场重要特征的$1^{th}$阶复电场($\vec{E}_1$)线。

d)在\textit{无}金属板条件下,推导两点间的一阶电压的线积分$V_1 |_{\theta=-\pi/2}^{\theta=+\pi/2}$的表达式。其中一点位于$\theta=+\pi/2$,另一点位于$\theta=-\pi/2$。



\subsubsection*{问题2.6之解}
a) 对$1^{th}$阶电场$\vec{E}_1(t)$应用积分形式的Faraday定律:
\begin{equation*}
\CMcal{V}_1(t) \equiv \int_{C} \vec{E}_1 (t) \cdot d\vec{s}=-\pi R^2 \mu_0 \frac{d\CMcal{H}_0(t)}{dt} \tag{S6.1}
\end{equation*}

线积分沿着圆柱表面逆时针方向进行(含窄缝)。式S6.1右侧包括圆柱($\pi R^2$)所定义的整个区域。
因为圆柱是理想导体,故在材料内有$\vec{\mathcal{E}}_1(t)=0$,对线积分仅有的非零贡献来自窄缝。以复幅值表示,有
\begin{equation*}
V_{1|0}=-j\omega \pi R^2 \mu_0 H_0 \tag{2.55}
\end{equation*}

b) 应用Ohm定律,得到$J_1$(单位长度):
\begin{equation*}
J_1=\frac{V_{1|0}}{\rho_s} \tag{S6.2}
\end{equation*}

c) 因为圆柱是理想导体,圆柱面上$\vec{E}_1$的切向分量必须为零:$\vec{E}_1$以直角离开或进入圆柱。
随着跨越圆柱的积分路径向窄缝左侧移动,积分区域减少,这令$|E_1|$变小。场线如图2.10所示。

d) 这是c)的特例。根据对称性可以准确计算出线积分:积分区域等于$\pi R^2/2$。于是:
\begin{equation*}
V_1 |_{-\pi/2}^{+\pi/2}=-\frac{1}{2}j\omega \pi R^2 \mu_0 H_0   \tag{S6.3}
\end{equation*}

式2.55和S6.3表明,在相同轴向坐标下,跨过圆柱的两点间的电压与电压触头方位有关。
在存在时变磁场(外施磁场,如本例;系统中电流产生的磁场)条件下测电压时一定要记住这一点。
超导体交流损耗的电测就是一个应当小心的非常好的实例。

\begin{figure}[htbp]
  \centering
 \includegraphics[scale=0.35]{chpt2/figs/fig2.10.eps}
  \caption{$\vec{E}_1$线垂直于理想导体圆柱。从$x=R$到$x=-R$,$|E_1|$是减小的。}
\end{figure}


\subsection{问题2.7:圆柱壳的感应加热}
\begin{figure}[htbp]
  \centering
 \includegraphics[scale=0.35]{chpt2/figs/fig2.11.eps}
  \caption{圆柱金属壳置于均匀的正弦时变磁场之中。}
\end{figure}
本问题处理金属(非超导)圆柱壳的感应加热。它是一个同时涉及时谐电磁场、能流(Poynting矢量)和能量耗散的好例子。
本问题和下一问题是交流损耗,特别是涡流损耗的先导实例,第7章将会进一步讨论。
感应加热在电炉中广泛使用,用以在导电性材料内获得高温;有它时也被用来作为超导线圈热行为的研究工具。
在超导磁体技术研究中,感应加热最常用脉冲磁场形式在超导线圈中产生小的正常区域,模拟暂态扰动。

图2.11给出一个置于正弦时变磁场($0^{th}$阶、均匀、$z$向)中``长"金属圆柱壳,其电导率为$\rho_e$,外径o.d.为$2R$,厚度$d\ll R$。即:
\begin{equation*}
\vec{\CMcal{H}}_\infty(t)=\mathrm{Re}(\vec{H}_0 e^{j\omega t})=\mathrm{Re}(H_0 e^{j\omega t})\vec{i}_z \tag{2.54}
\end{equation*}
式中,$H_0$是磁场复幅值。

我们首先用两种方法解出相应的场量。接下来,再用两种方法解出圆柱内的能量耗散。

\subsubsection*{第一部分:感应加热的场}
首先,用两种方法解出场量,如下文详述。

\textbf{方法一}

a) 使用\textit{积分}形式的Maxwell方程,忽略端部效应,证明在$r\le R$区域内的$1^{th}$阶电场$\vec{E}_1$
以及在$r\simeq R$的壳内的$1^{th}$阶电流密度$\vec{J}_1$,分别为:
\begin{align}
\vec{E}_1=&-\frac{j\omega \mu_0 r H_0}{2} \vec{i}_\theta\\
\vec{J}_1=&-\frac{j\omega \mu_0 R H_0}{2\rho_e} \vec{i}_\theta
\end{align}

b) 证明在$r\le R-d$区域内的$1^{th}$阶磁场$\vec{H}_1$可以表示为:
\begin{equation}
\vec{H}_1=-\frac{j\omega \mu_0 R d H_0}{2\rho_e}\vec{i}_z
\end{equation} 

c) 上面在准静态近似下得出的式2.56-2.58仅在``低"频条件下成立,
或者说仅在频率远小于``趋肤深度"频率$f_{sk}$时下成立。证明:
\begin{equation}
f_{sk}=\frac{\rho_e}{\pi \mu_0 R d}
\end{equation}

\textbf{方法二}

方法一得到的均随着$\omega$增大的$\vec{E}_1$、$\vec{J}_1$、$\vec{H}_1$仅在频率小于$f_{sk}$时成立。
下面,我们用一种新技术来推导在\textit{所有频率}下均成立的
室温孔内场$\vec{H}_T=\vec{H}_0+\vec{H}_r$的完整表达式。其中,$\vec{H}_T$是总磁场,
$\vec{H}_0$是原磁场,$\vec{H}_R$是室温孔内的系统反应场。
本方法首先通过将$\vec{H}_T=\vec{H}_0+\vec{H}_r$视为$0^{th}$阶场,
解出作为$1^{th}$阶磁场响应的$\vec{H}_R$。

d) 证明在$d\ll R$条件下,壳内的$\vec{H}_R$、$\vec{H}_T$和$\vec{J}$为:
\begin{align}
\vec{H}_R=&-\frac{j\omega \mu_0 R d H_0}{2\rho_e+j\omega \mu_0 R d} \vec{i}_z\\
\vec{H}_T=&\frac{2\rho_e H_0}{2\rho_e+j\omega \mu_0 R d} \vec{i}_z\\
\vec{J}=&-\frac{j\omega \mu_0 R H_0}{2\rho_e+j\omega \mu_0 R d} \vec{i}_\theta
\end{align}

\subsubsection*{第一部分之解}
根据$\theta$方向的对称性,知$\vec{E}_1$和$\vec{J}_1$均为$\theta$向的常量,仅依赖于$r$。
这样,Faraday感应定律可在$r$处的边线$\mathcal{C}$上进行,围成的面积为$\mathcal{S}$:
\begin{align}
\oint_{\mathcal{C}} \vec{E}_1 \cdot d\vec{s}=-j\omega \mu_0 \int_{\mathcal{S}} \vec{H}_0 \cdot d\vec{\mathcal{A}}\nonumber\\
\int_{0}^{2\pi} r E_{1\theta} d\theta=-j\omega \mu_0 \int_{0}^{r} 2\pi r H_0 dr\nonumber\\
E_{1\theta}\int_{0}^{2\pi} r d\theta=-j\omega \mu_0 H_0\int_{0}^{r} 2\pi r dr\nonumber\tag{S7.1}
\end{align}

在$r\le R$时,
\begin{equation*}
E_{1\theta} 2\pi r = -j\omega \mu_0 H_0 \pi r^2 \tag{S7.2}
\end{equation*}

式S7.2等号两侧同除$2\pi r$,有:
\begin{equation*}
E_{1\theta} = -\frac{j\omega \mu_0 r H_0}{2} \tag{S7.3}
\end{equation*}

于是:
\begin{equation*}
\vec{E_{1}} = -\frac{j\omega \mu_0 r H_0}{2}\vec{i}_\theta \tag{2.56}
\end{equation*}

$1^{th}$阶电流仅在壳内($r\simeq R$)流动:
\begin{equation*}
\vec{J}_1=\frac{\vec{E_{1}}(r\simeq R)}{\rho_e} = -\frac{j\omega \mu_0 R H_0}{2\rho_e}\vec{i}_\theta \tag{2.57}
\end{equation*}

在$d\ll R$时,我们可以将电流处理为$1^{th}$阶表面电流$\vec{K}_1$:
\begin{equation*}
\vec{K}_1=\vec{J}_1 \cdot d= -\frac{j\omega \mu_0 R d H_0}{2\rho_e}\vec{i}_\theta \tag{S7.4}
\end{equation*}

b) 在$r>R$时,$\vec{H}_1=0$;使用式2.6,我们可以将表面电流$\vec{K}_1$等效为
在$r=R$处$\vec{H}$的不连续:壳内壁为$\vec{H}_0+\vec{H}_1$,壳外为$\vec{H}_0$。于是:
\begin{align}
\vec{K}_1=&\vec{i}_r \times [\vec{H}_0-(\vec{H}_0+\vec{H}_1)]=-\frac{j\omega \mu_0 R d H_0}{2\rho_e}\vec{i}_\theta\nonumber\\
=&\vec{i}_r\times-\vec{H}_1=-\frac{j\omega \mu_0 R d H_0}{2\rho_e}\vec{i}_\theta\nonumber\tag{S7.5}
\end{align}

在$d\ll R$条件下从式S7.5中解出$\vec{H}_1(r\le R-d)$:
\begin{equation*}
\vec{H}_1=-\frac{j\omega \mu_0 R d H_0}{2\rho_e}\vec{i}_z \tag{2.58}
\end{equation*}

c) 式2.57、S7.4和2.58指出,$\vec{J}_1$、$\vec{K}_1$和$\vec{H}_1$的幅值均随着频率单调增长;
这意味着它不可能对所有$\omega$成立。
事实上,这些解仅在准静态近似可用的低频下有效。
特别的,式2.58给出的$\vec{H}_1$仅在$|\vec{H}_1|\ll |\vec{H}_0|$时有效:
\begin{equation*}
|\vec{H}_1|=\frac{\omega \mu_0 R d |H_0|}{2\rho_e}\ll |\vec{H}_0|\tag{S7.6}
\end{equation*}

由S7.6,我们可以得到准静态近似有效的频率上限,即通常所谓的趋肤深度频率$f_{sk}$:
\begin{equation*}
f_{sk}=\frac{\rho_e}{\pi \mu_0 R d} \tag{2.59}
\end{equation*}

可见,正弦时变磁场中的物体的$f_{sk}$不仅与其材料的电阻率有关,还与其尺寸有关。

d) 第二种方法在计算反应场时,设定$\vec{H}_1\equiv \vec{H}_R$,并在式2.58中用$\vec{H}_0+\vec{H}_R$代换$\vec{H}_0$:
\begin{equation*}
\vec{H}_R=-\frac{j\omega \mu_0 R d (\vec{H}_0+\vec{H}_R)}{2\rho_e} \tag{S7.7}
\end{equation*}

在S7.7中解出$\vec{H}_R$,得:
\begin{equation*}
\vec{H}_R=-\frac{j\omega \mu_0 R d \vec{H}_0}{2\rho_e+j\omega \mu_0 R d}\vec{i}_z \tag{2.60}
\end{equation*}

联立式2.60和$\vec{H}_T=\vec{H}_0+\vec{H}_R$,可得:
\begin{align}
\vec{H}_T&=\vec{H}_0+\vec{H}_R=H_0\left(1-\frac{j\omega\mu_0 R d}{2\rho_e+j\omega\mu_0 Rd}\right)\vec{i}_z\nonumber\\
&=\frac{2\rho_e H_0}{2\rho_e+j\omega \mu_0 R d}\vec{i}_z\nonumber\tag{2.61}
\end{align}

$\vec{J}$和$\vec{H}_R$通过$\nabla\times \vec{H}=\vec{J}$相联系,又$\vec{K}=\vec{J}d$,于是:
\begin{align}
\vec{J}=&\frac{1}{d}H_R \vec{i}_\theta \nonumber\tag{S7.8}\\
=&-\frac{j\omega \mu_0 R H_0}{2\rho_e+j\omega \mu_0 R d}\vec{i}_\theta\nonumber\tag{2.62}
\end{align}

可见,在低频极限下,式2.60给出的$\vec{H}_R$退化为式2.58给出的$\vec{H}_1$;
在高频极限下,$\vec{H}_R$退化为$-\vec{H}_0$而$\vec{H}_T$变为$0$。$\vec{J}$存在类似的行为。

\subsubsection*{第二部分:能量耗散}
现在,我们用两种方法解出圆柱内的能量耗散。

\textbf{方法一}

e) 我们可以直接计算$<p>=\vec{E}\cdot \vec{J}^* /2=\rho_e |J|^2 /2$(方程2.21)
而算得圆柱壳的电阻性能量损耗。式中的$\vec{J}$在式2.62给出。
证明在$d\ll R$条件下,壳内的时均总能量耗散(单位长度)的表达式为
\begin{equation}
<P>=2\pi R d<p>=\frac{\pi \rho_e \omega^2 \mu_0^2 R^3 d}{4\rho_e^2+\omega^2\mu_0^2 R^2 d^2} |H_0|^2
\end{equation}

\textbf{方法二}

提供给圆柱的复功率也可以视为从$r>R$处的源在$r=R$处进入圆柱内的Poynting能流。

f) 证明在$r=R$处流入圆柱的$1^{th}$阶复Poynting矢量$\vec{S}_1$的面积分(单位圆柱长度)为:
\begin{equation}
-\oint_{\mathcal{S}}\vec{S}_1 \cdot d\mathcal{A}=\frac{1}{2}(2\pi R)E_{1\theta} H_0^*=\frac{j\pi\rho_e\omega\mu_0 R^2}{2\rho_e+j\omega\mu_0 R d}|H_0|^2
\end{equation}

提示:$E_{1\theta}=\rho_e J_\theta$,而$J_\theta$已在式2.62给出。

g) 证明式2.64的等号右侧的实部与式2.63中给出的$<P>$是一致的。

h) 画出$<P>$以$\rho_e$为函数的图形。
由于理想导体($\rho_e=0$)和理想绝缘体($\rho_e=\infty$)均不产生损耗,
图形应从$<P>=0$开始,并在$\rho_e \rightarrow \infty$时趋于0。
我们注意到,其实式2.63给出的$<P>$已经指出了这个特点。

i) 从$<P>$vs. $\rho_e$的关系可以得到一个结论:存在临界电阻率$\rho_{e_c}$使该点$<P>$最大。
证明$\rho_{e_c}$的表达式:
\begin{equation}
\rho_{e_c}=\frac{\omega \mu_0 R d}{2}
\end{equation}

从式2.65可知,对一个给定的电阻率($\rho_e$)和样品尺寸($R,d$)组合,
存在一个可以令加热最大化的最优频率:该频率就是式2.59给出的趋肤深度频率$f_{sk}$。

j) 计算一个半径$R=10\ \mathrm{mm}$,壁厚$d=0.5\ \mathrm{mm}$,电阻率$\rho_e=2\times 10^{-10}\ \Omega\mathrm{m}$(大致为液氦温度下铜的电阻率)的铜管的$f_{sk}$。

\subsubsection*{第二部分之解}
e) 正弦情况下,时均能量耗散(单位长度)$<p>=\vec{E}\cdot \vec{J}^* /2=\rho_e |J|^2 /2$,
其中$\vec{J}$是复电流密度(方程2.62)。于是有:
\begin{equation*}
<p>=\frac{\rho_e}{2}|J_\theta|^2=\frac{\rho_e}{2}\left(\frac{\omega^2 \mu_0^2 R^2}{4\rho_e^2+\omega^2 \mu_0^2 R^2 d^2}\right)|H_0|^2 \tag{S7.9}
\end{equation*}

于是,壳内\textit{总}时均能耗(单位长度)$<P>$可以用$<p>$乘以壳的截面积得到:
\begin{equation*}
<P>=2\pi R d<p>=\frac{\pi \rho_e \omega^2 \mu_0^2 R^3 d}{4\rho_e^2+\omega^2 \mu_0^2 R^2 d^2}|H_0|^2 \tag{2.63}
\end{equation*}

下面考察$\rho_e$的两个极限:
\begin{align}
&\rho_e \ll \omega \mu_0 R d\mbox{ (良导体)时:}<P>\simeq \frac{\pi \rho_e R}{d}|H_0|^2\propto \rho_e\nonumber\tag{S7.10a}\\
&\rho_e \gg \omega \mu_0 R d\mbox{ (不良导体)时:}<P>\simeq \frac{\pi \omega^2 \mu_0^2 R^3 d}{4\rho_e}|H_0|^2\propto \frac{1}{\rho_e}\nonumber\tag{S7.10b}
\end{align}

如我们所期望的,在上面两种极限情况下都有$<P>\rightarrow 0$。

f) 复Poynting矢量$\vec{S}$的$1^{th}$阶展开为:
\begin{equation*}
\vec{S}_1=\frac{1}{2}(\vec{E}_0 \times \vec{H}_0^*+\vec{E}_0 \times \vec{H}_1^*+\vec{E}_1 \times \vec{H}_0^*) \tag{S7.11}
\end{equation*}

计算$1^{th}$阶Poynting矢量时,$E$和$H$场的下标必须不大于1。
在本例下,我们有$\vec{E}_0=0$,于是S7.11化简为:
\begin{equation*}
\vec{S}_1=\frac{1}{2}(\vec{E}_1 \times \vec{H}_0^*) \tag{S7.12}
\end{equation*}
式中的$\vec{E}_1$由式2.62给出:
\begin{equation*}
\vec{E}_1=\rho_e \vec{J}=-\frac{j\rho_e \omega \mu_0 R H_0}{2\rho_e+j\omega \mu_0 R d}\vec{i}_\theta \tag{S7.13}
\end{equation*}

于是有:
\begin{equation*}
-\oint_{\mathcal{S}}\vec{S}_1 \cdot d\mathcal{A}=\frac{1}{2}(2\pi R)E_{1\theta} H_0^*=\frac{j\pi\rho_e \omega \mu_0 R^2}{2\rho_e+j\omega \mu_0 R d} |H_0|^2 \tag{2.64}
\end{equation*}

g) 根据复数的基本运算法则,取f)最后结果的实部,易得。可以看到,这和方法一得到的结果是一致的。
\begin{equation*}
<P>=\frac{\pi \rho_e \omega^2 \mu_0^2 R^3 d}{4\rho_e^2+\omega^2 \mu_0^2 R^2 d^2}|H_0|^2 \tag{S7.14}
\end{equation*}

h) 如图2.12。

i) 将$<P>$对$\rho_e$求导,并令之在$\rho_{e_c}$处为0,有:
\begin{equation*}
\frac{d<P>}{d\rho_e} |_{\rho_{e_c}}=\left[\frac{\pi \omega^2 \mu_0^2 R^3 d}{4\rho_{e_c}^2+\omega^2 \mu_0^2 R^2 d^2}-\frac{8\pi \rho_{e_c} \omega^3 \mu_0^2 R^3 d}{(4\rho_{e_c}^2+\omega^2 \mu_0^2 R^2 d^2)^2}\right]=0 \tag{S7.15}
\end{equation*}

在S7.15中解出$\rho_{e_c}$,有:
\begin{equation*}
\rho_{e_c}=\frac{\omega \mu_0 R d}{2} \tag{2.65}
\end{equation*}

式2.65在均匀、正弦时变磁场施加于导体样品的感应加热应用中非常重要。
样品被样品中的感应涡流加热。
在式2.59给出的趋肤深度频率$f_{sk}$下,感应加热最大。

j) 将铜圆柱的参数$R=1\ \mathrm{cm}, d=0.5\ \mathrm{mm},\rho_e=2\times 10^{-10}\ \Omega\mathrm{m}$($\sim 4$ K时铜的电阻率)代入式2.59,有:
\begin{equation*}
f_{sk}=\frac{\rho_{e_c}}{\pi\mu_0 R d}\simeq 10Hz \tag{2.59}
\end{equation*}

\begin{figure}[htbp]
  \centering
 \includegraphics[scale=0.4]{chpt2/figs/fig2.12.eps}
  \caption{感应加热圆柱壳的功率耗散和电阻率的关系。}
\end{figure}

%\begin{quotation}
%\textbf{微波炉}\\
%\kaishu{微波炉基于感应加热,但它和我们这里研究的属于不同类型。
%在微波炉中,频率被设定为水分子的主要振动频率,使食物中的水吸收电磁能而被加热。}
%\end{quotation}


\subsection{问题2.8:金属扁带中的涡流损耗}
\begin{figure}[htbp]
  \centering
 \includegraphics[scale=0.4]{chpt2/figs/fig2.13.eps}
  \caption{宽度为b的金属扁带置于时变磁场中。}
\end{figure}

本题推导置于时变磁场中的金属扁带的涡流损耗的表达式,它可用于计算铜基底超导带中的涡流损耗计算。
(感应电流发热有用时叫感应加热;有害时就通常称为涡流损耗。)

图2.13给出了一条置于时变外磁场中电导率为$\rho_e$、宽为$b$($y$向)、厚为$a$($z$向)的``长''($x$向)金属扁带。外场$dB_0/dt=\dot{B_0}$,为$0^{th}$阶、均匀、$z$向。

a) 证明$1^{th}$阶电场$\vec{E}_1$可表示为:
\begin{equation}
E_{1x}=y\dot{B_0}
\end{equation}

b) 证明\textit{空间平均}能耗密度$\tilde{p}$(单位体积)可表示为:
\begin{equation}
\tilde{p}=\frac{(b\dot{B_0})^2}{12\rho_e}
\end{equation}

c) 当外场以频率$\omega$正弦变化,即$B(t)=B_0 \sin\omega t$时,证明\textit{时间平均}能耗密度$<\tilde{p}>$可表示为:
\begin{equation}
<\tilde{p}>=\frac{(b\omega\dot{B_0})^2}{24\rho_e}
\end{equation}


\subsubsection{问题2.8之解}
a) 由于$\vec{B}_0$是均匀的且系统不依赖于$x$,所以$\vec{E}_1$只能指向$x$向且仅依赖于$y$,
也即$\nabla\times \vec{E}_1=\partial \vec{B}_0/\partial t$可化简为:
\begin{equation*}
-\frac{dE_{1x}}{dy}=-\frac{B_0}{dt}=-\dot{B}_0 \tag{S8.1}
\end{equation*}

根据对称性,$E_{1x}(y=0)=0$,我们由S8.1可得:
\begin{equation*}
E_{1x}=y\dot{B_0}  \tag{2.66}
\end{equation*}

b) 扁带中的\textit{局域}能耗密度$p(y)$由$\vec{E}_1\cdot\vec{J}_1$给出。
总的能耗(单位长度)$P$于是为:
\begin{equation*}
P=a\int_{-b/2}^{b/2}p(y)dy=\frac{2a(\dot{B}_0)^2}{\rho_e}\int_{0}^{b/2}y^2dy=\frac{ab(b\dot{B_0})^2}{12\rho_e}  \tag{S8.2}
\end{equation*}

式S8.2在$B_0$变化``足够''慢以及材料``足够''阻性时成立。
也即,仅在$\vec{J}_1$感应出来的$1^{th}$阶感应磁场相比$\vec{B}_0$很小时才有效。

\textit{空间平均}能耗密度$\tilde{p}$(单位体积)可以由$P$除以带材截面得到:
\begin{equation*}
\tilde{p}=\frac{P}{ab}=\frac{(b\dot{B_0})^2}{12\rho_e}  \tag{2.67}
\end{equation*}

c) 在正弦激励下,\textit{时均}能耗密度$<p>$为:
\begin{equation*}
<\tilde{p}>=\frac{1}{2}E_{1x} J_{1x}^*  \tag{S8.3}
\end{equation*}

代入$E_{1x}=j\omega y B_0$,$J_{1x}=E_{1x}/\rho_e$,$<p>$在带材体积上的平均值为:
\begin{equation*}
<\tilde{p}>=\frac{2a(\omega B_0)^2}{2\rho_e (ab)}\int_{0}^{b/2} y^2 dy=\frac{(b\omega\dot{B_0})^2}{24\rho_e}  \tag{2.68}
\end{equation*}

可见,$\tilde{p}$和$<\tilde{p}>$分别正比于$(b\dot{B_0})^2$和$(b\omega \dot{B_0})^2$;
也就是说,两者对感应磁场和导体宽度均为平方依赖关系。


\subsection{讨论2.4:切分以减少涡流损耗}
若将扁带劈为两条,每一条宽度为$b/2$。由式2.67和2.68可知,$\tilde{p}$和总能耗将变为原来的$1/4$。
因此,可以通过将扁带切分的方法将涡流损耗降至任意小。
这种切分技术在电力变压器中广泛使用,其中的铁扼是由铁片堆叠成的。
我们将在第5章和第7章中看到,超导体也能从切分技术中获益:磁体导体一般都是细丝化的。


\subsection{问题2.9:Rogowski线圈}
Rogowski线圈是时变电流的电流计。
它是一种环形拾磁线圈,其输出电压正比于被Rogowski线圈包围的截面内通过的总电流。
图2.14a给出了一个Rogowski线圈,待测电流$I(t)$在线圈中间。
如图2.14a所示,Rogowski线圈包括$N$个串联的单匝小圆环。
各半径为$c$的圆环的中心位于电流中心的径向$R$处。
图2.14b定义了一匝的$x-y$坐标系。

a) 说明在$c\ll R$条件下,$N$匝的Rogowski线圈的总磁链$\Phi(t)=N\Phi_1(t)$近似为:
\begin{equation}
\Phi(t)\simeq\frac{\mu_0 N c^2}{2R}I(t)
\end{equation}
式中,$\Phi_1(t)$是与单匝交链的总磁通。

b) 证明$\Phi(t)$的准确表达式为:
\begin{equation}
\Phi(t)=\mu_0 N (R-\sqrt{R^2-c^2})I(t)
\end{equation}

拿出$N$匝中的一匝在放在$x-y$坐标系中,将其中心与原点对齐来计算$\Phi(t)$。

c) 证明在极限$(c/R)^4\ll 1$下,式2.70退化为式2.69。

d) 证明式2.70给出的$\Phi(t)$在Rogowski线圈的轴线偏离电流中心时也有效。

e) 计算一个$N = 3600; c = 3\ \mathrm{mm}; R = 0.5\ \mathrm{m}$的Rogowski线圈在$\Delta I(t)=1\ \mathrm{MA}$时两个端子之间产生的伏秒值。

\begin{figure}[htbp]
  \centering
 \includegraphics[scale=0.4]{chpt2/figs/fig2.14.eps}
  \caption{(a)Rogowski线圈。$N$匝,每匝直径为$2c$,环绕在待测的时变电流$I(t)$外围;
  (b)单匝(半径为$r$)的截面图。中心与$x-y$坐标系原点对齐,圆环到电流中心的距离为$R$。}
\end{figure}

\subsubsection{问题2.9之解}
a) 电流$I(t)$产生的磁场$H_\phi (t)$相对于电流方向是周向的。
距电流中心距离$R$处的$H_\phi(t)$为:
\begin{equation*}
H_\phi (t)=\frac{I(t)}{2\pi R} \tag{S9.1}
\end{equation*}

在$c\ll R$时,式S9.1给出的$H_\phi(t)$在每一圆环匝的横截面$\pi c^2$上几乎都是成立的。
由于Rogowski线圈有$N$个这样的圆环匝,我们有:
\begin{equation*}
\Phi(t)\simeq \frac{\mu_0 N c^2}{2R}I(t)  \tag{2.69}
\end{equation*}

Rogowski线圈输出电压$V(t)$于是可以写为:
\begin{equation*}
V(t)=\frac{d\Phi(t)}{dt}=\frac{\mu_0 N c^2}{2R}\frac{dI(t)}{dt}  \tag{S9.2}
\end{equation*}

b) 因为$H_\phi (t)$在各圆环匝上并非不变,单匝包围的总磁通$\Phi_1(t)$应由沿回路围成区域的积分得到。
若定回路中心坐标为$(0,0)$,有$x^2+y^2=c^2$,则$\Phi_1(t)$为:
\begin{align}
\Phi (t)=&\frac{\mu_0 I(t)}{2\pi}\int_{-c}^{c}\int_{-\sqrt{c^2-y^2}}^{\sqrt{c^2-y^2}}\frac{1}{R+y}dxdy\nonumber\\
=&\frac{\mu_0 I(t)}{\pi}\int_{-c}^{c}\frac{\sqrt{c^2-y^2}}{R+y}dy\nonumber\tag{S9.3}
\end{align}

式S9.3可以使用新变量$\xi\equiv R+y$得到闭式解(注意:$d\xi=dy$)。于是:
\begin{align}
\Phi_1 (t)=&\frac{\mu_0 I(t)}{\pi} \int_{R-c}^{R+c}\frac{\sqrt{c^2-R^2+2R\xi-\xi^2}}{\xi}d\xi\nonumber\\
=&\mu_0(R-\sqrt{R^2-c^2})I(t)\nonumber\tag{S9.4}
\end{align}

$N$匝的Rogowski线圈的交链磁通$\Phi(t)=N\Phi_1(t)$:
\begin{equation*}
\Phi (t)=\mu_0 N(R-\sqrt{R^2-c^2})I(t) \tag{2.70}
\end{equation*}

c) 式2.70可以写为:
\begin{equation*}
\Phi (t)=\mu_0 NI(t)\left(R-R\sqrt{1-\frac{c^2}{R^2}}\right) \tag{S9.5}
\end{equation*}

由于$x\ll 1$时有$\sqrt{1-x}\simeq 1-(1/2)x+(1/8)x^2-...$,式S9.5截取到二阶有:
\begin{align}
\Phi(t)&\simeq\mu_0 N\left[R-R\left(1-\frac{1}{2}\frac{c^2}{R^2}+\cdots\right)\right]\nonumber\tag{S9.6}\\
\Phi (t)&\simeq \frac{\mu_0 Nc^2}{2R} I(t)\nonumber\tag{2.69}
\end{align}

\begin{figure}[htbp]
	\centering
	\includegraphics[scale=0.4]{chpt2/figs/fig2.15.eps}
	\caption{正交于电流/Rogowski线圈组方向的横截面$(x,y)$,其中Rogowski线圈中心位于$(0,0)$,
电流中心在Rogowski线圈中心向下偏离$\delta_i$。}
\end{figure}

d) 图2.15给出正交于电流/Rogowski线圈组方向的横截面$(x,y)$,其中Rogowski线圈中心位于$(0,0)$,
电流中心在Rogowski线圈中心向下偏离$\delta_i$。

关键参数如图2.15中的定义:$r$为Rogowski中心$(0,0)$到一匝线圈上的点$A$的径向距离;
$\theta$是$y$轴和$r$之间的夹角;$s$是\textit{电流中心}到点$A$的距离;
$\epsilon$是$r$与$s$之间形成的角。根据几何条件,$s^2$为:
\begin{equation*}
s^2=(r\cos\theta+\delta_i)^2+r^2\sin^2\theta=r^2+\delta_i^2+2r\delta_i\cos\theta \tag{S9.7}
\end{equation*}

将$r$延长$\delta_i \cos\theta$形成一个直角(图2.15未画出),我们有:
\begin{equation*}
\cos\epsilon=\frac{r+\delta_i \cos\theta}{s} \tag{S9.8}
\end{equation*}

点$A$处的磁场由$H_A(t)=I(t)/2\pi s$给出;它在Rogowski线圈回路上点$A$处的法向分量为:
\begin{equation*}
H_{A\bot}=\frac{I(t)}{2\pi s}\cos\epsilon=\frac{I(t)}{2\pi s}\left(\frac{r+\delta_i \cos\theta}{s}\right) \tag{S9.9}
\end{equation*}

联立式S9.7和式S9.9,得到:
\begin{equation*}
H_{A\bot}=\frac{I(t)}{2\pi}\left(\frac{r+\delta_i \cos\theta}{r^2+\delta_i^2+2r\delta_i\cos\theta}\right) \tag{S9.10}
\end{equation*}

为了计算$\Phi(t)$,S9.10乘上$(N/2\pi)2\sqrt{c^2−(r−R)^2}$必须积分两次:第一次视径向距离$r$为常数,
对$\theta$从$0$至$2\pi$积分,计及$N$匝;第二次对$r$从$R-c$到$R+c$积分。
注意到$2\sqrt{c^2−(r−R)^2}$是每一匝直径为$c$的线圈在$r$处总的弦距离($z$向),它位于距离圆环匝中心的$r-R$处。
\begin{align}
\Phi(t)&=\frac{N I(t)}{2\pi^2}\int_{R-c}^{R+c}\int_{0}^{2\pi}\left[ \frac{(r+\delta_i \cos\theta)\sqrt{c^2-(r-R)^2}}{r^2+\delta_i^2+2r\delta_i \cos\theta}\right]d\theta dr\nonumber\tag{S9.11a}\\
&=\frac{N I(t)}{\pi}\int_{R-c}^{R+c} \left[ 0+\frac{\sqrt{c^2-(r-R)^2}}{r}\right]dr\nonumber\tag{S9.11b}
\end{align}

可见,积分中并无$\delta_i$。
下面,对S9.11b沿着圆环匝的径向从$r=R-c$到$r=R+c$积分,结果为:
\begin{equation*}
\Phi(t)=\frac{N I(t)}{\pi}\left[ \pi(R-\sqrt{R^2-c^2})\right] \tag{S9.11c}
\end{equation*}

于是,不管Rogowski线圈与其围住的电流$I(t)$是否同心,它都可以进行精确的测量。

e) 在$N = 3600, c = 3\ \mathrm{mm}; R = 0.5\ \mathrm{m}; \Delta I = 1\ \mathrm{MA}$时,式2.69可用。因为$(c/R)^4 = 1.3×10^{−9}\ll 1$。由$S9.2$得:
$$\int V(t)dt=\frac{\mu_0 N c^2 \Delta I}{2R}\simeq 41\ \mathrm{mVs}$$

在一个充满噪声的环境内,比如典型的试验聚变设备中,测到$40\ \mathrm{mVs}$级别的信号水平并不简单,但也不是完全得不到。

%\begin{quotation}
%\kaishu 正如我们所知,有已知的已知,他们是我们知道我们知道的;我们也知道有些是已知的未知。这就是说我们知道有些事情我们是不知道的。但是还有未知的未知---这些事我们不知道我们不知道。---Donald Rumsfeld,2002
%\end{quotation}