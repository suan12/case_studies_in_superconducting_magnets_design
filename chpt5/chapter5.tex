\chapter{励磁}
\section{引言}
本章我们使用Bean于1962年提出的唯象磁化理论来讨论第II类超导体的磁化问题。如第一章指出的,对多数超导磁体应用所关注的磁场范围($>\sim 0.5T$),第II类超导体
处于混合态,即在超导态的“海”中还存在正常态的“岛”。当第II类超导体处于时变磁场或时变传输电流中时,这些岛中将产生耗散,体现为磁通跳跃(一种暂态现象)或交流损耗。
所谓的Bean临界态模型,以闭式表达式阐明了消除磁通跳跃和最小化交流损耗的必要条件。

如今,已经有了可以完全消除磁通跳跃的生产LTS线/缆的成熟方法。我们在本章将学习到,磁通跳跃在HTS中并不像在LTS中是那么重要。如果仅在消除磁通跳跃方面磁化是重要的,
那在HTS应用中可将其视为次要考虑问题。然而,由于磁化在LTS和HTS的交流损耗中也起到重要作用,所以我们用一章来研究它。交流损耗将在第七章有更详细的讨论。

\section{第II类超导体的Bean理论}
\subsection{无传输电流}
和很多成功的理论一样,Bean模型通过一些假设,可用简单的数学推导出闭式表达式,与实验结果取得了很好的一致性。在Bean模型中,超导体有最简单的几何结构——
$x$方向宽度为$2a$,$y$和$z$向无限长。磁场($H, B, M$)指向y向,而电流($I, J$)在z向流动。在Bean模型中,$J=J_c$(临界电流密度),并假定其不依赖于磁场和温度。

于是,磁场本构关系可以简化为下式:
\begin{equation}
  M=\frac{B}{\mu_0} -H
\end{equation}

根据Bean模型,磁感应强度B在硬超导体内的次表面内不为0,而是等于超导体的体平均$\mu_0 H_s$,$H_s$是超导体内的磁场。

%%图5.1
\begin{figure}[htbp]
  \centering
 \includegraphics[scale=0.8]{chpt5/figs/fig5.1.eps}
  \caption{置于外磁场中的第II类超导体板}\label{fig:slabinfield}
\end{figure}
图\ref{fig:slabinfield}展示了第无限高($y$向)、无限深($z$向)、$2a$宽($x$向)第II类超导体板的。板此前未处于磁场中,外磁场$H_e$平行于板施加,将在板内产生$H_s(x)$。
根据安培定律$\nabla \times H = J=J_c$,我们可以得到超导体内的磁场$H_s(x)$:
%\begin{equation}
%  H_s(x)=
%  \begin{cases}
%           0, & \mbox{x*\le x \le x+ }  \\
%           H_e - J_c x, & \mbox{0\le x \le x* } \\
%           H_e + J_c (x-2a), & \mbox{x+ \le x \le 2a}
%   \end{cases}
%\end{equation}
注意到,$H_s(x)$的斜率等于$J_c$,当$J_c$大于0时(z向,朝向纸面外)大于0,$J_c$小于0时小于0。$x*$和$2a-x^+$给出磁场的穿透程度,表示为
\begin{equation}
  x*=\frac{H_e}{J_c}
\end{equation}

在$H_e=H_p\equiv J_c a$时,$x^*=x^+=a$,整个板处于临界态。$H_p$是所谓的穿透磁场,定义为
\begin{equation}
  H_p\equiv J_c a
\end{equation}

板内的平均磁感应强度由下式给出:
\begin{equation}
\begin{split}
\~{B}_s&=\frac{\mu_0}{2a}\int_{0}^{2a} H_s(x)dx \\
&=\frac{\mu_0}{2a}\times <\mbox{图5.1中的阴影面积}> \\
&=2\times \frac{\mu_0}{2a}\times \frac{H_e x^*}{2}=\frac{\mu_0 H_e^2}{2aJ_c}\\
&=\frac{\mu_0 H_e^2}{2H_p}
\end{split}
\end{equation}

根据定义$M=~{B}_s / \mu_0-H_e$,可得
\begin{equation}
  -M=H_e-\frac{H_e^2}{2H_p},(0\le H_e \le H_p)
\end{equation}

超导体是抗磁性的,-M是它的磁化强度。

随着外磁场的进一步增加,磁场将最终穿透整个板($H_e\ge H_p$),根据$~{B}_s=H_e-H_p/2$,有
\begin{equation}
  -M=\frac{1}{2}H_p=\frac{1}{2}J_c a, (H_e\ge H_p)
\end{equation}

图中的虚线磁化线对应$H_e=H_p$情况。
%5.2
\begin{figure}[htbp]
  \centering
 \includegraphics[scale=0.8]{chpt5/figs/fig5.2.eps}
  \caption{退场过程中的$H_s(x)$:$H_e\downarrow=H_m\rightarrow 0$}\label{fig:hreturn}
\end{figure}

图\ref{fig:hreturn}中的点线表示的是$H_s(x)$在$H_e=H_m>2H_p$时的情况。其中,$H_m$是外施磁场序列的最大值。

当$H_e$从$H_m$减至0的过程中,$H_s(x)$如图\ref{fig:hreturn}中的实线所示。当$H_e=H_m-2H_p$时,$-M$成为$-H_p /2$。
可以看到,外场从$H_m$到$H_e\downarrow=0$的退场过程中,$-M(H_e)$由下式给出
\begin{eqnarray}
% \nonumber % Remove numbering (before each equation)
  -M(H_e) =&\frac{1}{2}H_p-(H_m-H_e)+\frac{(H_m-H_e)^2}{4H_p}\\ \nonumber
                 & ,(H_e\downarrow=H_m\rightarrow H_m-2H_p) \\ \nonumber
  -M(H_e) =&-\frac{1}{2}H_p,\quad (H_e\downarrow=H_m-2H_p\rightarrow 0)
\end{eqnarray}

当外场施于“纯”板时,$-M$是$H_e$的二次函数。而在$H_e$退回0时,$-M(H_e)=-H_p /2$。“剩余”磁化如图\ref{fig:hreturn}中的虚划线所示。可知当置于外场中,
第II类超导体将会被磁化。剩余磁场不能通过外施磁场的方法去除。一种去除它的方法是加热超导体至临界温度$T_c$以上。
%5.3
\begin{figure}[htbp]
  \centering
 \includegraphics[scale=0.8]{chpt5/figs/fig5.3.eps}
  \caption{某硬超导体在外磁场($0\rightarrow H_{c2}\rightarrow 0$)下的磁化和磁场的关系。
  其中,实线表示$J_c=const$;虚线定性表示了电流随磁场下降的事实。}\label{fig:magvsh}
\end{figure}
图\ref{fig:magvsh}给出了完整的磁场从0增至$H_m=H_{c2}$又退回0的完整图像。其中,$H_{c2}$是超导体的上临界场。实线是基于由Bean的关于$J_c$不依赖磁场的假设
而导出的5.5-5.7式确定的。虚划线是对更接近实际情况的定性修正,反映了$J_c$随磁场衰减的事实,在$H_{c2}$时为0。注意到,磁化是有回滞的,在
$H_p<H_e<H_m-2H_p,\quad \Delta M=-M(H_e\uparrow)+M(H_e\downarrow)$范围内,磁场的幅值为$H_p=J_c a$。
于是,有时通过获得$J_e(H_e)$数据来做磁化的测量。
%5.4
\begin{figure}[htbp]
  \centering
 \includegraphics[scale=0.8]{chpt5/figs/fig5.4.eps}
  \caption{在图5.1给出的磁场$H_s(x)$下的$J(x)$}\label{fig:jtoh}
\end{figure}
图\ref{fig:jtoh}给出了在施加图\ref{fig:slabinfield}分布磁场下的板内电流分布。注意到$J_c=H_p /2a$。$y$向的单位长度净电流沿板的$z$向流动,由下式给出
\begin{equation}
  I=\int_{0}^{2a} J(x)dx=0
\end{equation}

\subsection{传输电流对励磁的效应}
当有传输电流$I_t$($y$向单位长度)在板中沿$+z$方向(流出纸面)时,我们看到在$x=2a$处磁场有一个$I_t/2$的增长,在$x=0$处有一个$I_t/2$的减少。

因为板内屏蔽电流是从每一个表面逐渐进入内部的,板内的场分布$H_s(x)$如图\ref{fig:hwithi}所示。图中的$x^*$和$x^+$由下式给出:
\begin{eqnarray}
% \nonumber % Remove numbering (before each equation)
  -\frac{1}{2}I_t + J_c x^* = 0 \\ \nonumber
  J_c(x^*-2a)+\frac{1}{2}I_t = 0 \\ \nonumber
  x^*=\frac{I_t}{2J_c}\quad \& \quad x^+ = 2a-\frac{I_t}{2J_c}
\end{eqnarray}

%%%%图5.5
\begin{figure}[htbp]
  \centering
 \includegraphics[scale=0.8]{chpt5/figs/fig5.5.eps}
  \caption{板内存在传输电流$I_t$时的磁场$H_s(x)$}\label{fig:hwithi}
\end{figure}

图5.6给出了板内的电流分布$J(x)$。沿着板宽度方向积分,我们可以得到板内的净电流就是$I_t$:
\begin{equation}
  I=\int_{0}^{2a}J(x)dx=J_c x^*+J_c(2a-x^+)=1/2 I_t +1/2 I_t=I_t
\end{equation}

\begin{figure}[htbp]
  \centering
 \includegraphics[scale=0.8]{chpt5/figs/fig5.6.eps}
  \caption{在图5.5给出的磁场$H_s(x)$下的$J(x)$}\label{fig:jtoh5.5}
\end{figure}

也即,板内的净电流就是外施电流。注意到,若外磁场$H_e\vec{i_y}$在$I_t$通入后施加,基本不会改变电流的分布(图5.5和5.6);但若外磁场
先于电流施加,则会出现不同的$H_s(x)$和$J(x)$。

\section{测量技术}
这里我们描述最经常使用的测量磁化的技术。图5.7指示了本项技术的关键组件:1)初级查找线圈;2)次级查找线圈;3)平衡分圧计。
图中未画出但也同等重要的是积分器,它将桥路输出电压$V_bg$转换为直接正比于$M(H_e)$的电压信号。测试样品置于初级查找线圈内,。
当初级查找线圈和次级查找线圈置于在两个线圈所占的空间内基本均匀的时变外磁场$H_e(t)$中,各查找线圈的端子上将出现感应
电压$V_{pc}(t)$和$V_{sc}(t)$:
\begin{eqnarray}
% \nonumber % Remove numbering (before each equation)
  V_{pt}(t) &=& \mu_0 N_{pc} A_{pc}\left[ \frac{dM}{dt}+(\frac{d\~{H}_e}{dt})_{pc}\right] \\ \nonumber
  V_{sc}(t) &=& \mu_0 N_{sc} A_{sc}\left( \frac{d\~{H}_e}{dt}\right)_{sc}
\end{eqnarray}

下标pc和sc分别表示初级线圈(primary coil)和次级线圈(second coil)。N是各线圈的匝数。A是耦合$H_e(t)$的每一匝线圈的有限面积。
$\~{H}_e$是磁场在各线圈内的空间平均值。

桥输出电压$V_bg$由下式给出:
\begin{equation}
  V_{bg}(t)=(k-1)V_{pt}(t)+kV_{sc}(t)
\end{equation}

其中,k是一个介于0-1的常数,表示分压计在初级线圈侧的分压系数(图5.7)。联立上两式,可得
\begin{equation}
  V_{bg}(t)=(k-1)\mu_0 N_{pc}A_{pc}\frac{dM}{dt}+(k-1)\mu_0 N_{pc}A_{pc}(\frac{d\~{H}_e}{dt})_{pc}+k\mu_0 N_{sc}A_{sc}(\frac{d\~{H}_e}{dt})_{sc}
\end{equation}

%%图5.7
\begin{figure}[htbp]
  \centering
 \includegraphics[scale=0.8]{chpt5/figs/fig5.7.eps}
  \caption{磁化测量原理图}\label{fig:magmeasure}
\end{figure}

通过调节分压系数k可以满足以下条件,令$V_{bg}(t)$正比于$dM/dt$:
\begin{eqnarray}
% \nonumber % Remove numbering (before each equation)
  &(k-1)\mu_0 N_{pc}A_{pc}(\frac{d\~{H}_e}{dt})_{pc}+k\mu_0 N_{sc}A_{sc}(\frac{d\~{H}_e}{dt})_{sc}=0 \\ \nonumber
  &V_{bg}(t)=(k-1)\mu_0 N_{pc}A_{pc}\frac{dM}{dt}
\end{eqnarray}

尽管实际上上式第一式所给的归零条件在很大范围内不是总能满足,但是第二式对多数情况都是很好的近似。
一般,$k$接近0.5。$V_bg{t}$馈入一个积分器,其输出正比于$M$。特别的,如果样品是“纯”的($M=0$),磁场
$H_e(t)$从0($t=0$)增($\uparrow$)至$H_e$($t=t_1$)时,我们有
\begin{equation}
  V_{mz}(H_e\uparrow)=\frac{1}{\tau_{it}}\int_{0}^{t_1}V_bg(t)dt=\frac{(k-1)\mu_0 N_{pc}A_{pc}}{\tau_{it}}M(H_e)
\end{equation}

式中,$\tau_{it}$是有效积分常数。如果$H_e>H_p$,此时有$M(H_e)=-H_p / 2=-J_c a / 2$,则上式简化为
\begin{equation}
    V_{mz}(H_e\uparrow>H_p)=-f_m \frac{(k-1)\mu_0 N_{pc}A_{pc}}{\tau_{it}}(\frac{J_c a}{2})
\end{equation}

因数$f_m$是磁性材料体积与样品总体积之比。之所以需要这个因数是因为待磁化测试的样品一般不全是由磁性材料组成。
比如多丝(层)导体,样品除了超导丝(层)外,还存在基底金属和其他非磁性材料。如果外场按$0\rightarrow H_m>H_p\rightarrow H_e\downarrow <H_m-2H_p$顺序,
我们有
\begin{equation}
    V_{mz}(H_e\downarrow<H_m-2H_p)=-f_m \frac{(k-1)\mu_0 N_{pc}A_{pc}}{\tau_{it}}(\frac{J_c a}{2})
\end{equation}

于是,$\Delta V_{mz}=V_{mz}(H_e>H_p)-V_{mz}(H_e\downarrow<H_m-2H_p)$正比于在$H_e$处磁化曲线的“宽度”:
\begin{equation}
    \Delta V_{mz}=-f_m \frac{(k-1)\mu_0 N_{pc}A_{pc}}{\tau_{it}} J_c a
\end{equation}

上式我们看出,$\Delta V_{mz}$是直接正比于$J_c$和$a$的。
%%图5.8
\begin{figure}[htbp]
  \centering
 \includegraphics[scale=0.7]{chpt5/figs/fig5.8.eps}
  \caption{$MgB_2$在$10K,20K,30K$三种温度下的磁化和磁场关系}\label{fig:magvfield}
\end{figure}
图5.8给出的是$MgB_2$在10K,20K,30K时,磁场按$0\rightarrow 1.7T\rightarrow 0 \rightarrow -1.7T\rightarrow 0$完整施加时的磁化与磁场的关系。注意到,
和图5.3不同,本图中还有$+M(H_e)$。因为凸显在$x$轴上(磁场)并不偏斜,我们可以认为本测试中初次、二次线圈已得到很好的平衡。

磁化的回滞表明,$MgB_2$是第II类超导体,它的抗磁性在每一个图线的第一部分(磁场从0增至1.7T时)明显可见。

从Bean模型可知,$H_p=J_c a$,即磁化直接正比于$J_c$。然而,实际上$J_c$不仅是磁场还是温度的减函数。图5.8中明显可见对$J_c$和$T$的依赖。
图中的$M$的单位是$emu/cm^3$,不是SI单位。

\section{专题}
\subsection{讨论5.1:传输电流磁化}
正如本书最初所述,在传输电流存在条件下的励磁依赖于外场和传输电流施加的顺序。
这里我们考虑三种情况:A) 先加磁场后加传输电流; B) 先通电流后加磁场;C) 磁场和电流交替改变。

\textbf{A.  先加磁场后加传输电流}

图5.9给出了厚度为$2a$的Bean板在施加如下特定磁场-电流序列后内部磁场$H_s(x)$的特征。
\begin{enumerate}
	\item 起始,$H_{s1}(x)$,有$H_e=2.5H_p$,无传输电流——点线。
	\item 接下来,$H_{s2}(x)$,通过传输电流$I_t=J_c=I_c/2$后,施加恒定外场——实线。其中,$J_c a=H_p$。
	\item 最后,$H_{s3}(x)$,传输电流进一步增加到$2J_c a=I_c$后,磁场$H_e =2.5 H_p$,最终$H_{s3}(0)=1.5 H_p$
	以及$H_{s3}(2a)=3.5H_p$——虚线。
\end{enumerate}

$H_{s1}(x)$和$H_{s3}(x)$是很直接的。$H_{s2}(x)$由三个分段函数$H_{s2_1}(x),H_{s2_2}(x),H_{s2_3}(x)$组成:
\begin{align*}% page321 第1个
H_{s2_{1}}(x)&=2H_{p}+J_{c}x=2J_{c}a+J_{c}x\qquad&(0\leq x\leq x*)\\
H_{s2_{2}}(x)&=2.5H_{p}-J_{c}x=2.5J_{c}a-J_{c}x\quad&(x*\leq x\leq x+)\\
H_{s2_{3}}(s)&=H_{p}+J_{c}x=J_{c}a+J_{C}x\qquad&(x+\leq x\leq 2a)
\end{align*}
式中,$x*$和$x+$由$H_{s1}(x)$和$H_{s2}(x)$的两个拐点给出。也即,$H_{s1}(x*)=H_{s2_1}(x*)$,
$H_{s1}(x+)=H_{s2_3}(x+)$:$x*=0.25a$,$x+=0.75a$。
\begin{figure}[htbp]
	\centering
	\includegraphics[scale=0.7]{chpt5/figs/fig5.9.eps}
	\caption{在$H_e=2.5H_p$下的磁场特征。首先,$I_t$=0(点线),然后$I_t=J_c a=I_c/2$(实线),最后$I_t=2J_c a=I_c$(虚线)。}
\end{figure}

图5.10给出了。。。。。

面积A1.。。。。。。

\begin{align*}
H_{s2_{1}}(X)&=(H_{e}-\frac{1}{2}I_{t})+J_{c}x\qquad&(0\leq x\leq x*)\\
H_{s2_{2}}(x)&=H_{e}-J_{c}x\qquad&(x*\leq x\leq x+)\\
H_{s2_{s}}(x)&=(H_{e}+\frac{1}{2}I_{t})+J_{c}(x-2a)\qquad&(x+\leq x\leq2a)
\end{align*}
我们解出$x*$和$x+$,确定$H_{s2_2}(x*)$和$H_{s2_2}(x+)$:
\begin{align*}
H_{{s}2_{1}}(x*)=H_{s2_{2}}(x*)\\
H_{e}-H_{p}i+J_{c}x*=H_{e}J_{c}x*\Rightarrow x*=\frac{H_{p}}{2J_{c}}i=\frac{1}{2}ai\\
H_{s2_{2}}(x*)=H_{e}-\frac{1}{2}aJ_{c}i=H_{e}-\frac{1}{2}H_{p}i
\end{align*}
以及
\begin{align*}
H_{s2_{2}}(x+)&=H_{s2_{3}}(x+)\\
H_{e}-J_{c}x+&=H_{e}+H_{p}i+J_{c}(x^{+}-2a)\Rightarrow x^{+}=a(1-\frac{1}{2}i)\\
H_{s2_{2}(x+)}&=H_{e}-H_{p}+\frac{1}{2}H_{p}i
\end{align*}

\begin{figure}[htbp]
	\centering
	\includegraphics[scale=0.7]{chpt5/figs/fig5.10.eps}
	\caption{用于励磁计算的有传输电流(实线)的磁场的特征。竖直点线分开三个区域:$A_1, A_2$和$A_3$。}
\end{figure}

M is proportional to the size of the “shaded area,” shown in Fig. 5.10, which is
the sum of three partitioned areas A1, A2, and A3.

The area of each trapezoid is its (base)×(height1+height2)/2.
\begin{align*}% page323 第1个
A_{1}&=\frac{1}{2}x*[H_{s1}(0)+H_{s2}(x*)]=\frac{1}{4}ai[(H_{e}-H_{p}i)+(H_{e}-\frac{1}{2}H_{p}i)]\\
&=\frac{1}{4}ai(2H_{e}-\frac{3}{2}H_{p}i)  \\
&=a(\frac{1}{2}H_{e}i-\frac{3}{8}H_{P}i^{2})
\end{align*}
\begin{align*}% page323 第2个
A_{2}&=\frac{1}{2}(x^{+}-x*)[H_{s2}(x*)+H_{s2}(x^{+})]\\
&=\frac{1}{2}(a-ai)(H_{e}-\frac{1}{2}H_{p}i+H_{e}-H_{p}+\frac{1}{2}H_{p}i)\\
&=\frac{1}{2}a(-i)(2H_{e}-H_{p})\\
&=a(H_{e}-H_{e}i-\frac{1}{2}H_{p}+\frac{1}{2}H_{p}i)
\end{align*}
\begin{align*}% page323 第3个
A_{3}&=\frac{1}{2}(2a-x+)[H_{s2}(x+)+H_{s3}(2a)]\\
&=\frac{1}{2}(a+\frac{1}{2}ai)(H_{e}-H_{p}+\frac{1}{2}H_{p}i+H_{e}+H_{P}i)\\
&=a(1+\frac{1}{2}i)(H_{e}-\frac{1}{2}H_{p}+\frac{3}{4}H_{p}i)\\
&=a(H_{e}+\frac{1}{2}H_{e}i-\frac{1}{2}H_{p}-\frac{1}{4}H_{p}i+\frac{3}{4}H_{p}i+\frac{3}{8}H_{p}i^{2})
\end{align*}
By combining these three areas, we may compute the shaded area:
\begin{align*}% page323 第4个
Shaded\quad area=&A_{1}+A_{2}+A_{3}\\
=&a(\frac{1}{2}H_{e}i-\frac{3}{8}H_{p}i^{2}+H_{e}-H_{e}i-\frac{1}{2}H_{p}+\frac{1}{2}H_{p}i\\
&+H_{e}+\frac{1}{2}H_{e}i-\frac{1}{2}H_{p}-\frac{1}{4}H_{P}i+\frac{3}{4}H_{P}i+\frac{3}{8}H_{p}i^{2})\\
=&a(2H_{e}-H_{p}+H_{p}i)
\end{align*}
With the shaded area known, M can be computed quickly:
\begin{align*}% page323 第4个
-M(i)&=H_{e}-\frac{1}{2a}\times(Shaded\quad area)\\
&=H_{e}-H_{e}+\frac{1}{2}H_{p}-\frac{1}{2}H_{p}i\\
&=\frac{1}{2}H_{p}(1-i)\qquad(5.17a)\\
&=-M(0)f_{1}(i)\qquad(5.17b)
\end{align*}
where f1(i) = 1 − i. −M(i) decreases linearly with i, becoming 0 at i = 1.

\textbf{B. 先通电流后加磁场}

这里向板加外部磁场和传输电流顺序是反过来的。。。。。。。

在图4.11中,。。。。。

开始,$H_e=0$:
\begin{align*}% page324 第1个
I_{t}=\int_{0}^{2a}J(x)dx=J_{c}(0.5)+J_{c}(2a-1.5a)=J_{c}a
\end{align*}
接下来,$H_e=2H_p$:
\begin{align*}% page324 第2个
I_{t}=\int_{0}^{2a}J(x)dx=-J_{c}(0.5a)+J_{c}(2a-0.5a)=J_{c}a
\end{align*}
为了确定板内的励磁。。。。。。。
\begin{align*}
H_{s1}(x)&=(H_{e}-H_{p}i)-J_{c}x\qquad(0\leq \leq x*)\\
H_{s2}(x)&=(H_{e}+H_{p}i)+J_{c}(x-2a)\quad(x*\leq x leq  2a)
\end{align*}
因为,。。。。。。
\begin{align*}% page324 第5个
x*=a-ai=a(1-i)
\end{align*}

\begin{figure}[htbp]
	\centering
	\includegraphics[scale=0.7]{chpt5/figs/fig5.11.eps}
	\caption{磁场特征,点线表示仅有电流,实现表示有磁场和电流。}
\end{figure}

Once x∗ is determined, we can compute Hs1(x∗):
\begin{align*}% page325 第1个
H_{s1}(x*)=H_{e}-H_{p}i-J_{c}a(1-i)=J_{e}-H_{p}
\end{align*}
We can now compute the shaded area, which is the sum of two areas, A1 and A2,
partitioned by the vertical dashed line in Fig. 5.12.
\begin{align*}% page325 第2个
A_{1}&=\frac{1}{2}a(1-i)(H_{e}-H_{p}i+H_{e}-H_{p}\\
&=a(1-i)(H_{e}\frac{1}{2}H_{p}-\frac{1}{2}H_{p}i)\\
&=a(H_{e}-H_{e}i\frac{1}{2}H_{p}+\frac{1}{2}H_{p}i^{2})
\end{align*}
\begin{align*}% page325 第3个
A_{2}&=\frac{1}{2}(2a-a+ai)(H_{e}+H_{p}i+H_{e}-H_{P})\\\notag
&=a(1+i)(H_E-\frac{1}{2}H_{p}+\frac{1}{2}H_{p}i)\\\notag
&=a(H_{e}+H_{e}-\frac{1}{2}H_{p}+\frac{1}{2}H_{p}i^{2})
\end{align*}
\begin{align*}% page325 第4个
Shaded\quad area&=A_{1}+A_{2}\\
&=a(2H_{e}-H_{p}+H_{p}i^{2})
\end{align*}
Once the shaded area is known, we have M:
\begin{subequations}
	\begin{align*}
-M(i)&=H_{e}-\frac{1}{2}(2H_{e}-H_{p}+H_{P}i^{2}\\\notag
&=\frac{1}{2}H_{p}(1-i^{2})\\
&=-M(0)f_{2}(i)
	\end{align*}
\end{subequations}
式中,$f_2(i) = 1 − i^2$。励磁时电流$i$的二次函数。

\begin{figure}[htbp]
	\centering
	\includegraphics[scale=0.7]{chpt5/figs/fig5.12.eps}
	\caption{用以计算励磁的同时有传输电流和磁场的场特性磁场特征。竖直点线分出两个区域$A_1$和$A_2$。}
\end{figure}


\textbf{C. 磁场和电流交替改变}

Finally, we shall consider Hs(x) and −M(i) for the slab when the following sequence
of field and transport current is applied.

\begin{enumerate}
	\item 开始
	\item 当
	\item 当
	\item 现在
	\item 再一次
\end{enumerate}

Figure 5.13 shows the field profile Hs(x) after Step 5, consisting of five piece-wise
solid lines, the second and third of which, useful to compute M(i), are given below.
\begin{align*}
H_{s2}(x)&=H_{e}+H_{p}i-J_{c}x\qquad(x*\leq x\leq a)\\
H_{s3}(x)&=H_{e}+H_{p}i+J_{c}(x-2a)\quad(a\leq x\leq a^{+})
\end{align*}
where x∗ and x may be solved from: Hs2(x∗)=Hs3(x )=He. Thus:
\begin{align*}
H_{s2}(x*)=H_{e}\Rightarrow H_{e}+H_{p}i-J_{c}x*\\
x*=\frac{H_{p}i}{J_{c}}=ai\\
H_{s3}(x+)=H_{e}\Rightarrow H_{e}+H_{P}i+J_{c}(x^{+}-2a)\\
x^{+}=2a-\frac{H_{p}}{J_{c}}i=2a-ai
\end{align*}

\begin{figure}[htbp]
	\centering
	\includegraphics[scale=0.7]{chpt5/figs/fig5.13.eps}
	\caption{第五步之后的磁场特征。}
\end{figure}

The magnetization is computed from appropriate areas, shown in Fig. 5.14, in
which the slab is divided into four “white” areas, from left to right, designated A1
(rectangle), A2 (trapezoid), A3 (trapezoid), and A4 (rectangle minus “triangle”).
In the figure, “base” and “height” are given by:
\begin{align*}
base&=x+-x*=(2a-ai)-ai=(a(1-i)\\
height&=H_{e}-H_{s2}(a)=H_{e}-(H_{e}+H_{p}i-J_{c}a)\\
&=J_{c}a-J_{p}i=H_{p}(1-i)
\end{align*}
The two “dotted” areas in Fig. 5.14 are equal in magnitude but have “opposite”
signs, hence they cancel out when we perform the area integral. The sum of the
areas, A1, A2, A3, and A4, is given by:
\begin{align*}
\sum_{j=1}^{4}A_{j}&=2aH_{e}-crossed\quad area\\
crossed\quad area&=\frac{1}{2}(base)\times(height)\\
\sum_{j=1}^{4}A_{j}&=2aH_{e}-\frac{1}{2}2a(1-i)H_{p}(1-i)\\
&=2aH_{e}-aH_{p}(1-i)^{2}
\end{align*}
The magnetization, −M(i), is thus given by:
\begin{align*}% page327 第6个
-M(i)&=H_{e}-\frac{1}{2a}[2aH_{e}-a \grave{}H_{p}(1-i)^{2}]\\
&=\frac{1}{2}H_{p}(1-i)^{2}
\end{align*}
\begin{subequations}
	\begin{align*}
-M(i)=&-M(0)(1-i)^{2}\\
=&-M(0)f_{3}(i)
	\end{align*}
\end{subequations}
式中,$f_3(i) = (1 − i)^2$。

\begin{figure}[htbp]
	\centering
	\includegraphics[scale=0.7]{chpt5/figs/fig5.14.eps}
	\caption{用以计算励磁的第五步之后的磁场特征。}
\end{figure}

\begin{figure}[htbp]
	\centering
	\includegraphics[scale=0.6]{chpt5/figs/fig5.15.eps}
	\caption{讨论5.1研究的三种归一化励磁 vs. 归一化电流关系。}
\end{figure}

\textbf{Magnetization Functions—Summary}

Figure 5.15 presents three normalized magnetization functions, f1(i), f2(i), and
f3(i), where i = It/Ic. It is interesting to note how different sequences of transport
current and external field applications affect M(i). These f(i) functions were validated
with experimental results [5.3, 5.4], thereby making Bean’s model accepted
quite quickly after its formulation.



\subsection{讨论5.2:SQUID用于磁化测量}
A SQUID (Superconducting Quantum Interference Device), based on the principle
of the Josephson effect, is an electronic device that can be used to measure
changes in magnetic field with extremely high resolution—individual flux quanta
of magnitude 2.0×10−15 Wb(Tm2).

A typical SQUID magnetization measurement setup consists of a test sample, at
a constant temperature, placed in a uniform field. The test sample is moved back
and forth in the uniform field; during each cycle it cuts through measurement
coils, one located at one end of the test sample and the other located at the other
end. The induced current in each measurement coil is measured by the SQUID in
terms of the field generated by the current, which, in turn, is a measure of the test
sample’s magnetization. Because SQUIDs operate best in low-field environments
(perhaps no higher than ∼100 oersted or ∼0.01 T), they are usually shielded from
the high-field environment of the test sample.



\subsection{讨论5.3:“Bean细丝”中的磁化}
\subsubsection{第一部分:磁场平行于细丝轴}
For an infinitely long superconducting filament of diameter df subjected to an
external magnetic field parallel to the filament’s axis (z),$H_e\vec{i_z}$, we may use the
same assumptions as Bean to derive expressions for its magnetization. For an
infinitely long filament exposed to$H_e\vec{i_z}$, Ampere’s law (Eq. 2.5) is given by:
\begin{equation}% page329 第1个
\frac{dH_{z}}{dr}\vec{\imath}_{\theta}=-J_{c}\vec{\imath}_{\theta}
\end{equation}
Equation 5.20 states that an axially (z) directed magnetic field within the filament,
Hs(r), is a linear function of r with a slope of Jc.

\textbf{A. 初始态}

For He ≤ Hp, where Hp is the critical-state field, the field within the filament,
Hs(r), is zero from r=0 to r∗=(df/2−He/Jc) and varies as Jcr from r∗ to df /2:
\begin{equation}% page329 第2个
H_{s}(r)=H_{e}\frac{r-r*}{\frac{d_{f}}{2}-r*}
\end{equation}
Note that r∗=0 when He=Hp, where Hp is the critical-state field:
\begin{equation}% page329 第3个
H_{p}=\frac{1}{2}J_{c}d_{f}
\end{equation}
Using steps similar to those taken with Eq. 5.4, we may compute the average
magnetic induction within the filament,$\~{B}_s$:
\begin{align}% page329 第4个
\tilde{B_{s}}&=\frac{4\mu_{o}}{\pi d_{f}^{2}}\int_{r*}^{\frac{d_{f}}{2}}H_{e}\frac{r-r*}{\frac{d_{f}}{2}-r*}(2\pi r)dr\\
&=\frac{8\mu_{o}H_{e}}{d_{f}^{2}(\frac{d_{f}}{2}-r*)}(\frac{1}{24}d_{f}^{3}-\frac{1}{8}d_{f}^{2}r*+\frac{1}{6}r*^{3})
\end{align}
Unlike in the case of a slab, where the integration may be performed geometrically
from Hs(x), here the “area” integration must be performed mathematically. By
inserting r∗=(df/2−He/Jc) into Eq. 5.23 and noting that Hp=Jcdf/2, we obtain:
\begin{equation}%page329 第5个
\frac{\tilde{B}_{s}}{\mu_{o}}=\frac{2H^{2}_{e}}{d_{f}J_{c}}-\frac{4H^{3}_{e}}{3(d_{f}J_{c})^{2}}=\\frac{H^{2}_{e}}{H_{P}}-\frac{H_{e}^{3}}{3H^{2}_{p}}
\end{equation}
根据定义$M=\~{B}_s/\mu_{0}-H_e$,我们有:
\begin{equation}%page329 第6个
-M=H_{e}-\frac{H^{2}_{e}}{H_{P}}+\frac{H^{3}_{e}}{3H^{2}_{p}}\quad(0\leq H_{e}\leq H_{P})
\end{equation}
Note that Eq. 5.25 is similar to, but clearly different from, Eq. 5.5 for the slab.

\textbf{B. 临界态及以上}

For He≥Hp the filament is in the critical state, and its magnetization is constant
and given from Eq. 5.25 with He=Hp:
\begin{equation}%page329 第7个
-M=\frac{1}{3}H_{p}=\frac{1}{3}(\frac{J_{c}d_{f}}{2})\quad(H_{e}\geq H_{p})
\end{equation}
The “magnetization factor” for the filament is 1/3; for the slab it is 1/2 (Eq. 5.6).

\subsubsection{第二部分:磁场垂直于细丝轴}
When the applied external field is perpendicular to the axis of a filament of diameter
df , the current distribution within the filament is complicated for He ≤Hp,
the critical field. For He≥Hp, a total current of Jcπd2
f/8 is induced flowing in the
+z-direction, and the same magnitude in the −z-direction. Figure 5.16 shows the
current distributions for: a) a Bean slab (2a); and b) a filament of diameter df .

We may compute the magnetization, M, by integrating the magnetic moment, mA,
per unit volume. Here, we derive expressions for the critical state magnetization
for a Bean slab of width 2a and a filament of diameter df .

\textbf{A. Bean板}

For a Bean slab in the critical state, the magnetic moment mA per unit length in
both z- and y-directions, from Jc(x) shown in Fig. 5.16a, is given by:
\begin{equation}%page330 第1个
m_{A}=\int_{0}^{a}2xJ_{c}(x)dx=J_{c}a^{2}
\end{equation}
The conductor volume per unit length in both z- and y-directions is 2a. Thus:
\begin{align*}%page330 第2个
M=\frac{m_{A}}{2a}=\frac{1}{2}J_{c}a\tag{5.27b}
\end{align*}
M given by Eq. 5.27b, except for the sign, is identical to that given by Eq. 5.6.

\textbf{B. 细丝}

For a filament of diameter df , the magnetic moment mA per unit length in the
z-direction, from Jc(x, y) shown in Fig. 5.16b, is given by:
\begin{equation}%page330 第3个
m_{A}=\int_{-\frac{d_{f}}{2}}^{\frac{d_{f}}{2}}2xJ_{c}(x,y)dxdy=\frac{1}{6}J_{c}d_{f}^{2}
\end{equation}
The conductor volume per unit length in the z-direction is πd2
f /4. Thus:
\begin{subequations}
	\begin{align*}
M&=\frac{\mathbf{4m_{A}}}{\pi d_{f}^{2}}=(\frac{4}{3\pi})J_{c}(\frac{d_{f}}{2})\simeq0.424J_{c}(\frac{d_{f}}{2})\sim0.5J_{c}a\\
H_{p}&=(\frac{8}{3\pi})J_{c}(\frac{d_{f}}{2})
	\end{align*}
\end{subequations}
This is nearly the same (8/3π∼1) as that for a Bean slab of thickness df .

\begin{figure}[htbp]
	\centering
	\includegraphics[scale=0.6]{chpt5/figs/fig5.16.eps}
	\caption{Induced current distributions in a) Bean slab of width 2a and b) an infinitely
		long filament of diameter dd, both subjected to an external field He in the y-direction.。}
\end{figure}

\subsection{讨论5.4:磁化中的$J_c$}
We demonstrate here how critical current density (Jc) data may be extracted
from magnetization (M) data. This method of extracting Jc data from M data
is quite useful when dealing with superconductor test samples too short for a
standard voltage vs. current measurement technique. Test samples too small for
V (I) measurement were common in the early days of HTS, and Bean’s model
discussed above was extremely useful.

In a V (I) measurement the sample must be “long” to: 1) generate a detectable
voltage with the very low electric field that defines the superconducting-to-normal
transition—the typical criterion is between 0.1 to $1\ \mathrm{\mu V/cm}$; and 2) keep the contact
resistance to the lead wires at each end of the test sample “low,” thereby preventing
excessive heating at the ends which might cause a premature normal transition.
The test samples should normally be at least 10mm long; perhaps under certain
circumstances they can be as short as 5 mm, but not much shorter than this. It
depends largely on the level of critical current.

Figure 5.8 presented the magnetization vs. applied field traces at 10 K, 20 K, and
30K of a short length (15mm) of copper/MgB2 composite wire of an equivalent
circular diameter of 1.038 mm[5.7]; the MgB2 itself has a diameter of 0.531 mm.
Here, the unit of magnetization is given in emu/cm3 corresponding to the total
wire diameter of 1.038 mm. The external field is along the wire axis, i.e., the same
configuration as in DISCUSSION 5.3 Part 1. To compute the superconductor’s Jc,
for example, at 10K in zero field, we treat the wire as a Bean rod of infinite length
and 0.531mm diameter. First, we convert emu/cm3 into the SI unit equivalent,
A/m, by multiplying it by 1000. (See Appendix I.)

At 10K in zero field, the magnetization, from Fig. 5.8, is 60 emu/cm3 or 60 kA/m.
To translate this to M corresponding to the volume of just the MgB2, we must
multiply 60 kA/m by (1.038/0.531)2 = 4.0. Solving Eq. 5.26 for Jc with M =
240 kA/m and df =5.31×10−4 m, we obtain:
\begin{align*}%page331 第5个
J_{c}(0\ \mathrm{T};10\ \mathrm{K})&=\frac{6\ \mathrm{M}(0\ \mathrm{T};10\ \mathrm{k})}{d_{f}}\\
&=\frac{6(240\times10^{3}\ \mathrm{A/m})}{(0.531\times10^{-3}\ \mathrm{m})}\\
&a=2.7\times10^{9}\ \mathrm{A/m^{2}}
\end{align*}


\subsection{问题5.1:磁化测量}
This problem applies the magnetization measurement technique discussed in 5.4 to
one of the four superconductors used in the Hybrid III SCM, to confirm that there
would be no flux jumping. The absence of flux jumping is one of the necessary
conditions for magnets that are not “cryostable”—this point will be discussed in
more detail in CHAPTER 6.

Table 5.1 presents specifications of the superconductor, a bare NbTi composite
strip with overall dimensions of 9.2mm width and 2.6mm thickness. (Not all
parameters in the table, e.g., twist pitch, are relevant for this problem.)

The test sample consisted of 52 (13×4) 100-mm long strips assembled in a rectangular
solid of square cross section, 38mm×38 mm, as shown in Fig. 5.17. Each
bare strip was electrically insulated with a thin tape. In the orientation shown
in Fig. 5.17a, each strip presents its narrow surface to the external magnetic induction
Be; in the orientation shown in Fig. 5.17b, each strip is broadside to
Be. The test sample assembly was placed inside a rectangular-bore (cross section
107mm×42mm) search coil set containing a primary search coil and two secondary
coils (Fig. 5.17c). The test assembly midplane coincided with that of the primary
search coil, whose midplane coincided with that of an external magnet generating
Be. The midplane-to-midplane distance between the primary and one of the
secondary coils was 70 mm. The primary coil had 500 turns of fine copper wire
over an axial distance of 40mm centered on its midplane; each secondary search
coil had 280 turns, extending an axial distance of 20mm centered on its midplane.
The turn density in the axial direction in each search coil was uniform.

When an external magnetic induction Be was swept at a rate of 0.05 T/s between
0 and 5T with the test sample at 4.2K with its orientation as in Fig. 5.17a, a
plot of −M (given in Vmz) vs. Be plot similar to that shown in Fig. 5.8 was
obtained. +Vmz is the integrator output proportional to −M, the negative of
the test sample magnetization. The effective integration time,$\tau_{it}$, was 1 s; the
balancing potentiometer’s constant k was 0.5. Assume negligible voltage drift.

\colorbox{red}{表5.1}


\begin{figure}[htbp]
	\centering
	\includegraphics[scale=0.6]{chpt5/figs/fig5.17.eps}
	\caption{Magnetization measurement details, dimensions in mm.
		(a) Each strip presents its narrow surface to the external magnetic
		induction, Be; (b) Each strip is broadside to Be; (c) Search coil setup.。}
\end{figure}

a) Make a ballpark estimate of ΔVmz at Be∼2.5T (magnetization trace “width”
in Fig. 5.8, given in volts). Note that $\tau_{it}$ = 1 s and k = 0.5. Assume df = 2a,
where df is the filament diameter and 2a is the width of the Bean slab.

b) A 1.8-K measurement was performed by pumping on the cryostat and reducing
the liquid helium bath pressure to 12.6 torr. The technician who controlled
the cryostat pressure noticed that pressure control was more difficult,
because of an increased liquid boil-off rate, when the test sample orientation
was as in Fig. 5.17b rather than as in Fig. 5.17a. Is this an aberration or
does his observation make sense? Explain.

c) The z-component of the external induction Be over the radial space occupied
by the search coil may be approximated to vary as:
\begin{equation}%page333 第1个
B_{e}(z)\simeq B_{e}(0)[1-c(\frac{z}{z_{o}})^{2}]\qquad(5.30)
\end{equation}
where z◦ = 75 mm. Based on information you have, compute the value of c.

\subsubsection{问题5.1之解}
a) Equation 5.13 indicates that search coils need to be balanced; otherwise, a
term proportional to the applied field contributes to the apparent magnetization.
Since the −M(H) trace shown in Fig. 5.8 is not tilted, the search coils are balanced.
From Eq. 5.14b:
\begin{align*}%page334 第1个
V_{bg}(t)=(k-1)\mu_{o}N_{pc}A_{pc}\frac{dM}{dt}\tag{5.14b}
\end{align*}
We have from Eq. 5.16c:
\begin{align*}%page334 第2个
\Delta V_{mz}=-f_{m}\frac{(k-1)\mu_{o}N_{pc}A_{pc}}{\tau_{it}}J_{c}a\tag{5.16c}
\end{align*}
We have: k = 0.5; τit = 1s; Npc = 500; Apc = (13)(0.1 m)(2.6×10−3 m) =
3.38×10−3 m2 [also acceptable is (0.1m)×(38×10−3 m) = 3.8×10−3 m2]; fm =(NbTi
volume)/total composite volume)= $1/(\gamma_{c/s} + 1)$ = 0.25.

\textbf{$J_c$的估计(4.2 K,2.5 T)}

From Table 5.1 we have Jc at 4.2K and 5T of 2.0×109 A/m2. It is generally
accepted that for a given temperature, Jc(Be) may be approximated, based on
Eq. 1.3, by:
\begin{align*}%page334 第3个
2.0\times10^{9}A/m^{2}=\frac{J_{0}B_{0}}{5T+0.3T}\Rightarrow J_{0}B_{0}=10.6\times 10^{9}\quad\ \mathrm{AT/m^{2}}
\end{align*}
where for NbTi, B0 ∼ 0.3T. J0 is the zero-field critical current density, which is
usually difficult to measure. Thus from the Jc value at 5T and B0 = 0.3T, we
can first solve for J0B0:
\begin{align*}%page334 第4个
2.0\times 10^{9}\ \mathrm{A/m^{2}}=\frac{J_{0}B_{0}}{5T+0.3T}\Rightarrow J_{0}B_{0}=10.6\times 10^{9}\ \mathrm{A/m^{2}}
\end{align*}
Once J0B0 is known, then Jc may be solved at 2.5 T. Thus:
\begin{align*}%page334 第5个
J_{c}(2.5T)=\frac{10.6\times10^{9}\ \mathrm{AT/m^{2}}}{2.8T}=3.8\times^{9}\ \mathrm{A/m^{2}}
\end{align*}
Inserting appropriate values into Eq. 5.16c, we have:
\begin{align*}%page334 第6个
\Delta V_{mz}&=-0.25\frac{(-0.5)(4\pi\times 10^{-7}\ \mathrm{H/m})(500)(3.38\times 10^{-3}\ \mathrm{m^{2}})}{1\ \mathrm{s}}\\
&\times(3.8\times 10^{9} A/m^{2})(50\times 1^{-6} m)\\
&\simeq 50\ \mathrm{mV}
\end{align*}

Because the strip is flattened from a round conductor by a process that squeezes
the conductor between rollers, the projected diameter of filaments in the direction
parallel to Be would be actually slightly less than the equivalent circular-area
radius, a = 50 $\mathrm{\mu m}$, which is used in the above computation for ΔVmz. If a radius
less than 50 $\mathrm{\mu m}$is used, ΔVmz would be less than 50 mV.

b) The anisotropic shape of the NbTi filaments makes magnetization in the orientation
of Fig. 5.17b greater than that in the orientation of Fig. 5.17a—the “effective”
a is greater. Thus there will be more magnetization loss.

If the aspect ratio of the filaments is the same as for the conductor, eddy current
loss will be proportional to$(a\.{H}_e)^2$in the orientation of Fig. 5.17b and$(b\.{H}_e)^2$
in the orientation Fig. 5.17a—review PROBLEM 2.8. Thus eddy current loss is
greater by a factor of (9.2/2.6)2 = 12.5 for Fig. 5.17b than for Fig. 5.17a.

The increased heat load on the helium due to higher magnetization and eddy
current losses causes a higher liquid helium boil-off rate; thus the technician’s
observation makes sense.

c) With balanced search coils, we have:
\begin{align*}%page335 第1个
N_{pc}A_{pc}(\frac{d\tilde{B}_{e}}{dt})_{pc}=N_{sc}A_{sc}(\frac{d\tilde{B}_{e}}{dt})_{sc}\tag{S1.2}
\end{align*}
%Because Apc = Asc, we have: Npc[˜Be]pc = Nsc[˜Be]sc. From symmetry, we consider
only the upper half (the unit mm is omitted in the following equations):
\begin{align*}%page335 第2个
[\tilde{B}_{e}]_{pc}=\frac{B_{e}(0)}{20}\int_{0}^{20}[1-c(\frac{z}{z_{0}})^{2}]dz\tag{S1.3a}
\end{align*}
\begin{align*}%page335 第3个
[\tilde{B}_{e}]_{sc}=\frac{B_{e}(0)}{20}\int_{60}^{80}[1-c(\frac{z}{z_{0}})^{2}]dz\tag{S1.3b}
\end{align*}
The Npc%[˜Be]pc = Nsc[˜Be]sc equality gives:
\begin{align*}%page335 第4个
\frac{250}{20}\int_{0}^{20}[1-c(\frac{z}{z_{0}})^{2}]dz&=\frac{280}{20}\int_{60}^{80}[1-c(\frac{z}{z_{0}^{2}})]dz\\\tag{S1.4}
250[20-\frac{c}{3}\frac{(20)^{3}}{(75)^{2}}]dz&=280[80-\frac{c}{3}\frac{(80)^{3}}{(75)^{2}}-60+\frac{c}{3}\frac{(60)^{3}}{(75)^{2}}]\\
	5000-118.5c&=22400-8495.4c-16800+1584c\\
	c&\simeq\frac{600}{4793}\simeq 0.125
\end{align*}


\subsection{讨论5.5:磁扩散和热扩散}
Before studying the flux jump criterion next in PROBLEM 5.2, we derive here basic
equations of magnetic and thermal diffusion to identify two diffusivities: magnetic
diffusivity, Dmg, and thermal diffusivity, Dth. The relative sizes of these two
diffusivities are quite different for electrically conductive normal metals ($D_{th}\gg D_{mg}$) and for Type II superconductors ($D_{th}\ll D_{mg}$). This condition of $D_{th}\ll D_{mg}$
in Type II superconductors makes the penetration of flux into a superconductor
an adiabatic process, leading, as we shall study in PROBLEM 5.2, to the criterion
for flux jumping.

To derive the magnetic diffusion equation, the applicable Maxwell’s equations are
Ampere’s law and Faraday’s law, both in differential forms:
\begin{align*}%page336 第1个
Ampere's\quad law:\quad \nabla\times\vec{H}=\vec{J}_{f}\tag{2.5}
\end{align*}
\begin{align*}%page336 第2个
Faraday's\quad law:\quad\nabla\times\vec{E}=-\frac{\partial\vec{B}}{\partial t}\tag{2.8}
\end{align*}
For the slab (width 2a) geometry, we can express Eqs. 2.5 and 2.8 as:
\begin{equation}%page336 第3个
Amperer's\quad law:\frac{\partial H_{y}}{\partial x}=J_{z}=\frac{E_{z}}{\rho_{e}}
\end{equation}
\begin{equation}%page336 第4个
Faraday's\quad law:\frac{\partial E_{z}}{\partial x}=\frac{\partial B_{y}}{\partial t}=\mu_{o}\frac{\partial H_{y}}{\partial t}
\end{equation}
where$\rho_e$is the material’s electrical resistivity. From Eqs. 5.31 and 5.32, we obtain:
\begin{align*}%page336 第5个
\rho_{e}\frac{\partial^{2}H_{y}}{\partial x^{2}}=\mu_{o}\frac{\partial H_{y}}{\partial t}
\end{align*}
\begin{equation}%page336 第6个
\frac{\rho_{e}}{\mu_{o}}\frac{\partial^{2}H_{y}}{\partial x^{2}}\equiv D_{mg}\frac{\partial^{2}H_{y}}{\partial x^{2}}=\frac{\partial H_{y}}{\partial t}
\end{equation}
Equation 5.33 is a magnetic diffusion equation, for which:
\begin{equation}%page336 第7个
D_{mg}=\frac{{\rho}_{e}}{\mu_{o}}
\end{equation}
Similarly, the one-dimensional thermal diffusion equation having constant thermal
properties is given by:
\begin{equation}%page336 第8个
k\frac{\partial^{2}T}{\partial x^{2}}=C\frac{\partial T}{\partial t}
\end{equation}
where k and C are, respectively, the material’s thermal conductivity and heat
capacity. Dividing both sides of Eq. 5.35a by C, we obtain:
\begin{align*}%page336 第9个
\frac{k}{C}\frac{\partial^{2}T}{\partial x^{2}}\equiv D_{th}\frac{\partial^{2}T}{\partial x^{2}}=\frac{\partial T}{\partial t}\tag{5.35b}
\end{align*}
Equation 5.35b is a thermal diffusion equation, for which:
\begin{equation}%page336 第10个
D_{th}=\frac{k}{C}\qquad(5.36)
\end{equation}
Note that Eq. 5.36 and 4.20 are equivalent, because$C=\rho c_p$

\colorbox{red}{表5.2}

Table 5.2 presents approximate values of electrical and thermal properties and
corresponding diffusivities at 4K and 80K for stainless steel and copper. From
Table 5.2 we can clearly see that stainless steel, a stand-in for normal-state superconductors,
and copper are opposite with respect to magnetic and thermal
diffusivities. Specifically, changes in magnetic field propagate quickly through
stainless steel, whereas temperature gradients are relatively slow to propagate;
hence, large nonuniform temperature distributions can be created in stainless steel
during changing magnetic fields. Physically, it means that magnetic heating happens
essentially adiabatically in Type II superconductors. In copper, the reverse
is true: the magnetic field diffuses very slowly, while any nonuniformity in temperature
is quickly “evened out.” Therefore, copper in intimate contact with Type
II superconductor can alleviate field-motion-induced instability in Type II superconductors.
This thinking is the essence of dynamic stability, one of the stability
criteria developed during the 1960s and 1970s [5.8] and applied also to HTS in the
late 1980s [5.9].

\subsection{问题5.2:磁通跳跃判据}

\begin{equation}%page338 第1个
e_{\phi}=\frac{\mu_{o}J_{c}|\Delta J_{c}|a^{2}}{3}\quad(5.37)
\end{equation}
\begin{equation}%page338 第2个
J_{c}(T)=J_{c_{o}}(\frac{T_{c}-T}{T_{c}-T_{op}})\quad(5.38)
\end{equation}
\begin{equation}%page338 第3个
\Delta J_{c}=-J_{c_{o}}(\frac{\Delta T}{T_{c}-T_{op}}\quad(5.39)
\end{equation}
\begin{equation}%page338 第4个
a_{c}=\sqrt{\frac{3\tilde{C}_{s}(T_{c}-T_{op})}{\mu_{o}J_{c_{o}}^{2}}}\quad(5.40)
\end{equation}

\subsubsection{问题5.2之解}

\begin{equation}%page339 第1个
H_{s1}(x)=H_{e}+J_{c}(x-a)\quad(S2.1a)
\end{equation}
\begin{equation}%page339 第2个
H_{s2}(x)=H_{e}+(J_{c}-| \Delta J_{c}|)(x-a)(S2.1b)
\end{equation}
\begin{equation}%page339 第3个
\oint_{C}\vec{E}\cdot d\vec{s}=-\mu_{o}\int_{S}\frac{\triangle H_{s}(x)\vec{\imath}_{y}\cdot d\vec{A}}{\Delta t}
\end{equation}
\begin{equation}%page339 第4个
\Delta H_{S}(x)=H_{s2}(x)-H_{s1}(x)\\
=|\Delta J_{c}|(a-x)\qquad(S2.3)
\end{equation}
\begin{equation}%page339 第5个
E_{z}(x)=\mu_{o}\frac{|\Delta J_{c}|}{\Delta t}\int_{0}^{x}(a-x)dx\\
\mu_{o}\frac{|\Delta J_{c}|}{\Delta t}(ax-\frac{x^{2}}{2})\qquad(S2.4)
\end{equation}
\begin{equation}%page339 第6个
\varepsilon_{\phi}=\int_{0}^{a}p(x)\Delta tdx\\
=\mu_{o}J_{c}|\Delta J_{c}|\int_{0}^{a}(ax-\frac{x^{2}}{2})dx=\frac{\mu_{o}J_{c}|\Delta J_{c}|a^{3}}{3}\quad(S2.5)
\end{equation}
\begin{equation}%page339 第7个
e_{\phi}=\frac{\mu_{o}J_{c}|\Delta J_{c}|a^{2}}{3}\qquad(5.37)
\end{equation}



\begin{equation}%page340 第1个
\int S_{x}(a)dt=\Delta E_{m}+\varepsilon_{\phi}\qquad(S2.6)
\end{equation}



\begin{equation}%page340 第2个
E_{z}(a)=\mu_{o}\frac{|\Delta J_{c}|a^{2}}{2\Delta t}\qquad(S2.7)
\end{equation}
\begin{equation}%page340 第3个
\vec{S}(a)=\mu_{o}\frac{|\Delta J_{c}|a^{2}}{2\Delta t}\vec{\imath}_{z}\times H_{e}\vec{\imath}_{y}=-\mu_{o}\frac{H_{e}|\Delta J_{c}|a^{2}}{2\Delta t}\quad(S2.8)
\end{equation}
\begin{equation}%page340 第4个
\int S_{x}(a)dt=\mu_{o}\frac{H_{e}|\Delta J_{c}|a^{2}}{2}\qquad(S2.9)
\end{equation}


\begin{align*}%page340 第5个
\Delta E_{m}&=\frac{\mu_{o}}{2}\int_{0}^{a}[H_{s2}^{a}(x)-H_{s1}^{2}(x)]dx\tag(S2.10)\\
&=\frac{\mu_{o}}{2}\int_{0}^{a}\{[H_{e}+(J_{c}-|\Delta J_{c}|)(x-a)]^{2}-[H_{e}+J_{c}(x-a)]^{2}\}dx\\\notag
&=\frac{\mu_{o}}{2}\int_{0}^{a}[-2H_{e}|\Delta J_{c}|(x-a)-2J_{C}|\Delta J_{c}|(x-a)^{2}+|\Delta J_{c}|^{2}(x-a)^{2}]dx\notag
\end{align*}



\begin{equation}%page340 第6个
\Delta E_{m}=\mu_{o}(\frac{H_{e}|\Delta J_{c}|a^{2}}{2}-\frac{J_{c}|\Delta J_{c}|a^{3}}{3})\quad(S2.11)
\end{equation}
\begin{equation}%page340 第7个
\varepsilon_{\phi}=\int S_{x}(a)dt-\Delta E_{m}\qquad(S2.12)
\end{equation}
\begin{equation}%page340 第8个
\varepsilon_{\phi}=\mu_{o}\frac{H_{e}|\Delta J_{c}|a^{2}}{2}-\mu_{o}(\frac{H_{e}|\Delta J_{c}|a^{2}}{2}-\frac{J_{c}|\Delta J_{c}|a^{3}}{3})\\
=\mu_{O}\frac{J_{c}|\Delta J_{c}|a^{3}}{3}\qquad(S2.13)
\end{equation}
\begin{equation}%page340 第9个
e_{\phi}=\frac{              \varepsilon_{\phi}}{a}=\frac{\mu_{o}J_{c}|\Delta J_{c}|a^{2}}{3}\qquad(5.37)
\end{equation}
\begin{equation}%page341 第1个
\Delta J_{c}=-J_{c_{c}}(\frac{\Delta T}{T_{c}-T_{op}})\qquad(5.39)
\end{equation}
\begin{equation}%page341 第2个
|\Delta J_{c}|=\frac{J_{c_{o}}\Delta T}{T_{c}-T_{op}}\qquad(S2.14)
\end{equation}
\begin{equation}%page341 第3个
e_{\phi}=\frac{\mu_{o}J_{c_{o}}^{2}\Delta T a^{2}}{3(T_{c}-T_{op})}\qquad(S2.15)
\end{equation}
\begin{equation}%page341 第4个
\Delta T_{s}=\frac{e_{\phi}}{\tilde{C}_{s}}>0\qquad(S2.16)
\end{equation}
\begin{equation}%page341 第5个
\frac{\Delta T_{s}}{\Delta T}<\frac{\mu_{o}J_{c_{o}}^{2}a^{2}}{3\tilde{C}_{s}(T_{c}-T_{op})}\qquad(S2.17)
\end{equation}
\begin{equation}%page341 第6个
a_{C}=\sqrt{\frac{3\tilde{C}_{s}(T_{c}-T_{op})}{\mu_{o}J_{c_{o}}^{2}}}\qquad(5.40)
\end{equation}

\subsection{问题5.3:磁通跳跃}

\begin{equation}%page342 第1个
e_{\phi}=\frac{(\mu_{o}J_{p})^{2}}{6\mu_{o}}\qquad(5.41)
\end{equation}

\subsubsection{问题5.3之解}

\begin{equation}%page343 第1个
\Delta H(x)=H_{p}\frac{(a-x)}{a}\qquad(S3.1)
\end{equation}
\begin{equation}%page343 第2个
E(x)=\mu_{o}\frac{H_{p}}{\Delta t}\int_{0}^{x}\frac{a-x}{a}dx=\frac{\mu_{o}H_{p}}{a\Delta t}(ax-\frac{x^{2}}{2})\quad(S3.2)
\end{equation}
\begin{equation}%page343 第3个
\vec{S}(a)=-\frac{\mu_{o}}{2\Delta t}H_{p}aH_{e}\vec{\imath}_{x}\qquad(S3.3)
\end{equation}
\begin{equation}%page343 第4个
e_{s}=\frac{\int S_{x}(a)dt}{a}=\frac{\mu_{o}}{2}H_{p}H_{e}\qquad(S3.4)
\end{equation}
\begin{equation}%page343 第5个
e_{m1}=\frac{\mu_{o}}{2a}\int_{0}^{a}[H_{e}+J_{c}(x-a)]^{2}dx\qquad(S3.5)
\end{equation}
\begin{equation}%page343 第6个
=\frac{\mu_{o}}{2a}(H_{e}^{2}a-H_{e}J_{c}a^{2}+\frac{J_{c}^{2}a^{3}}{3})=\frac{\mu_{o}}{2}H_{e}^{2}-\frac{\mu_{o}}{2}H_{e}H_{p}+\frac{\mu_{o}}{6}H_{p}^{2}(S3.6)
\end{equation}
\begin{equation}%page343 第7个
\Delta e_{m}=e_{m2}-e_{m1}=\frac{\mu_{o}}{2}H_{e}H_{p}-\frac{\mu_{o}}{6}H_{p}^{2}\qquad(S3.7)
\end{equation}
\begin{equation}%page343 第8个
e_{\phi}=\frac{\mu_{o}}{2}H_{p}H_{e}-\frac{\mu_{o}}{2}H_{e}H_{p}+\frac{\mu_{o}}{6}H_{p}^{2}=\frac{\mu_{o}}{6}H_{p}^{2}\quad(S3.8)
\end{equation}
\begin{equation}%page343 第9个
e_{\phi}=\frac{(\mu_{o}H_{p})^{2}}{6\mu_{o}}\qquad(5.41)
\end{equation}
\begin{equation}%page343 第10个
e_{\phi}\simeq\frac{(0.5T)^{2}}{(6)(4\pi\times10^{-7}\ \mathrm{H/m})}\simeq33\times10^{3}\ \mathrm{J/m^{3}}
\end{equation}


\subsection{问题5.4:导线换位}


\begin{figure}
	\centering
	\includegraphics[scale=0.6]{chpt5/figs/fig5.20.eps}
	\caption{Two-dimensional conductor consisting of a normal
		metal slab sandwiched between two Bean slabs.}
\end{figure}



\begin{equation}%page345 第1个
E_{1x}=\mu_{o}\dot{H}_{0z}y\qquad(5.42)
\end{equation}
\begin{equation}%page345 第2个
I_{cp}=\int_{0}^{\ell}J_{cu}dy=\frac{\mu_{o}\dot{H}_{0z}}{\rho_{cu}}\int_{0}^{\ell}ydy=\frac{\mu_{o}\dot{H}_{0z}\ell^{2}}{2\rho_{cu}}\quad(5.43)
\end{equation}
\begin{equation}%page345 第3个
\ell_{c}=\sqrt{\frac{2\rho_{cu}J_{c}d_{f}}{\mu_{o}\dot{H}_{0z}}}\qquad(5.44)
\end{equation}


\subsubsection{问题5.4之解}

\begin{equation}%page346 第1个
\frac{\partial E_{1y}}{\partial x}-\frac{\partial E_{1x}}{\partial y}=-\mu_{o}\dot{H}_{0z}\qquad(S4.1)
\end{equation}
\begin{equation}%page346 第2个
E_{1x}=\mu_{o}\dot{H}_{0z}y\qquad(5.42)
\end{equation}
\begin{equation}%page346 第3个
I_{cp}=\int_{0}^{\ell}J_{cu}dy=\frac{\mu_{o}\dot{H}_{0z}}{\rho_{cu}}\int_{0}^{\ell}ydy=\frac{\mu_{o}\dot{H}_{0z}\ell^{2}}{2\rho_{cu}}\quad(5.43)
\end{equation}
\begin{equation}%page346 第4个
\ell_{c}=\sqrt{\frac{2\rho_{cu}J_{c}d_{f}}{\mu_{o}\dot{H}_{0z}}}\qquad(5.44)
\end{equation}
\begin{equation}%page346 第5个
N_{f}=\frac{4I_{c}}{\pi d_{f}^{2}J_{c}}=\frac{4(100A)}{\pi(0.2\times10^{-6}m)^{2}(2\times10^{9}A/m^{2})}\\
=1.6\times10^{6}
\end{equation}


\subsection{问题5.5:导体磁化}




\begin{figure}
	\centering
	\includegraphics[scale=0.5]{chpt5/figs/fig5.21.eps}
	\caption{导体1,2,3的磁化迹线}
\end{figure}


\subsubsection{问题5.5之解}

\begin{equation}%page348 第1个
\ell_{p2}=2\sqrt{\frac{2\rho_{cu}J_{c}d_{f}}{\mu_{o}\dot{H}_{0z}}}
\end{equation}


\begin{equation}%page348 第2个
\ell_{p2}=2\sqrt{\frac{(2)(2\\times 10^{10}\ \mathrm{\Omega m})(4\times 10^{4}\ \mathrm{A/m})}{0.09\ \mathrm{T/S}}}\\
=2.7\times 10^{-2}\ \mathrm{m}=27\ \mathrm{mm}
\end{equation}


\subsection{讨论5.6:换位}




\subsection{讨论5.7:HTS中的磁通跳跃?}

\textbf{A. “完全”磁通跳跃的尺度判据}

\begin{equation}%page349 5.45
e_{\phi}(T_{op})\geq h_{s}(T_{c})-h_{s}(T_{op})
\end{equation}


\begin{equation}%page349 第2个
a_{c}=\sqrt{\frac{6[h_{s}(T_{c})-h_{s}(T_{op})]}{\mu_{o}J_{c}^{2}(T_{op})}}
\end{equation}

\textbf{B. HTS中的磁通跳跃}


