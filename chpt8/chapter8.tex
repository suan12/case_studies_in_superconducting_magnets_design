\chapter{保护}
\section{引言}
保护是五大关键设计和运行项目之一——其他四项是稳定性、机械完整性、制冷和导体。
\subsection{热能密度 vs. 磁能密度}
除非绕组得到了保护,不然磁体绕组的一小部分,即“热点”,就要吸收掉存储于绕组中的大部分磁能。这样,该部分将过热并永久性损坏。
不过,熔化磁体中单位绕组体积的热能密度要远大于磁体存储的磁能密度。

仅考虑将磁体内部空间内存储的能量全部绝热转换为热,引起铜(绕组的一种代表性材料)的焓密度$h_{Cu}(T)$变化。如果是从4K(或者80K)加热到它的熔点1356K,
那么初始磁感应密度$B_0$将高达$~150 T$:
\begin{eqnarray}\label{eqn: 8.1}
% \nonumber % Remove numbering (before each equation)
  \frac{B_0^2}{2\mu_0}&=& h_{Cu}(1356K)-h_{Cu}(4K/80K)\approx 5.2\times 10^9 J/m^3 \nonumber\\
  B_0 &\approx
   &\sqrt{2(4\pi \times 10^-7 H/m)(5.2\times 10^9 J/m^3)}\approx 115 T
\end{eqnarray}



\begin{equation}% 8.1第一个
\frac{B_{0}^{2}}{2\mu_o}=h_{cu}(1356\ \mathrm{K})-h_{cu}(4\ \mathrm{K}\ \mathrm{or}80\ \mathrm{K})\simeq 5.2\times 10^9\ \mathrm{J/m^3}
\end{equation}
\begin{equation}% 8.1第二个
B_0\simeq\sqrt{2(4\pi\times 10^{-7}\ \mathrm{H/m})(5.2\times 10^9\ \mathrm{J/m^3})}\simeq 115\ \mathrm{T}
\end{equation}
\begin{equation}% page468 3.79
E_m=\frac{1}{2}LI^2
\end{equation}
\begin{equation}% page468 3.81
L=\mu_oa_1\ \mathcal{L}(\alpha,\beta)N^2
\end{equation}
\begin{equation}% 8.2
B_o=\frac{\mu_oNI}{2a_1(\alpha-1)\beta}F(\alpha,\beta)
\end{equation}
\begin{equation}% page468 3.13b
F(\alpha,\beta)=\beta\left(\frac{\alpha+\sqrt{\alpha^2+\beta^2}}{1+\sqrt{1+\beta^2}}\right)
\end{equation}
\begin{equation}% 8.3
NI=\frac{2a_1(\alpha-1)\beta B_o}{\mu_oF(\alpha,\beta)}
\end{equation}
\begin{equation}% 8.4
E_m=\frac{4a_{1}^{3}(\alpha-1)^2\beta^2\ \mathcal{L}(\alpha,\beta)}{F^2(\alpha,\beta)}\left(\frac{B_{o}^{2}}{2\mu_o}\right)
\end{equation}
\begin{equation}% 8.5
V_w=2\pi a_{1}^{3}(\alpha^2-1)\beta
\end{equation}
\begin{equation}% 8.6
V_r=f_eV_w=f_r2\pi a_{1}^{3}(\alpha^2-1)\beta
\end{equation}
\begin{equation}% 8.7
e_{mr}=\frac{E_m}{V_r}=\frac{2(\alpha-1)\beta\ \mathcal{L}(\alpha,\beta)}{f_r\pi(\alpha+1)F^2(\alpha,\beta)}\left(\frac{B_{o}^{2}}{2\mu_o}\right)
\end{equation}
\begin{equation}% 8.8a和8.8b
C_{cd}(T)\frac{dT}{dt}\simeq\rho_{cd}(T)J_{cd_o}^{2}(t) 
\simeq\rho_m(T)J_{cd_o}^{2}(t)
\end{equation}
\begin{equation}% 8.9a
A_{cd}C_{cd}(T)\frac{dT}{dt}=\frac{\rho_m(T)}{A_m}I_{op}^{2}(t)
\end{equation}
\begin{equation}% 8.9b
C_m(T)\frac{dT}{dt}=\left(\frac{A_m}{A_{cd}}\right)\rho_m(T)J_{m_o}^{2}=\left(\frac{\gamma_{m/s}}{1+\gamma_{m/s}}\right)\rho_m(T)J_{m_o}^{2}
\end{equation}
\begin{equation}% 8.9c和8.9d
\int_{T_i}^{T_f}\frac{C_m(T)}{\rho_m(T)}dT=\left(\frac{A_m}{A_{cd}}\right)J_{m_o}^{2}\tau_{ah} 
=\left(\frac{\gamma_{m/s}}{1+\gamma_{m/s}}\right)J_{m_o}^{2}\tau_{ah}
\end{equation}
\begin{equation}% 8.10a
Z(T_f,T_i)\equiv\int_{T_i}^{T_f}\frac{C_m(T)}{\rho_m(T)}dT
\end{equation}
\begin{equation}% 8.10b
Z(T_f,T_i)\simeq\frac{1}{\tilde{\rho}_m}\int_{T_i}^{T_f}C_m(T_f)dT=\frac{H_m(T_f)-H_m(T_i)}{\tilde{\rho}_m}
\end{equation}
\begin{equation}% 8.11
Z(T_f,T_i)=Z(T_f,0)-Z(T_i,0)=Z(T_f)-Z(T_i)
\end{equation}
\begin{equation}% 8.12a
\tau_{ah}^{i}(T_f,T_i)=\left(\frac{1+\gamma_{m/s}}{\gamma_{m/s}}\right)\frac{Z(T_f,T_i)}{J_{m_o}^{2}}
\end{equation}
\begin{equation}% 8.12b
J_{m_o}^{i}(T_f,T_i)=\sqrt{\left(\frac{1+\gamma_{m/s}}{\gamma_{m/s}}\right)\frac{Z(T_f,T_i)}{\tau_{ah}}}
\end{equation}
\begin{equation}% 8.13
C_m(T)\frac{dT}{dt}=\left(\frac{A_m}{A_{cd}}\right)\rho_m(T)J_{m}^{2}(t)
\end{equation}
\begin{equation}% 8.14
L\frac{dI_m(t)}{dt}+[R_D+r(t)]I_m(t)=0
\end{equation}
\begin{equation}% 8.15
J_m(t)=J_{m_o}e^{-t/\tau_{dg}}
\end{equation}
\begin{equation}% 8.16a 8.16b 8.16c
Z(T_f,T_i)=\left(\frac{A_m}{A_{cd}}\right)\int_{0}^{\infty}J_{m_o}^{2}e^{-2t/\tau_{dg}}dt=\left(\frac{A_m}{A_{cd}}\right)J_{m_o}^{2}\times\frac{1}{2}\tau_{dg} 
=\left(\frac{A_m}{A_{cd}}\right)J_{m_o}^{2}\left(\frac{L}{2R_D}\right) 
=\left(\frac{\gamma_{m/s}}{1+\gamma_{m/s}}\right)J_{m_o}^{2}\left(\frac{L}{2R_D}\right)
\end{equation}
\begin{equation}% 8.17a
L=\frac{2E_m}{I_{op}^{2}}
\end{equation}
\begin{equation}% 8.17b
R_D=\frac{V_D}{I_{op}}
\end{equation}
\begin{equation}% 8.18a
Z(T_f,T_i)=\left(\frac{A_m}{A_{cd}}\right)\frac{J_{m_o}^{2}E_m}{V_DI_{op}}
\end{equation}
\begin{equation}% 8.18b
Z(T_f,T_i)=\frac{J_{m_o}^{2}E_m}{A_{cd}V_D}
\end{equation}
\begin{equation}% 8.19
J_{m_o}^{D}=\frac{A_{cd}V_DZ(T_f,T_i)}{E_m}
\end{equation}
\begin{equation}% 8.20
L\frac{dI_m(t)}{dt}+r(T)I_m(t)=0
\end{equation}
\begin{equation}% 8.21
R_{nz}=\frac{\rho_m(T_f)\ell_{nz}}{4A_m}
\end{equation}
\begin{equation}% 8.22
J_m(t)=J_{m_o}e^{-t/(L/R_{nz})}
\end{equation}
\begin{equation}% 8.23a
Z(T_f,T_i)=\left(\frac{A_m}{A_{cd}}\right)\int_{0}^{\infty}J_{m_o}^{2}e^{-2t/(L/R_{nz})}dt
\end{equation}
\begin{equation}% 8.23b 8.23c
Z(T_f,T_i)=\frac{1}{2}\left(\frac{A_m}{A_{cd}}\right)J_{m_o}^{2}\left(\frac{L}{R_{nz}}\right) 
=\frac{1}{2}\left(\frac{A_m}{A_{cd}}\right)J_{m_o}^{2}\tau_{dg}
\end{equation}
\begin{equation}% 8.24
\ell_{nz}=f_r\pi a_1(\alpha+1)N
\end{equation}
\begin{equation}% 8.25
R_{nz}=f_r\frac{\rho_m(T_f)\pi a_1(\alpha+1)N}{4A_m}
\end{equation}
\begin{equation}% 8.26
N=\sqrt{\frac{L}{\mu_oa_1\ \mathcal{L}(\alpha,\beta)}}
\end{equation}
\begin{equation}% 8.27
R_{nz}=f_r\frac{\pi(\alpha+1)\rho_m(T_f)}{4A_m}\sqrt{\frac{a_1L}{\mu_o\ \mathcal{L}(\alpha,\beta)}}
\end{equation}
\begin{equation}% 8.28a
\tau_{dg}=\frac{L}{R_{nz}}=\frac{4A_m}{f_r\pi(\alpha+1)\rho_m(T_f)}\sqrt{\frac{\mu_o\ \mathcal{L}(\alpha,\beta)L}{a_1}}
\end{equation}
\begin{equation}% 8.28b
\tau_{dg}=\frac{4}{f_r\pi(\alpha+1)\rho_m(T_f)J_{m_o}}\sqrt{\frac{2\mu_o\ \mathcal{L}(\alpha,\beta)E_m}{a_1}}
\end{equation}
\begin{equation}% 8.29
\rho_m(T_f)Z(T_f,T_i)=\left(\frac{A_m}{A_{cd}}\right)\frac{2J_{m_o}}{f_r\pi(\alpha+1)}\sqrt{\frac{2\mu_o\ \mathcal{L}(\alpha,\beta)E_m}{a_1}}
\end{equation}
\begin{equation}% 8.30a
J_{m_o}^{sh}=\frac{1}{2}\left(\frac{A_{cd}}{A_m}\right)f_r\pi(\alpha+1)\rho_m(T_f)Z(T_f,T_i)\sqrt{\frac{a_1}{2\mu_o\ \mathcal{L}(\alpha,\beta)E_m}}
\end{equation}
\begin{equation}% 8.30b
J_{m_o}^{sh}(T_f,T_i)=\frac{1}{2}\left(\frac{A_{cd}}{A_m}\right)\frac{f_r\pi(\alpha+1)\rho_m(T_f)Z(T_f,T_i)}{\mu_o\ \mathcal{L}(\alpha,\beta)NI_{op}}
\end{equation}
\begin{equation}% 8.31a两个
A_{cd}\ell_{cd}C_{cd}(T)\frac{dT}{dt}=\frac{V_{op}^{2}}{R_n(T)}=\frac{V_{op}^{2}A_m}{\rho_m(T)\ell_{cd}}
C_m(T)\frac{dT}{dt}\simeq\left(\frac{A_m}{A_{cd}}\right)\frac{V_{op}^{2}}{\rho_m(T)\ell_{cd}^{2}}
\end{equation}
\begin{equation}% 8.31b
\int_{T_i}^{T_f}C_m(T)\rho_m(T)dT=\left(\frac{A_m}{A_{cd}}\right)\frac{V_{op}^{2}}{\ell_{cd}^{2}}\tau_{ah}
\end{equation}
\begin{equation}% 8.32a
Y(T_f,T_i)\equiv\int_{T_i}^{T_f}C_m(T)\rho_m(T)dT
\end{equation}
\begin{equation}% 8.32b
Y(T_f,T_i)\simeq\tilde{\rho}_m[H_m(T_f)-H_m(T_i)]
\end{equation}
\begin{equation}% 8.32c
Y(T_f,T_i)=Y(T_f)-Y(T_i)
\end{equation}
\begin{equation}% 8.33
Y(T_f,T_i)=\left(\frac{A_m}{A_{cd}}\right)\frac{V_{op}^{2}\tau_{ah}}{\ell_{cd}^{2}}
\end{equation}
\begin{equation}% 8.34
\ell_{cd}=\pi(\alpha+1)\sqrt{\frac{a_1L}{\mu_o\ \mathcal{L}(\alpha,\beta)}}
\end{equation}
\begin{equation}% 8.35
Y(T_f,T_i)=\left(\frac{A_m}{A_{cd}}\right)\frac{\mu_o\ \mathcal{L}(\alpha,\beta)V_{op}^{2}\tau_{ah}}{\pi^2(\alpha+1)^2a_1L}
\end{equation}


\subsection{热点和热点温度}

\subsection{绕组材料的温度数据}

\subsection{$T_f$的安全、风险、高度风险区间}

\subsection{温度引起的应变}


\section{绝热加热}
\subsection{恒定电流模式下的绝热加热}
\subsection{恒定放电量模式下的绝热加热}
\subsection{引线短接的磁体的绝热加热}
\subsection{恒定电压模式下的绝热加热}

\section{高电压}
\subsection{电弧环境}
\subsection{Paschen电压试验}
\subsection{失超磁体内的电压峰值}

\section{正常区传播(NZP)}

\subsection{轴向NZP速度}
\subsection{“制冷”条件下的NZP}
\subsection{横向匝间速度}
\subsection{热-流体失超恢复(THQB)}
\subsection{交流损耗诱导的NZP}

\section{计算机仿真}

\section{自保护磁体}
\subsection{尺度限制}

\section{孤立磁体的被动保护}


\section{主动保护}
\subsection{过热}
\subsection{多线圈磁体中的过压}
\subsection{主动保护技术:检测-抑制}
\subsection{主动保护技术:检测-激活加热器}
\subsection{失超电压保护技术:基本电桥}

\section{专题}
\subsection{问题1:大型超导磁体的回温}

\newpage
\subsection{问题2:6 kA气冷HTS引线的保护}

\newpage
\subsection{问题3:制冷机制冷的NbTi磁体的保护}

\newpage
\subsection{问题4:混合III SCM的“热点”温度}

\newpage
\subsection{讨论5:失超电压探测——一个变种}

\newpage
\subsection{问题6:抑制电阻的设计}

\newpage
\subsection{讨论7:磁体的“缓慢”放电模式}

\newpage
\subsection{讨论8:低阻电阻器设计}

\newpage
\subsection{讨论9:过热\& 内部电压判据}

\newpage
\subsection{讨论10:Bi2223带电流引线的保护}

\newpage
\subsection{讨论11:$MgB_2$磁体的主动保护}

\newpage
\subsection{问题12:NMR磁体的被动保护}

\newpage
\subsection{讨论13:HTS磁体到底要不要保护?}
