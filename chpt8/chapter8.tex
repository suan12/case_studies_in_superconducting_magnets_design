\chapter{保护}
\section{引言}
保护是五大关键设计和运行项目之一——其他四项是稳定性、机械完整性、制冷和导体。




\subsection{热能密度 vs. 磁能密度}
除非绕组得到了保护,不然磁体绕组的一小部分,即“热点”,就要吸收掉存储于绕组中的大部分磁能。这样,该部分将过热并永久性损坏。
不过,熔化磁体中单位绕组体积的热能密度要远大于磁体存储的磁能密度。

仅考虑将磁体内部空间内存储的能量全部绝热转换为热,引起铜(绕组的一种代表性材料)的焓密度$h_{Cu}(T)$变化。如果是从4K(或者80K)加热到它的熔点1356K,
那么初始磁感应密度$B_0$将高达$~150 T$:
\begin{equation}% 8.1第一个
\frac{B_{0}^{2}}{2\mu_o}=h_{cu}(1356\ \mathrm{K})-h_{cu}(4\ \mathrm{K}\ \mathrm{or}80\ \mathrm{K})\simeq 5.2\times 10^9\ \mathrm{J/m^3}
\end{equation}
\begin{equation}% 8.1第二个
B_0\simeq\sqrt{2(4\pi\times 10^{-7}\ \mathrm{H/m})(5.2\times 10^9\ \mathrm{J/m^3})}\simeq 115\ \mathrm{T}
\end{equation}

\subsection{热点和热点温度}

\begin{equation}% page468 3.79
E_m=\frac{1}{2}LI^2
\end{equation}
\begin{equation}% page468 3.81
L=\mu_oa_1\ \mathcal{L}(\alpha,\beta)N^2
\end{equation}
\begin{equation}% 8.2
B_o=\frac{\mu_oNI}{2a_1(\alpha-1)\beta}F(\alpha,\beta)
\end{equation}
\begin{equation}% page468 3.13b
F(\alpha,\beta)=\beta\left(\frac{\alpha+\sqrt{\alpha^2+\beta^2}}{1+\sqrt{1+\beta^2}}\right)
\end{equation}
\begin{equation}% 8.3
NI=\frac{2a_1(\alpha-1)\beta B_o}{\mu_oF(\alpha,\beta)}
\end{equation}
\begin{equation}% 8.4
E_m=\frac{4a_{1}^{3}(\alpha-1)^2\beta^2\ \mathcal{L}(\alpha,\beta)}{F^2(\alpha,\beta)}\left(\frac{B_{o}^{2}}{2\mu_o}\right)
\end{equation}
\begin{equation}% 8.5
V_w=2\pi a_{1}^{3}(\alpha^2-1)\beta
\end{equation}
\begin{equation}% 8.6
V_r=f_eV_w=f_r2\pi a_{1}^{3}(\alpha^2-1)\beta
\end{equation}
\begin{equation}% 8.7
e_{mr}=\frac{E_m}{V_r}=\frac{2(\alpha-1)\beta\ \mathcal{L}(\alpha,\beta)}{f_r\pi(\alpha+1)F^2(\alpha,\beta)}\left(\frac{B_{o}^{2}}{2\mu_o}\right)
\end{equation}


\subsection{绕组材料的温度数据}



\subsection{安全、风险和高度风险$T_f$区间}


\subsection{温度引起的应变}



\section{绝热加热}

\begin{equation}% 8.8a和8.8b
C_{cd}(T)\frac{dT}{dt}\simeq\rho_{cd}(T)J_{cd_o}^{2}(t) 
\simeq\rho_m(T)J_{cd_o}^{2}(t)
\end{equation}

\subsection{恒定电流模式下的绝热加热}
\begin{equation}% 8.9a
A_{cd}C_{cd}(T)\frac{dT}{dt}=\frac{\rho_m(T)}{A_m}I_{op}^{2}(t)
\end{equation}
\begin{equation}% 8.9b
C_m(T)\frac{dT}{dt}=\left(\frac{A_m}{A_{cd}}\right)\rho_m(T)J_{m_o}^{2}=\left(\frac{\gamma_{m/s}}{1+\gamma_{m/s}}\right)\rho_m(T)J_{m_o}^{2}
\end{equation}
\begin{equation}% 8.9c和8.9d
\int_{T_i}^{T_f}\frac{C_m(T)}{\rho_m(T)}dT=\left(\frac{A_m}{A_{cd}}\right)J_{m_o}^{2}\tau_{ah} 
=\left(\frac{\gamma_{m/s}}{1+\gamma_{m/s}}\right)J_{m_o}^{2}\tau_{ah}
\end{equation}
\begin{equation}% 8.10a
Z(T_f,T_i)\equiv\int_{T_i}^{T_f}\frac{C_m(T)}{\rho_m(T)}dT
\end{equation}
\begin{equation}% 8.10b
Z(T_f,T_i)\simeq\frac{1}{\tilde{\rho}_m}\int_{T_i}^{T_f}C_m(T_f)dT=\frac{H_m(T_f)-H_m(T_i)}{\tilde{\rho}_m}
\end{equation}
\begin{equation}% 8.11
Z(T_f,T_i)=Z(T_f,0)-Z(T_i,0)=Z(T_f)-Z(T_i)
\end{equation}
\begin{equation}% 8.12a
\tau_{ah}^{i}(T_f,T_i)=\left(\frac{1+\gamma_{m/s}}{\gamma_{m/s}}\right)\frac{Z(T_f,T_i)}{J_{m_o}^{2}}
\end{equation}
\begin{equation}% 8.12b
J_{m_o}^{i}(T_f,T_i)=\sqrt{\left(\frac{1+\gamma_{m/s}}{\gamma_{m/s}}\right)\frac{Z(T_f,T_i)}{\tau_{ah}}}
\end{equation}

\subsection{恒定放电模式下的绝热加热}

\begin{equation}% 8.13
C_m(T)\frac{dT}{dt}=\left(\frac{A_m}{A_{cd}}\right)\rho_m(T)J_{m}^{2}(t)
\end{equation}
\begin{equation}% 8.14
L\frac{dI_m(t)}{dt}+[R_D+r(t)]I_m(t)=0
\end{equation}
\begin{equation}% 8.15
J_m(t)=J_{m_o}e^{-t/\tau_{dg}}
\end{equation}
\begin{equation}% 8.16a 8.16b 8.16c
Z(T_f,T_i)=\left(\frac{A_m}{A_{cd}}\right)\int_{0}^{\infty}J_{m_o}^{2}e^{-2t/\tau_{dg}}dt=\left(\frac{A_m}{A_{cd}}\right)J_{m_o}^{2}\times\frac{1}{2}\tau_{dg} 
=\left(\frac{A_m}{A_{cd}}\right)J_{m_o}^{2}\left(\frac{L}{2R_D}\right) 
=\left(\frac{\gamma_{m/s}}{1+\gamma_{m/s}}\right)J_{m_o}^{2}\left(\frac{L}{2R_D}\right)
\end{equation}
\begin{equation}% 8.17a
L=\frac{2E_m}{I_{op}^{2}}
\end{equation}
\begin{equation}% 8.17b
R_D=\frac{V_D}{I_{op}}
\end{equation}
\begin{equation}% 8.18a
Z(T_f,T_i)=\left(\frac{A_m}{A_{cd}}\right)\frac{J_{m_o}^{2}E_m}{V_DI_{op}}
\end{equation}
\begin{equation}% 8.18b
Z(T_f,T_i)=\frac{J_{m_o}^{2}E_m}{A_{cd}V_D}
\end{equation}
\begin{equation}% 8.19
J_{m_o}^{D}=\frac{A_{cd}V_DZ(T_f,T_i)}{E_m}
\end{equation}

\subsection{引线短接的磁体的绝热加热}
\begin{equation}% 8.20
L\frac{dI_m(t)}{dt}+r(T)I_m(t)=0
\end{equation}
\begin{equation}% 8.21
R_{nz}=\frac{\rho_m(T_f)\ell_{nz}}{4A_m}
\end{equation}
\begin{equation}% 8.22
J_m(t)=J_{m_o}e^{-t/(L/R_{nz})}
\end{equation}
\begin{equation}% 8.23a
Z(T_f,T_i)=\left(\frac{A_m}{A_{cd}}\right)\int_{0}^{\infty}J_{m_o}^{2}e^{-2t/(L/R_{nz})}dt
\end{equation}
\begin{equation}% 8.23b 8.23c
Z(T_f,T_i)=\frac{1}{2}\left(\frac{A_m}{A_{cd}}\right)J_{m_o}^{2}\left(\frac{L}{R_{nz}}\right) 
=\frac{1}{2}\left(\frac{A_m}{A_{cd}}\right)J_{m_o}^{2}\tau_{dg}
\end{equation}
\begin{equation}% 8.24
\ell_{nz}=f_r\pi a_1(\alpha+1)N
\end{equation}
\begin{equation}% 8.25
R_{nz}=f_r\frac{\rho_m(T_f)\pi a_1(\alpha+1)N}{4A_m}
\end{equation}
\begin{equation}% 8.26
N=\sqrt{\frac{L}{\mu_oa_1\ \mathcal{L}(\alpha,\beta)}}
\end{equation}
\begin{equation}% 8.27
R_{nz}=f_r\frac{\pi(\alpha+1)\rho_m(T_f)}{4A_m}\sqrt{\frac{a_1L}{\mu_o\ \mathcal{L}(\alpha,\beta)}}
\end{equation}
\begin{equation}% 8.28a
\tau_{dg}=\frac{L}{R_{nz}}=\frac{4A_m}{f_r\pi(\alpha+1)\rho_m(T_f)}\sqrt{\frac{\mu_o\ \mathcal{L}(\alpha,\beta)L}{a_1}}
\end{equation}
\begin{equation}% 8.28b
\tau_{dg}=\frac{4}{f_r\pi(\alpha+1)\rho_m(T_f)J_{m_o}}\sqrt{\frac{2\mu_o\ \mathcal{L}(\alpha,\beta)E_m}{a_1}}
\end{equation}
\begin{equation}% 8.29
\rho_m(T_f)Z(T_f,T_i)=\left(\frac{A_m}{A_{cd}}\right)\frac{2J_{m_o}}{f_r\pi(\alpha+1)}\sqrt{\frac{2\mu_o\ \mathcal{L}(\alpha,\beta)E_m}{a_1}}
\end{equation}
\begin{equation}% 8.30a
J_{m_o}^{sh}=\frac{1}{2}\left(\frac{A_{cd}}{A_m}\right)f_r\pi(\alpha+1)\rho_m(T_f)Z(T_f,T_i)\sqrt{\frac{a_1}{2\mu_o\ \mathcal{L}(\alpha,\beta)E_m}}
\end{equation}
\begin{equation}% 8.30b
J_{m_o}^{sh}(T_f,T_i)=\frac{1}{2}\left(\frac{A_{cd}}{A_m}\right)\frac{f_r\pi(\alpha+1)\rho_m(T_f)Z(T_f,T_i)}{\mu_o\ \mathcal{L}(\alpha,\beta)NI_{op}}
\end{equation}


\subsection{恒定电压模式下的绝热加热}

\begin{equation}% 8.31a两个
A_{cd}\ell_{cd}C_{cd}(T)\frac{dT}{dt}=\frac{V_{op}^{2}}{R_n(T)}=\frac{V_{op}^{2}A_m}{\rho_m(T)\ell_{cd}}
C_m(T)\frac{dT}{dt}\simeq\left(\frac{A_m}{A_{cd}}\right)\frac{V_{op}^{2}}{\rho_m(T)\ell_{cd}^{2}}
\end{equation}
\begin{equation}% 8.31b
\int_{T_i}^{T_f}C_m(T)\rho_m(T)dT=\left(\frac{A_m}{A_{cd}}\right)\frac{V_{op}^{2}}{\ell_{cd}^{2}}\tau_{ah}
\end{equation}
\begin{equation}% 8.32a
Y(T_f,T_i)\equiv\int_{T_i}^{T_f}C_m(T)\rho_m(T)dT
\end{equation}
\begin{equation}% 8.32b
Y(T_f,T_i)\simeq\tilde{\rho}_m[H_m(T_f)-H_m(T_i)]
\end{equation}
\begin{equation}% 8.32c
Y(T_f,T_i)=Y(T_f)-Y(T_i)
\end{equation}
\begin{equation}% 8.33
Y(T_f,T_i)=\left(\frac{A_m}{A_{cd}}\right)\frac{V_{op}^{2}\tau_{ah}}{\ell_{cd}^{2}}
\end{equation}
\begin{equation}% 8.34
\ell_{cd}=\pi(\alpha+1)\sqrt{\frac{a_1L}{\mu_o\ \mathcal{L}(\alpha,\beta)}}
\end{equation}
\begin{equation}% 8.35
Y(T_f,T_i)=\left(\frac{A_m}{A_{cd}}\right)\frac{\mu_o\ \mathcal{L}(\alpha,\beta)V_{op}^{2}\tau_{ah}}{\pi^2(\alpha+1)^2a_1L}
\end{equation}

\begin{equation}% 8.36
\tau_{ah}^{\upsilon}=\left(\frac{A_{cd}}{A_m}\right)\frac{\pi^2(\alpha+1)^2a_1L}{\mu_o\ \mathcal{L}(\alpha,\beta)}\left[\frac{Y(T_f,T_i)}{V_{op}^{2}}\right]
\end{equation}


\section{高电压}

\begin{equation}% 8.37a
V=L\frac{dI}{dt}
\end{equation}
\begin{equation}% 8.37b
V=\frac{2E_m}{I_{o}^{2}}\left(\frac{\Delta I}{\Delta t}\right)\approx\frac{2E_m}{I_o\Delta t}
\end{equation}

\subsection{电弧环境}

\subsection{Paschen电压试验}

\subsection{失超磁体内的电压峰值}


\begin{equation}% 8.38
[V_{in}]_{mx}=f_r(1-f_r)R_{nz}I_{op}
\end{equation}
\begin{equation}% 8.39
[V_{in}]_{mx}=f_r(1-f_r)\frac{\pi(\alpha+1)\rho_m(T_f)a_1}{4A_m}NI_{op}
\end{equation}
\begin{equation}% 8.40a
J_{m_o}^{V}=\frac{2}{f_r(1-f_r)}\left[\frac{F(\alpha,\beta)}{\pi(\alpha^2-1)\beta}\right]\left[\frac{\mu_oV_{bk}I_{op}}{a_{1}^{2}\rho_m(T_f)B_o}\right]
\end{equation}
\begin{equation}% 8.40b
J_{m_o}^{V}=\frac{2}{f_r(1-f_r)}\left[\frac{\sqrt{\ \mathcal{L}(\alpha,\beta)}}{\pi(\alpha+1)}\right]\left[\frac{V_{bk}I_{op}}{\rho_m(T_f)}\sqrt{\frac{2\mu_o}{a_1E_m}}\right]
\end{equation}

\section{正常区传播(NZP)}

\subsection{轴向NZP速度}

\begin{equation}% 8.41a
C_n(T)\frac{\partial T_n}{\partial t}=\frac{\partial}{\partial x}\left[k_n(T)\frac{\partial T_n}{\partial x}\right]+\rho_n(T)J^2
\end{equation}
\begin{equation}% 8.41b
C_s(T)\frac{\partial T_s}{\partial t}=\frac{\partial}{\partial x}\left[k_s(T)\frac{\partial T_s}{\partial x}\right]
\end{equation}
\begin{equation}% 8.42
\frac{\partial T_n}{\partial t}=\frac{\partial T}{\partial z}\frac{\partial z}{\partial t}=-U_\ell\frac{dT}{dz}
\end{equation}
\begin{equation}% 8.43a
-C_n(T)U_\ell\frac{dT_n}{dz}=\frac{d}{dz}\left[k_n(T)\frac{dT_n}{dz}\right]+\rho_n(T)J^2
\end{equation}
\begin{equation}% 8.43b
-C_s(T)U_\ell\frac{dT_s}{dz}=\frac{d}{dz}\left[k_s(T)\frac{dT_s}{dz}\right]
\end{equation}
\begin{equation}% 8.44a
(z<0)      \frac{d}{dz}\left[k_n(T)\frac{dT_n}{dz}\right]+C_n(T)U_\ell\frac{dT_n}{dz}+\rho_n(T)J^2=0
\end{equation}
\begin{equation}% 8.44b
(z>0)      \frac{d}{dz}\left[k_s(T)\frac{dT_s}{dz}\right]+C_s(T)U_\ell\frac{dT_s}{dz}=0
\end{equation}
\begin{equation}% 8.45a
(z<0)      C_nU_\ell\frac{dT_n}{dz}+\rho_nJ^2=0
\end{equation}
\begin{equation}% 8.45b
(z>0)     k_s\frac{d^2T_s}{dz^2}+C_sU_\ell\frac{dT_s}{dz}=0
\end{equation}
\begin{equation}% 8.46a
T_s(z)=Ae^{-cz}+T_{op}
\end{equation}
\begin{equation}% 8.46b
T_s(z)=(T_t-T_{op})\exp\left(-\frac{C_sU_\ell}{k_s}z\right)+T_{op}
\end{equation}
\begin{equation}% 8.47a
k_n\frac{dT_n}{dz}\mid_0=k_s\frac{dT_s}{dz}\mid_0
\end{equation}
\begin{equation}% 8.47b
-\frac{k_n\rho_nJ^2}{C_nU_\ell}=-C_sU_\ell(T_t-T_{op})
\end{equation}
\begin{equation}% 8.48
U_\ell=J\sqrt{\frac{\rho_nk_n}{C_nC_s(T_t-T_{op})}}
\end{equation}
\begin{equation}% 8.49
U_\ell=J\sqrt{\frac{\rho_n(T_t)k_n(T_t)}{\left[C_n(T_t)-\frac{1}{k_n(T_t)}\frac{dk_n}{dT}\mid_{T_t}\int_{T_{op}}^{T_t}C_s(T)dT\right]\int_{T_{op}}^{T_t}C_s(T)dT}}
\end{equation}
\begin{equation}% 8.50a
U_\ell=\frac{J}{C_o}\sqrt{\frac{\rho_nk_n}{(T_t-T_{op})}}
\end{equation}
\begin{equation}% 8.50b
U_\ell=\frac{J}{C_o(\tilde{T})}\sqrt{\frac{\rho_n(\tilde{T})k_n(\tilde{T})}{(T_t-T_{op})}}
\end{equation}
\begin{equation}% 8.51a
U_\ell=\frac{J_m}{C_{cd}}(\tilde{T})\sqrt{\frac{\rho_m(\tilde{T})k_m(\tilde{T})}{T_t-T_{op}}}
\end{equation}
\begin{equation}% 8.51b
U_\ell\simeq\frac{J_m}{C_m}(\tilde{T})\sqrt{\frac{\rho_m(\tilde{T})k_m(\tilde{T})}{T_t-T_{op}}}
\end{equation}



\subsection{“制冷”条件下的NZP}

\begin{equation}% 8.52
U_t=U_\ell\sqrt{\frac{1}{2}\left(\frac{\delta_{cd}}{\delta_i}\right)\left[\frac{k_i(\tilde{T})}{k_m(\tilde{T})}\right]}
\end{equation}
\begin{equation}% 8.53
\frac{1}{k_{i}^{\prime}}=\frac{1}{k_i}+R_{th_{ct}^{1}}+R_{th_{ct}^{2}}
\end{equation}
\begin{equation}% 8.54
U_t=U_\ell\sqrt{\frac{1}{2}\left(\frac{\delta_{cd}}{\delta_i}\right)\frac{k_i}{k_m[1+k_i(R_{th_{ct}^{1}}+R_{th_{ct}^{1}})]}}
\end{equation}
\begin{equation}% 8.55
U_t=U_\ell\sqrt{\frac{1}{2}\left(\frac{\delta_{cd}}{\delta_i}\right)\frac{1}{k_m(R_{th_{ct}^{1}}+R_{th_{ct}^{1}})}}
\end{equation}


\subsection{横向匝间速度}

\subsection{热-流体失超恢复(THQB)}

\subsection{交流损耗诱导的NZP}

\section{计算机仿真}
\begin{equation}% 8.56
\frac{d_{cd}}{U_t}\ll\frac{2\pi a_1}{U_\ell}
\end{equation}


\section{自保护磁体}

\subsection{尺度限制}

\begin{equation}% 8.57
\frac{a_1(\alpha-1)}{U_t}<\tau_{dg}
\end{equation}
\begin{equation}% 8.58
\frac{a_1(\alpha-1)}{U_t}=\frac{Z(T_f,T_i)}{J_{m_o}^{2}}
\end{equation}
\begin{equation}% 8.59
[a_1(\alpha-1)]_{ah}^{i}=\frac{Z(T_f,T_i)}{J_{m_o}C_m(\tilde{T})}\sqrt{\frac{\rho_m(\tilde{T})k_i(\tilde{T})\delta_{cd}}{2\delta_i(T_t-T_{op})}}
\end{equation}
\begin{equation}% 8.60a 8.60b 8.60c
[a_1(\alpha-1)]_{ah}^{sh}=U_t\left(\frac{L}{R_{nz}}\right) 
=\frac{J_{m_o}}{C_n(\tilde{T})}\sqrt{\frac{\rho_m(\tilde{T})k_i(\tilde{T})\delta_{cd}}{2\delta_i(T_t-T_{op})}}\left(\frac{L}{R_{nz}}\right)  
=\frac{J_{m_o}}{C_n(\tilde{T})}\sqrt{\frac{\rho_m(\tilde{T})k_i(\tilde{T})\delta_{cd}}{2\delta_i(T_t-T_{op})}}\times 
\frac{4A_m}{f_r\pi(\alpha+1)\rho_m(T_f)}\sqrt{\frac{\mu_o\mathcal{L}(\alpha,\beta)L}{a_1}}
\end{equation}
\begin{equation}% 8.60d
[a_1(\alpha-1)]_{ah}^{sh}=\frac{1}{C_m(\tilde{T})}\sqrt{\frac{\rho_m(\tilde{T})k_i(\tilde{T})\delta_{cd}}{2\delta_i(T_t-T_{op})}}\times 
\frac{4}{f_r\pi(\alpha+1)\rho_m(T_f)}\sqrt{\frac{2\mu_o\mathcal{L}(\alpha,\beta)E_m}{a_1}}
\end{equation}


\section{孤立磁体的被动保护}

\begin{equation}% 8.61
\frac{E_r}{E_m}=\frac{0.5\zeta(1-k)+(1+k)}{\zeta+(1+k)}
\end{equation}
\begin{equation}% 8.62a
\frac{I_1(t)}{I_0}=\frac{R(1+k)^2}{2r}\exp\left(-\frac{Rt}{2L}\right)\left[1-\frac{R(1+k)^2}{2r}\right]\exp\left[-\frac{rt}{(1-k^2)L}\right]
\end{equation}
\begin{equation}% 8.62b
\frac{I_2(t)}{I_0}=(1+k)\exp\left(-\frac{Rt}{2L}\right)-k\exp\left[-\frac{rt}{(1-k^2)L}\right]
\end{equation}
\begin{equation}% 8.63
E_d=E_m+E_s-E_{R1}-E_{R2}
\end{equation}
\begin{equation}% page500第一个
E_s=(100\ \mathrm{A})\int_{0}^{0.4\ \mathrm{s}}[V_1(t)+V_2(t)]dt+(10\ \mathrm{V})\int_{0.4\ \mathrm{s}}^{2\ \mathrm{s}}\left[I_1(t)+\frac{V_1(t)}{R_1}\right]dt 
\simeq 200\ \mathrm{J}+650\ \mathrm{J}\simeq 850\ \mathrm{J}
\end{equation}
\begin{equation}% page500 第二个和第三个
E_{R1}=\frac{1}{R_1}\int_{0}^{2\ \mathrm{s}}V_1(t)^2dt\simeq 50\ \mathrm{J} 
E_{R2}=\frac{1}{R_2}\int_{0}^{2\ \mathrm{s}}V_2(t)^2dt\simeq 300\ \mathrm{J}
\end{equation}

\section{主动保护}
\subsection{过热}


\subsection{多线圈磁体中的过压}


\subsection{主动保护技术:检测-抑制}


\subsection{主动保护技术:检测-激活加热器}


\subsection{失超电压保护技术:基本电桥}

\section{专题}
\subsection{问题1:大型超导磁体的回温}

\newpage
\subsection{问题2:6 kA气冷HTS引线的保护}

\newpage
\subsection{问题3:制冷机制冷的NbTi磁体的保护}

\newpage
\subsection{问题4:混合III SCM的“热点”温度}

\newpage
\subsection{讨论5:失超电压探测——一个变种}

\newpage
\subsection{问题6:抑制电阻的设计}

\newpage
\subsection{讨论7:磁体的“缓慢”放电模式}

\newpage
\subsection{讨论8:低阻电阻器设计}

\newpage
\subsection{讨论9:过热\& 内部电压判据}

\newpage
\subsection{讨论10:Bi2223带电流引线的保护}

\newpage
\subsection{讨论11:$MgB_2$磁体的主动保护}

\newpage
\subsection{问题12:NMR磁体的被动保护}

\newpage
\subsection{讨论13:HTS磁体到底要不要保护?}
