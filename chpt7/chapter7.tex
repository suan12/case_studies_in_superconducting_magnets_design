\chapter{交流及其他损耗}
\section{引言}

\section{交流损耗}
\subsubsection*{超导体朝向与外磁场的关系}

\subsubsection*{时变磁场}

\subsubsection*{交流损耗的能量密度表}

\subsection{磁滞损耗}

\begin{equation}% 7.1
\int\left[-\int_{S}^{}\vec{E}\times\vec{H}\cdot d\vec{\ \mathcal{A}}\right]dt=\int_{\nu}^{}\left[\int\vec{E}\cdot\vec{J}dt+\frac{1}{2}\mu_oH^2+\mu_oH\int\vec{H}\cdot d\vec{M}\right]d\nu
\end{equation}

\subsubsection*{Bean板的磁滞损耗}
\begin{equation}% 7.2
\int\left[-\int_{S}\vec{E}(x)\times\vec{H}_e\cdot d\vec{\ \mathcal{A}}\right]dt=\int_{0}^{2a}\left[\int\vec{E}(x)\cdot\vec{J}_c(x)dt+\frac{1}{2}\mu_oH_{s}^{2}(x)\right]dx
\end{equation}
\begin{equation}% 7.3a
e_{hy}=\frac{1}{2a}\int_{0}^{2a}\left[\int J_cE(x)dt\right]dx
\end{equation}
\begin{equation}% 7.3b
e_{hy}=\frac{1}{2a}\{\int\left[-\int_{S}\vec{E}(x)\times\vec{H}_e\cdot d\vec{\ \mathcal{A}}\right]dt-\frac{1}{2}\mu_oH\int_{0}^{2a}H_{s}^{2}(x)dx\}
\end{equation}
\begin{equation}% 7.4a
e_{hy}=\mu_o\oint\vec{H}_ed\vec{M}_e(\vec{H}_e)
\end{equation}
\begin{equation}% 7.4b
e_{hy}=-\mu_o\oint M(H_e)dH_e
\end{equation}


\subsubsection*{外磁场时间序列下的Bean板}
\begin{equation}% 7.5
H_e(t)=0*(\ \mathrm{Virgin slab})\rightarrow H_m\rightarrow 0\rightarrow -H_m\rightarrow 0\rightarrow H_m\rightarrow 0\rightarrow -H_m\rightarrow 0
\end{equation}


\subsection{多丝复合物中的耦合损耗}

\subsubsection*{耦合时间常数}
\begin{equation}% 7.6
\tau_{cp}=\frac{\mu_o\ell_{p}^{2}}{8\pi\rho_{ef}}
\end{equation}


\subsubsection*{有效基底电阻}

\begin{equation}% 7.7a
\rho_{ef0}=\frac{1-\lambda_f}{1+\lambda_f}\rho_m
\end{equation}
\begin{equation}% 7.7b
\rho_{ef\infty}=\frac{1+\lambda_f}{1-\lambda_f}\rho_m
\end{equation}

\subsection{涡流损耗}

\section{其他损耗}

\subsection{分段电阻}



\subsubsection*{搭接电阻(接头)}
\begin{equation}% 7.8
R_{sd}=\frac{\rho_{sd}\delta_{sd}}{a\ell_{sp}}
\end{equation}


\subsubsection*{接触电阻}
\begin{equation}% 7.9a 7.9b
R_{sp}=R_{cA}+R_{sd}+R_{cB} 
=\frac{R_{ct}}{A_{ct}}
\end{equation}
\begin{equation}% 7.9c
R_{sp}=\frac{R_{ct}}{A_{ct}}\simeq R_{sd}
\end{equation}
\begin{equation}% 7.10
R_{ct}\simeq\rho_{sd}\delta_{sd}
\end{equation}


\subsubsection*{机械接触开关}

\subsection{机械扰动}

\subsubsection*{导体移动和矫正}
\begin{equation}% 7.11
e_f=\mu_ff_{L_r}\Delta r_f
\end{equation}
\begin{equation}% page411最后一个
\Delta r_f=\frac{e_f}{\mu_ff_{L_r}}=\frac{(1300\ \mathrm{J/m^3})}{(0.3)(2\times 10^8\ \mathrm{N/m^3})}\simeq 20\times 10^{-6}\ \mathrm{m}=20\ \mathrm{\mu m}
\end{equation}


\subsubsection*{填充材料分裂和矫正}

\section{声发射技术}
\subsection{机械事件探测——LTS磁体}

\subsection{应用于HTS磁体}




\section{专题}
\subsection{问题7.1:磁滞能量密度——在“小”磁场时间序列的“纯”Bean板}

\begin{equation}% 5.5
-M(H_e)=H_e-\frac{H_{e}^{2}}{2H_p}        (H_e=0*\rightarrow H_m\leq H_p)
\end{equation}
\begin{equation}% 7.12
-M(H_e)=H_e+\frac{H_{e}^{2}-2H_mH_e-H_{m}^{2}}{4H_p}      (H_e=H_m\rightarrow 0)
\end{equation}
\begin{equation}% 7.13a
e_{hy}=\frac{\mu_oH_{m}^{3}}{6H_p}        (0\leq H_m\leq H_p)
\end{equation}
\begin{equation}% 7.14a
e_{hy}=\frac{\mu_oH_{m}^{3}}{24H_p}       (0\leq H_m\leq H_p)
\end{equation}
\begin{equation}% 7.15a
e_{hy}=\frac{5\mu_oH_{m}^{3}}{24H_p}      (0\leq H_m\leq H_p)
\end{equation}

\subsubsection{问题7.1之解}
\begin{equation}% page416 1
E_z(x)=\mu_o\frac{dH_e}{dt}(x_+-x)
\end{equation}
\begin{equation}% page416 2
E_z(x)dt=\mu_o(x_+-x)dH_e
\end{equation}
\begin{equation}% page416 3 and 4
e_{hy}=\frac{1}{a}\int_{0}^{a}\left[\int J_cE(x)dt\right]dx 
=\frac{\mu_oF_c}{a}\int_{0}^{H_m}\left[\int_{0}^{x_+(x_+-x)dx}\right]dH_e
\end{equation}
\begin{equation}% page416 5 and 6
e_{hy}=\frac{\mu_oJ_c}{a}\int_{0}^{H_m}\left(x_{+}^{2}-\frac{x_{+}^{2}}{2}\right)dH_e 
=\frac{\mu_oJ_c}{a}\int_{0}^{H_m}\frac{H_{e}^{2}}{2J_{c}^{2}}dH_e=\frac{\mu_oH_{m}^{3}}{6aJ_c}
\end{equation}
\begin{equation}% 7.13a
e_{hy}=\frac{\mu_oH_{m}^{3}}{6H_p}    (H_m\leq H_p)
\end{equation}
\begin{equation}% page416 S1.3
E_z(x)dt=\mu_o(x-x_-)dH_e
\end{equation}
\begin{align*}% page416 S1.4
e_{hy}&=\frac{\mu_oJ_c}{a}\int_{H_m}^{0}\left[\int_{0}^{x_-}(x-x_-)dx\right]dH_e=\frac{\mu_oJ_c}{a}\int_{H_m}^{0}\left(\frac{x_{-}^{2}}{2}-x_{-}^{2}\right)dH_e \\\notag
&=\frac{\mu_oJ_c}{a}\int_{H_m}^{0}\frac{(H_m-H_e)^2}{8J_{c}^{2}}dH_e=\frac{\mu_oH_{m}^{3}}{24aJ_c}
\end{align*}
\begin{equation}% 7.14a
e_{hy}=\frac{\mu_oH_{m}^{3}}{24H_p}       (0\leq H_m\leq H_p)
\end{equation}
\begin{equation}% page417 S1.5a
E_z(0)=\mu_o\frac{H_e}{J_c}\left(\frac{dH_e}{dt}\right)
\end{equation}
\begin{equation}% page417 S1.6a两个
e_{py1}\equiv\frac{1}{a}\int\left[-\int_{S}\vec{E}(x)\times\vec{H}_e\cdot d\vec{\ \mathcal{A}}\right]dt=\frac{\mu_o}{aJ_c}\int_{0}^{H_m}H_{e}^{2}dH_e
e_{py1}=\frac{\mu_oH_{m}^{3}}{3H_p}
\end{equation}
\begin{equation}% page417 S1.5b
E_z(0)=-\mu_o\left(\frac{H_m-H_e}{2J_c}\right)\frac{dH_e}{dt}
\end{equation}
\begin{equation}% page417 S1.6b两个
e_{py2}=\frac{\mu_o}{2aJ_c}\int_{H_m}^{0}(H_m-H_e)H_edH_e
e_{py2}=-\frac{\mu_oH_{m}^{3}}{12H_p}
\end{equation}
\begin{equation}% page417 S1.7a
H_s(x)=J_cx        (0\leq x\leq x_0)
\end{equation}
\begin{equation}% page417 S1.7b
H_s(x)=H_m-J_cx    (x_0\leq x\leq H_m/J_c)
\end{equation}
\begin{equation}% page417 S1.8
e_{m_f}=\frac{\mu_o}{2a}\int_{0}^{a}H_{s}^{2}(x)dx=\frac{\mu_o}{2a}\left(2\times\int_{0}^{\frac{H_m}{2J_c}}J_{c}^{2}x^2dx\right)
e_{m_f}=\frac{\mu_oH_{m}^{3}}{24H_p}
\end{equation}
\begin{equation}% 7.15a两个
e_{hy}=e_{py1}+e_{py2}-e_{m_f} 
=\frac{\mu_oH_{m}^{3}}{3H_p}-\frac{\mu_oH_{m}^{3}}{12H_p}-\frac{\mu_oH_{m}^{3}}{24H_p}
e_{hy}=\frac{5\mu_oH_{m}^{3}}{24H_p}       (0\leq H_m\leq H_p)
\end{equation}
\begin{equation}% page418 S1.9
e_{py}=e_{hy}+e_{m_f}-e_{m_i}
\end{equation}


\begin{align*}% page418 S1.10两个
e_{m_{f1}}&=\frac{\mu_o}{2a}\int_{0}^{a}H_{s}^{2}(x)dx \\\notag
&=\frac{\mu_o}{2a}\left[\int_{0}^{\frac{H_m}{J_c}}(H_m-J_cx)^2dx\right]
\end{align*}
\begin{align*}
e_{m_{f1}}=\frac{\mu_oH_{m}^{3}}{6H_p}
\end{align*}



\begin{equation}% page418 S1.11a
e_{py1}=\frac{\mu_oH_{m}^{3}}{6H_p}+\frac{\mu_oH_{m}^{3}}{6H_p}=\frac{\mu_oH_{m}^{3}}{3H_p}
\end{equation}
\begin{equation}% page418 S1.11b
e_{py2}=\frac{\mu_oH_{m}^{3}}{24H_p}+\frac{\mu_oH_{m}^{3}}{24H_p}-\frac{\mu_oH_{m}^{3}}{6H_p}=\frac{\mu_oH_{m}^{3}}{12H_p}
\end{equation}


\subsection{问题7.2:磁滞能量密度——在“中”磁场时间序列的“纯”Bean板}


\begin{equation}% 5.5和5.6
-M(H_e)=H_e-\frac{H_{e}^{2}}{2H_p}                          (H_e=0*\rightarrow H_p)
=\frac{1}{2}H_p                                      (H_e=H_p\rightarrow H_m)
\end{equation}
\begin{equation}% 5.7a
-M(H_e)=\frac{1}{2}H_p-(H_m-H_e)+\frac{(H_m-H_e)^2}{4H_p}   (H_e=H_m\rightarrow 0)
\end{equation}
\begin{equation}% 7.13b
e_{hy}=\frac{1}{2}\mu_oH_pH_m\left(1-\frac{2H_p}{3H_m}\right)   (H_p\leq H_m\leq 2H_p)
\end{equation}
\begin{equation}% 7.15b
e_{hy}=\frac{1}{2}\mu_oH_pH_m\left[1-\frac{2H_p}{3H_m}+\frac{1}{12}\left(\frac{H_m}{H_p}\right)^2\right]     (H_p\leq H_m \leq 2H_p)
\end{equation}

\subsubsection{问题7.2之解}
\begin{equation}% page420 S2.1
e_{hy1^\prime}=\frac{1}{6}\mu_oH_{p}^{2}       (H_e=0\rightarrow H_p)
\end{equation}
\begin{equation}% page420 S2.2
E_z(x)=\mu_o\frac{dH_e}{dt}(a-x)
\end{equation}

\begin{subequations}
\begin{align*}% page420 S2.3a和2.3b
e_{hy1^\prime}&=\frac{1}{a}\int_{0}^{a}\left[\int J_cE(x)dt\right]dx=\frac{\mu_oJ_c}{a}\int_{0}^{a}\left[\int_{H_p}^{H_m}(a-x)dH_e\right]dx \\
&=\mu_oJ_c(H_m-H_p)\int_{0}^{a}\frac{a-x}{a}dx=\frac{1}{2}\mu_oH_p(H_m-H_p)
\end{align*}
\end{subequations}




\begin{equation}% page420 7.13b两个
e_{hy}=\frac{1}{6}\mu_oH_{p}^{2}+\frac{1}{2}\mu_oH_p(H_p-H_m) 
=\frac{1}{2}\mu_oH_pH_m-\frac{1}{3}\mu_oH_{p}^{2}
e_{hy}=\frac{1}{2}\mu_oH_pH_m\left(1-\frac{2H_p}{3H_m}\right)       (H_p\leq H_m\leq 2H_p)
\end{equation}
\begin{equation}% page420 S2.4a
e_{py1^\prime}=\frac{1}{3}\mu_oH_{p}^{2}
\end{equation}
\begin{equation}% page420 S2.5
E_z(0)=\mu_oa\frac{dH_e}{dt}
\end{equation}
\begin{equation}% page420 2.4b两个
e_{pyq^{\prime\prime}}\equiv\frac{1}{a}\int\left[-\int_{S}\vec{E}(x)\times\vec{H}_e\cdot d\vec{\ \mathcal{A}}\right]dt=\mu_o\int_{H_p}^{H_m}H_edH_e
e_{pyq^{\prime\prime}}=\frac{1}{2}\mu_o(H_{m}^{2}-H_{p}^{2})
\end{equation}
\begin{equation}% page421 S1.6b
e_{py2}=-\frac{\mu_oH_{m}^{3}}{12H_p}
\end{equation}
\begin{equation}% page421 S2.7两个
e_{m_f}=\frac{\mu_o}{2a}\int_{0}^{a}H_{s}^{2}(x)dx=\frac{\mu_o}{2a}\left[\int_{0}^{\frac{H_m}{2J_c}}J_{c}^{2}x^2dx+\int_{\frac{H_m}{2J_c}}^{a}(H_m-J_cx)^2dx\right]
e_{m_f}=\frac{1}{2}\mu_oH_{m}^{2}-\frac{\mu_oH_{m}^{3}}{8H_p}-\frac{1}{2}\mu_oH_mH_p+\frac{1}{6}\mu_oH_{p}^{2}
\end{equation}

\begin{align*}% page421 S2.8一个
e_{hy}&=e_{py1^\prime}+e_{py1^{\prime\prime}}+e_{py}-e_{m_f} \\\notag
&=\frac{1}{3}\mu_oH_{p}^{2}+\frac{1}{2}\mu_o(H_{m}^{2}-H_{p}^{2})-\frac{\mu_oH_{m}^{3}}{12H_p} 
-\left(\frac{1}{2}\mu_oH_{m}^{2}-\frac{\mu_oH_{m}^{3}}{8H_p}-\frac{1}{2}\mu_oH_mH_p+\frac{1}{6}\mu_oH_{p}^{2}\right) \\\notag
&=-\frac{1}{3}\mu_oH_{p}^{2}+\frac{\mu_oH_{m}^{3}}{24H_p}+\frac{1}{2}\mu_oH_pH_m \tag{S2.8}
\end{align*}

\begin{equation}% 7.15b
e_{hy}=\frac{1}{2}\mu_oH_pH_m\left[1-\frac{2H_p}{3H_m}+\frac{1}{12}\left(\frac{H_m}{H_p}\right)^2\right]     (H_p\leq H_m\leq 2H_p)
\end{equation}

\begin{equation}% 7.15a
e_{hy}=\frac{5\mu_oH_{m}^{3}}{24H_p}    (H_m\leq H_p) 
=\frac{5}{24}\mu_oH_{p}^{2}
\end{equation}

\begin{equation}% 7.15b
e_{hy}=\frac{1}{2}\mu_oH_pH_m\left[1-\frac{2H_p}{3H_m}+\frac{1}{12}\left(\frac{H_m}{H_p}\right)^2\right]    (H_p\leq H_m\leq 2H_p) 
=\frac{1}{2}\mu_oH_{p}^{2}\left[1-\frac{2}{3}+\frac{1}{12}\right]=\frac{5}{24}\mu_oH_{p}^{2}
\end{equation}


\subsection{问题7.3:磁滞能量密度——在“大”磁场时间序列的“纯”Bean板}

\begin{equation}% 7.16a
e_{hy2^\prime}=\frac{1}{3}\mu_oH_{p}^{2}
\end{equation}
\begin{equation}% 7.16b
e_{hy2^{\prime\prime}}=\frac{1}{2}\mu_oH_pH_m\left(1-\frac{2H_p}{H_m}\right)
\end{equation}
\begin{equation}% 7.14b
e_{hy}=\frac{1}{2}\mu_oH_pH_m\left(1-\frac{4H_p}{3H_m}\right)     (H_m\geq 2H_p)
\end{equation}
\begin{equation}% 7.15c
e_{hy}=\mu_oH_pH_m\left(1-\frac{H_p}{H_m}\right)      (H_m\geq 2H_p)
\end{equation}

\subsubsection{问题7.3之解}






\newpage
\subsection{讨论4:磁滞能量密度——磁化的Bean板}

\newpage
\subsection{讨论5:载有直流电流的Bean板}

\newpage
\subsection{问题6:磁滞能量密度——载有直流电流的Bean板}

\newpage
\subsection{问题7:自场磁滞能量密度——Bean板}

\newpage
\subsection{讨论8:磁体整体的交流损耗}

\newpage
\subsection{讨论9:测量交流损耗的技术}

\newpage
\subsection{讨论10:CICC中的交流损耗}

\newpage
\subsection{讨论11:HTS中的交流损耗}

\newpage
\subsection{问题12:$Nb_3Sn$中的磁滞损耗}

\newpage
\subsection{问题13:混合III SCM中的交流损耗}

\newpage
\subsection{讨论14:混合III NbTi线圈中的分段耗散}

\newpage
\subsection{讨论15:持续模式运行\&“指数”}

