\chapter{交流及其他损耗}
\section{引言}

\section{交流损耗}
\subsubsection*{超导体朝向与外磁场的关系}

\subsubsection*{时变磁场}

\subsubsection*{交流损耗的能量密度表}

\subsection{磁滞损耗}
\subsubsection*{Bean板的磁滞损耗}

\subsubsection*{外磁场时间序列下的Bean板}

\subsection{多丝复合物中的耦合损耗}

\subsubsection*{耦合时间常数}

\subsubsection*{有效基底电阻}


\subsection{涡流损耗}

\section{其他损耗}

\subsection{分段电阻}

\subsubsection*{搭接电阻(接头)}

\subsubsection*{接触电阻}

\subsubsection*{机械接触开关}

\subsection{机械扰动}

\subsubsection*{导体移动和矫正}

\subsubsection*{填充材料分裂和矫正}

\section{声发射技术}
\subsection{机械事件探测——LTS磁体}

\subsection{应用于HTS磁体}

\section{专题}
\subsection{问题1:磁滞能量密度——在“小”磁场时间序列的“纯”Bean板}

\newpage
\subsection{问题2:磁滞能量密度——在“中”磁场时间序列的“纯”Bean板}

\newpage
\subsection{问题3:磁滞能量密度——在“大”磁场时间序列的“纯”Bean板}

\newpage
\subsection{讨论4:磁滞能量密度——磁化的Bean板}

\newpage
\subsection{讨论5:载有直流电流的Bean板}

\newpage
\subsection{问题6:磁滞能量密度——载有直流电流的Bean板}

\newpage
\subsection{问题7:自场磁滞能量密度——Bean板}

\newpage
\subsection{讨论8:磁体整体的交流损耗}

\newpage
\subsection{讨论9:测量交流损耗的技术}

\newpage
\subsection{讨论10:CIC导体中的交流损耗}

\newpage
\subsection{讨论11:HTS中的交流损耗}

\newpage
\subsection{问题12:$Nb_3Sn$中的磁滞损耗}

\newpage
\subsection{问题13:混合III SCM中的交流损耗}

\newpage
\subsection{讨论14:混合III NbTi线圈中的分段耗散}

\newpage
\subsection{讨论15:持续模式运行\&“指数”}

