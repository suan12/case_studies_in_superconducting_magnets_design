\chapter{稳定性}
\section{引言}
可靠性是所有器件设备都必须满足的一个主要指标,超导磁体当然不例外。历史上看,可靠性曾是超导磁体中最困难,因而也是最具挑战性的一个方面。
如图1.5所示,超导电性存在于由三个参数(电流密度$J$,磁场$H$和温度$T$)为边界的相体积内部。

这三个参数中的电流密度和磁场,至少在正常运行条件下,设计者是可以很好的定义并控制它们的。甚至在复杂的故障模式条件下,比如含有多个螺管的混合磁体或嵌套多线圈磁体,
电流密度和磁场也是可控的。可以说:磁体设计者能够牢牢掌控这两个参数。
到了温度参数,就不如此了。温度这三个参数中最难控的。相对运行点的温度偏移,在时间上难以预测,在空间上(即在线圈内部)就更难控制了。
储存在磁体内的磁场能和机械能,很容易转化为热能,引起线圈内部某些位置的导体温度上升到超过其临界值。
实际上,最终超导磁体的所有“稳定性问题”都可以归为磁体设计者不能控制线圈温度保持在运行点。

本章我们考虑 1)超导线圈内控制温度的基本物理问题;2)线圈内不可预测温升发生可能性的稳定性评估方法。
第7和第8章同样讨论不同条件下线圈内温升:第7章讨论温升的原因或源;第8章讨论控制磁体不可预测温升的保护方法。
首先,LTS和HTS磁体的稳定性问题就截然不同。

\textbf{LTS vs. HTS}

如图1.6给出的,稳定性的实现难度和费用随运行温度提高而降低。在下面的讨论中,我们将比较HTS和LTS的“稳定裕度”,并证明HTS磁体实际上
是非常稳定的。换句话说,任何HTS磁体都会达到其运行电流,且不会存在``提前失超(premature quench)``。
这种偶然事件仍经常影响``高性能``(即绝热和高电流密度)LTS磁体。
这意味着稳定性对于HTS磁体来说,不像在LTS磁体一样,是设计和运行中很严重的问题。
不过,稳定性仍然是HTS磁体的关键问题[6.1–6.8]。

\section{稳定性理论和标准}
我们从含温度$T$的单位超导体体积的功率密度方程出发,讨论载有额定电流$I_{op}$的磁体的热稳定性:
\begin{equation}
C_{cd}(T)\frac{\partial T}{\partial t}=\nabla ·[k_{cd}(T)\nabla T]+\rho_{cd}(T)J_{cd_o}^2(t)+g_d(t)-(\frac{f_p\ \mathrm{P_D}}{A_{cd}})g_q(T)
\end{equation}
式中,等式左侧导体热能密度的时间变化率,其中$C_{cd}(T)$是导体单位体积的热容。
这个公式是Stekly针对复合超导体,即由超导体和正常金属基底组成的超导体,提出的。
对于纯稳态稳定性,左侧项必须总为零;实际上,对大多数线圈,甚至对绝热线圈,
一个在运行温度$T_{op}$附近很小的温度漂移$\Delta T_{op}$都是允许的。
如前所述,因为HTS磁体允许的$\Delta T_{op}$通常远大于LTS磁体,稳定性对HTS磁体来说,
几乎不成问题。这一点下面将更详细的阐述。

等号右侧的各项均为单位体积值。第一项是通过热传导进入复合导体的热量,式中$k_{cd}(T)$是复合导体的热导率。
第二项是焦耳热,式中$\rho_{cd}(T)$是复合导体的电阻率(在超导态为零),$J_{cd_o}(t)$是运行电流$I_{op}(t)$
下的依赖于时间的电流密度。$g_d(t)$给出的是主要由磁和机械效应产生的非焦耳热。最后一项表示冷却,其中$f_p$
是复合导体与制冷机接触的湿润周长$\mathcal{P}_D$分数,$A_{cd}$是复合导体截面积,$g_d(T)$是与制冷剂
之间的对流热流。

稳定性(以及第8章将要讨论的保护)理论和概念的发展史就是对方程6.1简化求解的历史。
表6.1列出了由方程6.1在特定工况下导出的概念。
表中,标为0的参数表示它在方程中可以忽略或者不予考虑。$\surd$表示要予以考虑。
在讨论方程6.1的每一项之前,我们先简要讨论表6.1中列出的概念。

\colorbox[red]{表格6.1}

\subsection{公式6.1涉及的概念}
下面将简要讨论方程6.1所涉及的以及表6.1列出的概念。
\begin{description}
  \item[磁通跳跃] 第五章已经考察得到了避免通常会影响LTS的多数磁通跳跃的准则。
  \item[低温稳定性] 低温稳定性的基本概念是在1960年代中期作为实现磁体可靠运行的工程方案而提出的[6.9]。
  在一个低温稳定的复合导体中,超导体与高导电基底金属同步处理[6.10],导体的大部分表面与制冷工质接触以
  保证“局域”冷却。如表6.1所给出的,除了焦耳热项和冷却项,其他项可以忽略。1970年代很多成功的磁体都是
  低温稳定的[6.11, 6.12];当前,它仅用于``大型``LTS磁体。稍后我们将看到,它并不用于HTS磁体。
  低温稳定的概念将在本章的专题中进一步研究。
  \item[动态稳定性] 第五章研究过第II类超导体中当磁扩散远大于热扩散时,若导体的尺度(如带材)不足以抑制它,
  将发生磁通跳跃。通过将超导体和高热导率材料,比如铜,复合,我们可以平衡这两种扩散效应,实现无磁通跳跃
  的稳定运行。LTS带材如今已很少使用;磁通跳跃也和HTS带材不一样(讨论5.7)。所以,后文将不再讨论这个准则。
  \item[等效面积] ``等效面积``准则是低温稳定性的特例,即包含了方程6.1中的热传导项$\Delta\cdot[k_{cd}(T)\Delta T]$。
  这样,可以提高低温稳定磁体的总体电流密度。本准则将在专题中进一步讨论。
  \item[MPZ] 最小传播区域(minimum propagating zone, MPZ)的概念考虑在绕组中施加局域扰动$g_d(t)$对
  线圈性能的影响[6.13]。MPZ概念表明,即使在绕组中存在少量的正常态区域,磁体仍可能保持超导态,当然前提是
  正常态区域体积小于MPZ理论定义的临界尺度。它在绝热磁体中的重要性首先被Wilson在1970年代末期注意到[1.27],
  此后,它成为分析绝热磁体稳定性不可或缺的概念。MPZ概念将在专题中进一步研究。  
  \item[不稳定情况] 表6.1中最后两种情况涉及非稳定态的绕组热行为。第8章将讨论。
\end{description}

\subsection{热能}


\begin{equation}% page357 第1个
\Delta e_h=\int_{T_{op}}^{T_{cs}(I_{op})}{C_{cd}(T)d(T)}=\int_{T_{op}}^{T_{op}+[\Delta T_{op}(I_{op})]_{st}}{C_{cd}(T)d(T)}
\end{equation}
\begin{equation}% page357 第2个
[\Delta T_{op}(I_{op})]_{st}=(T_c-T_{op})(1-i_{op})
\end{equation}
\begin{equation}% page360 第1个
A_{cd}=A_{cs}+A_m+A_{\bar{m}}
\end{equation}
\begin{equation}% page360 第2个
A_{wd}=A_{cd}+A_S+A_{in}+A_q
\end{equation}
\begin{equation}% page360 第3个
J_c\equiv\frac{I_c}{A_{sc}}
\end{equation}
\begin{equation}% page360 第4个
J_c\equiv\frac{I_c}{A_{sc}+A_{\bar{m}}}
\end{equation}
\begin{equation}% page360 第5个
J_e=J_{cd}\equiv\frac{I_c}{A_{cd}}=\frac{I_c}{A_{sc}+A_m+A_{\bar{m}}}
\end{equation}
\begin{equation}% page361 第1个
J_{m}\equiv\frac{I_m}{A_m}    \        or        \    J_m(t)\equiv\frac{I_m(t)}{A_m}
\end{equation}
\begin{equation}% page361 第2个
I_{op}=I_t=I_m+I_s   \   or     \ I_{op}(t)=I_{t}(t)=I_m(t)+I_s(t)
\end{equation}
\begin{equation}% page361 第3个
J_{m_o}\equiv I_{op}   \    or\      J_{m_o}(t)\equiv\frac{I_{op}(t)}{A_m}
\end{equation}
\begin{equation}% page361 第4个
\lambda J\equiv\frac{I}{A_{wd}}
\end{equation}
\begin{equation}% page361 第5个
\lambda J_{op}\equiv\frac{I_op}{A_{wd}}   \  or   \   \lambda J_{op}(t)\equiv\frac{I_{op}(t)}{A_{wd}}
\end{equation}
\begin{equation}% page361 第6个
J_{cic_o}\equiv\frac{I_{op}}{A_{cic}}\ or\  J_{cic_o}\equiv\frac{I_{op}(t)}{A_{cic}}
\end{equation}
\begin{equation}% page361 第7个
A_{cic}\equiv A_{cd}+A_S+A_q
\end{equation}

\subsection{热能量}

\subsection{热传导}

\subsection{焦耳热}

\subsection{扰动谱}

\subsection{稳定裕度 vs 扰动能}

\subsection{冷却}

\section{电流密度}

\subsection{横截面积}

\subsection{复合物超导体}

\subsection{绕组中的电流密度}


\section{专题}
\subsection{讨论6.1:低温稳定性——电路模型}


\begin{equation}% page362 第1个 6.10a
V_{cd}=R_mI_m\simeq R_m{(I_t-Ic)}
\end{equation}
\begin{equation}% page362 第2个 6.10b
G_{j}=V_{cd}I_t
\end{equation}
\begin{equation}% page362 第3个 6.11
{(G_j)}\simeq R_{m}I_{t}{(I_t-I_c)}
\end{equation}



\subsection{问题6.1:低温稳定性——温度依赖}
\begin{equation}% page363 第1个 6.12
I_c(T)=I_{c_o}(\frac{T_c-T}{T_c-T_{o_p}}) (T_{o_p}\leq T \leq T_c)
\end{equation}
\begin{equation}% page363 第2个 6.13a
G_j(T)=0 (T_{o_p}\leq T \leq T_{cs})
\end{equation}
\begin{equation}% page363 第3个 6.13b
G_j(T)=R_mI_t^2(\frac{T-T_{cs}}{T_c-T_{cs}})(T_{cs}\leq T\leq T_c)
\end{equation}
\begin{equation}% page363 第4个 6.13c
G_j(T)=R_mI_t^2 (T\ge T_c)
\end{equation}

\subsubsection{问题6.1之解}
\begin{equation}% page364 第1个 6.13a
G_j(T)=0 (T_{o_p}\leq T \leq T_{cs})
\end{equation}
\begin{equation}% page364 第2个 S1.1
G_j(T)=R_mI_t[I_t-I_{co}(\frac{T_c-T}{T_c-T_{o_p}})] (T_{cs}\leq T\leq T_c)
\end{equation}

\begin{equation}% page364 第3个 S1.2
I_{c_o}=I_t(\frac{T_c-T_{op} }{T_c-T_{cs}})
\end{equation}
\begin{equation}% page364 第4个 6.13b
G_j(T)=R_mI_t^2(\frac{T-T_{cs}}{T_c-T_{cs}}) (T_{cs}\leq T\leq T_c)
\end{equation}
\begin{equation}% page364 第5个 6.13c
G_j(T)=R_mI_t^2 (T\ge T_c)
\end{equation}
\begin{equation}% page364 第6个 6.14a
T_{cs}\leq T\leq T_c:
\end{equation}
\begin{equation}% page364 第6个 6.14a
G_j(T)=R_m(T)I_t^2(\frac{T-T_{cs}}{T_c-T_{cs}})
\end{equation}
\begin{equation}% page364 第7个 6.14b
T\ge T_c:
\end{equation}
\begin{equation}% page364 第7个 6.14b
G_j(T)=R_m(T)I_t^2
\end{equation}


\subsection{讨论6.2:Stekly低温稳定性判据}
\begin{equation}% page365 第1个 6.15
C_{cd}(T)\frac {\partial T}{\partial t}=\nabla\cdot[k_{cd}(T)\nabla T]+\rho _{cd}(T)J_{cd}^2(t)+g_d(t)-(\frac{f_p\ \mathrm{P}_D}{A_{cd}})g_q(T)
\end{equation}
\begin{equation}% page365 第2个 6.16a
G_j(T)=R_mI_{c_o}^2(\frac{T-T_{op}}{T_c-T_{op}}) (T_{o_p}\leq T \leq T_c)
\end{equation}
\begin{equation}% page365 第2个 6.16b
\rho_{cd}J^2_{cd}(t)=\frac{\rho_{m}I_{co}^2}{A_{cd}A_m}
(\frac{T-T_{op}}{T_c-T_{op}}) (T_{o_p}\leq T \leq T_c)
\end{equation}
\begin{equation}% page365 第3个 6.17
g_q(T)=h_q(T-T_b)\simeq h_q(T-T_{op})
\end{equation}
\begin{equation}% page365 第4个 6.18
\frac{f_pP_Dh_q(T-T_{op}}{A_cd}\geq \frac{\rho_m I_{co}^2}{A_{cd}A_m}(\frac{T-T_{op}}{T_c-T_{op}})
\end{equation}
\begin{equation}% page365 第4个 6.18
\frac{\rho_m I_{co}^2}{f_pP_DA_mh_q(T_c-T_{op})}\leq 1
\end{equation}
\begin{equation}% page365 第5个 6.19
\alpha_{sk}=\frac{\rho_m I_{co}^2}{f_pP_DA_mh_q(T_c-T_{op})}
\end{equation}
\begin{equation}% page366 第1个 6.20
A_m=\frac {\rho_mI_{co}^2}{\alpha_{sk}f_pP_Dh_q(T_c-T_{op})}
\end{equation}
\begin{equation}% page366 第2个 6.6
J_{op}=\frac{I_{op}}{A_{sc}+A_ {\bar{m}}+A_m}
\end{equation}
\begin{equation}% page366 第2个 6.21a
=(\frac{\gamma_{m/s}}{\gamma_{m/s}+1})J_{m_0}
\end{equation}
\begin{equation}% page366 第3个 6.21b
\gamma_{m/s}\equiv \frac{A_m}{A_{sc}+A_m}
\end{equation}




\subsection{讨论6.3:复合物超导体}



\subsection{问题6.2:低温稳定性——非线性冷却曲线}
\begin{equation}% page367 第1个 6.22
[J_{m_o}]_{sk}=\sqrt{\frac{f_pP_Dq_{fm}}{\rho_mA_m}}
\end{equation}

\subsubsection{问题6.2之解}
\begin{equation}% page368 第1个 S2.1
\frac {\rho_mI_{co}^2}{A_m}=f_pP_Dq_{fm}
\end{equation}
\begin{equation}% page368 第2个 6.22
[J_{m_o}]_{sk}=\sqrt{\frac{f_pP_Dq_{fm}}{\rho_mA_m}}
\end{equation}
\begin{equation}% page368 第3个 s1.2
I_{c_o}=I_t(\frac{T_c-T_{op} }{T_c-T_{cs}})
\end{equation}

\begin{equation}% page368 第4个
g_j(T)=(\frac{A_{cd}}{f_pP_D})R_mI_t^2=(\frac{A_{cd}}{f_pP_D})R_mI_{co}^2\frac{(T_c-T_{cs})^2(T-T_{cs})}{(T_c-T_{op})^3}
\end{equation}
\begin{equation}% page368 第5个
\frac{dg_j(T)}{dT}=(\frac{A_{cd}}{f_pP_D})R_mI_{co}^2\frac{(T_c-T_{cs})^2}{(T_c-T_{op})^3}
\end{equation}



\subsection{讨论6.4:等效面积判据}
\begin{equation}% page369 第1个 6.23
\int_{T_{op}}^{T_{eq}}[g_q(T)-(\frac{A_{cd}}{f_pP_D})g_j(T)]d(T)
=\int_{T_{op}}^{T_{eq}}[g_q(T)-\hat{g}_j(T)]dT=0
\end{equation}
\begin{equation}% page369 第2个 6.24a
g_j(T)=\rho_m(T)J_{m_o}^2(\frac{T-T_{op}}{T_c-T_{op}}) (T_{op}\leq T \leq T_c)
\end{equation}
\begin{equation}% page369 第3个 6.24b
g_j(T)=\rho_m(T)J_{m_o}^2 (T \geq T_c)
\end{equation}


\subsection{讨论6.5:超导体“指数”n}
\begin{equation}% page370 第1个 6.25a
V_s=V_c(\frac{I_s}{I_c})^n
\end{equation}
\begin{equation}% page370 第2个 6.25b
E_s=E_c(\frac{I_s}{I_c})^n
\end{equation}



\subsection{问题6.3:复合物超导体(n)——电路模型}
\begin{equation}% page371 第1个 6.26a
R_s=R_c(\frac{I_s}{I_c})^{(n-1)}
\end{equation}
\begin{equation}% page371 第2个 6.26b
R_{dif}=nR_c(\frac{I_s}{I_c})^{(n-1)}
\end{equation}

\subsubsection{问题6.3之解}

\begin{equation}% page372 第1个 S3.1
R_s=\frac{V_s}{I_s}=\frac{V_c}{I_s}(\frac{I_s}{I_c})^2=\frac{V_c}{I_c}(\frac{I_s}{I_c})^{(n-1)}
\end{equation}
\begin{equation}% page372 第2个 6.26a
R_s=R_c(\frac{I_s}{I_c})^{(n-1)}
\end{equation}
\begin{equation}% page372 第3个 S3.2
R_{dif}=\frac{\partial V_s}{\partial I_s}=\frac{nV_c}{I_c}(\frac{I_s}{I_c})^{(n-1)}
\end{equation}
\begin{equation}% page372 第4个 6.26a
R_{dif}=nR_c(\frac{I_s}{I_c})^{(n-1)}
\end{equation}
\begin{equation}% page372 第5个S3.3a
I_t=I_m+I_s
\end{equation}
\begin{equation}% page372 第6个S3.3b
V_m=R_mI_m=V_s=V_c(\frac{I_s}{I_c})^{n}
\end{equation}
\begin{equation}% page372 第6个S3.3c
3\times 10^{-4}\ \mathrm{\Omega}\times I_m[\ \mathrm{A}]=10^{-5}V(\frac{90\ \mathrm{A}-I_m[\ \mathrm{A}]}{100\ \mathrm{A}})^{15}
\end{equation}
\begin{equation}% page372 第7个S3.4
P_{cd}=R_mI_mI_t=V_sI_t
\end{equation}



\subsection{问题6.4:电流脉冲下的YBCO}
\begin{equation}% page375 第1个6.27a
R_m(T)=0.190+1.530(\frac{T-77}{293-77})\ [\ \mathrm{m\Omega}]
\end{equation}
\begin{equation}% page375 第1个6.27b
I_c(T)=100(\frac{93-T}{93-77})[\ \mathrm{A}]
\end{equation}
\begin{equation}% page375 第1个6.27C
V_s(T)=5[\frac{I_s(T)}{I_c(T)}]^{10} [\ \mathrm{\mu V}]
\end{equation}


\subsubsection{问题6.4之解}

\begin{equation}% page376 第1个S4.1a
I_t=I_s+I_m
\end{equation}
\begin{equation}% page376 第2个S4.1b
R_mI_m=V_s(I_s)
\end{equation}
\begin{equation}% page376 第3个S4.2a
I_s=(290\ \mathrm{A})-I_m
\end{equation}
\begin{equation}% page376 第4个S4.2b
(0.19\times10^{-3}\ \mathrm{\Omega})I_m=(5\times10^{-6}\ \mathrm{V})(\frac{290\ \mathrm {A}-I_m}{100\ \mathrm{A}})^{10}
\end{equation}
\begin{equation}% page376 第5个S4.3
(18\times10^{-3}\ \mathrm{V})=(5\times10^{-6}\ \mathrm{V})(\frac{195.26\ \mathrm{A}}{100\ \mathrm{A}})^n
\end{equation}
\begin{equation}% page376 第6个S4.4
V_{cd}=40\times10^{-3}\ \mathrm{V}=R_m(T)I_m
\end{equation}
\begin{equation}% page376 第7个S4.5a
10\times10^{-3}=[{[0.190+1.530(\frac{T-77}{293-77})]^\times10^{-3}\ \mathrm{\Omega}}]I_m(T)
\end{equation}
\begin{equation}% page376 第8个S4 .5b I_m(T)=\frac{40\times10^{-3}\ \mathrm{V}}{[0.190+1.530(\frac{T-77}{293-77})]\times10^{-3}\ \mathrm{\Omega}}I_m(T)
\end{equation}
\begin{equation}% page376 第9个S4 .5c
=\frac{40}{0.190+1.530(\frac{T-77}{293-77})}\ \mathrm{A}
\end{equation}
\begin{equation}% page377 第1个S4.6
40\times10^{-3}\ \mathrm{V}=(5\times10^{-6}V)[\frac{300A-I_m(T)}{I_c(T)}]^{12.24}
\end{equation}
\begin{equation}% page377 第2个S4.7
40\times10^{-3}\ \mathrm{V}=(5\times10^{-6}V)[\frac{300\ \mathrm{A}-\frac{40\ \mathrm{A}}{[0.190+1.530(\frac{T-77}{293-77})]}}{(100\ \mathrm{A})(\frac{93-T}{93-77})}]^{12.24}
\end{equation}
\begin{equation}% page377 第3个S4.8
8000=[\frac{16(406200T-28027799)}{(93-T)(135400T-6793895)}]^{12.24}
\end{equation}


\subsection{讨论6.6:CICC导体}
\begin{equation}% page378 第1个6.28a
C_{cd}(T)\frac{\partial T}{\partial t}=\nabla.[k_{cd}(T)\nabla T]+\rho_{cd}(T)J_{cd_o}^2(t)+g_d(t)-(\frac{f_pP_D}{A_{cd}})h_{he}(T-T_{he})
\end{equation}
\begin{equation}% page378 第2个6.28b
C_{he}(T_{he})\frac{\partial T_{he}}{\partial t}=(\frac{f_{p}P_D}{A_cd})h_{he}(T-T_{he})
\end{equation}
\begin{equation}% page379 第1个6.29
h_{he}=0.0259(\frac{k_{he}}{D_{hy}})Re^{0.8}Pr^{0.4}(\frac{T_{he}}{T_{cd}})^{-0.716}
\end{equation}
\begin{equation}% page381 第1个6.30
I_{lim}=\sqrt{\frac{A_mf_p\ \mathrm{P}_Dh_{he}(T_c-T_{op})}{\rho_m}}
\end{equation}




\subsection{问题6.5:冷却复合物导体的伏安关系}


\begin{equation}% page383 第1个6.31
V=\frac{R_m(I-I_{co})}{1-\frac{R_mII_{co}}{f_pP_{cd}\ell h_q(T_c-T_{op})}}
\end{equation}
\begin{equation}% page383 第2个6.32
\ \mathrm{u}(\ \mathrm{i})=\frac{\ \mathrm{i}-1}{1-\alpha_{ski}}
\end{equation}


\subsubsection{问题6.5之解}
\begin{equation}% page384 第1个S5.1
G_{j}(T_{op}+\triangle T)=VI=R_m\{I-I_{co}[\frac{T_c-(T_{op}+\Delta T)}{T_c-T_{op}}]\}I
\end{equation}
\begin{equation}% page384 第1个S5.1
=R_mI[(I-I_{co})+\frac{I_{co}\Delta T}{T_c-T_{op}}]
\end{equation}
\begin{equation}% page384 第2个S52
\Delta T=\frac{R_mI(I-I_{c_o})(T-T_{op})}{f_pP_{cd}\ell h_q(T_c-T_{op})-R_mI_{c_o}I}
\end{equation}
\begin{equation}% page384 第3个S5.3a
V=R_m\{(I-I_{c_o})+\frac{I_{c_o}}{T_c-T_{op}}[\frac{R_mI(I-I_{co})(T_c-T_{op})}{f_pP_{cd}\ell h_q(T_c-T_{op})-R_mI_{c_o}I}]\}
\end{equation}
\begin{equation}% page384 第4个S5.3b
=R_m(I-I_{c_o}))+\frac{R_m^2I_{c_o}(I-I_(co))}{f_pP_{cd}\ell h_q(T_c-T_{op})-R_mI_{c_o}I}
\end{equation}
\begin{equation}% page384 第5个6.31
V=\frac{R_m(I-I_{co})}{1-\frac{R_mII_{co}}{f_pP_{cd}\ell h_q(T_c-T_{op})}}
\end{equation}
\begin{equation}% page384 第6个
V=\frac{R_m(I-I_{co})}{1-\alpha_{sk}(I/I_{co})}
\end{equation}
\begin{equation}% page384 第7个
v(i)=\frac{i-1}{1-\alpha_{sk}i}
\end{equation}
\begin{equation}% page384 第8个
\alpha_{sk}=\frac{\rho_mI_{co^2}}{f_pP_{cd}A_mh_q(T_c-T_{op})}
\end{equation}
\begin{equation}% page384 第9个
=\frac{(4\times10^{-10}\ \mathrm{\Omega m})(100\ \mathrm{A})^2}{(1)(2\times10^{-2}\ \mathrm{m^2})(10^4\ \mathrm{W/m^2K})(5.2\ \mathrm{K}-4.2\ \mathrm{K})}=0.1
\end{equation}
\begin{equation}% page384 第10个
v(i)=\frac{10(i-1)}{10-i}
\end{equation}


\subsection{问题6.6:混合III SCM的稳定性分析}

\begin{equation}% page387 第1个
q_k=a_k(T^{n_k}_{cd}-T^{n_k}_b)
\end{equation}
\subsubsection{问题6.6之解}



\subsection{讨论6.7:cryostable vs 准绝热磁体}



\subsection{讨论6.8:MPZ概念}



\begin{equation}% page391 第1个
R_{mz}=\sqrt{\frac{3k_{wd}(T_c-T_{op})}{\rho_mJ_m^2}}
\end{equation}



\subsection{问题6.7:绝热绕组中的耗散能密度}

\begin{equation}% page392 第1个
0=\nabla.[k_{wd}(T)\nabla T]+g_d(t)
\end{equation}
\begin{equation}% page392 第2个
T(\rho)=\frac{a_1^2gd}{4k_{wd}}[(1-\rho^2)+(\frac{\alpha^2-1}{\ln \alpha})\ln\rho]+T_{op}
\end{equation}
\begin{equation}% page392 第3个
\rho_{mx}=\sqrt{\frac{\alpha^2-1}{2\ln \alpha}}
\end{equation}
\begin{equation}% page392 第4个
g_{d_c}=\frac{4k_{wd}\Delta T_{mx}}{a_1^2\{1+\frac{\alpha^2-1}{2\ln \alpha}[\ln{(\frac{\alpha^2-1}{2\ln\alpha})}-1]\}}
\end{equation}
\begin{equation}% page392 第5个
g_{dc}=(\frac{k_{wd}\Delta T_{mx}}{a_1^2})\gamma_{d_c}(\alpha)
\end{equation}
\begin{equation}% page392 第6个
\gamma_{d_c}(\alpha)\equiv\frac{4}{1+\frac{\alpha-1}{2\ln\alpha}[\ln(\frac{\alpha^2-1}{2\ln\alpha})-1]}
\end{equation}


\subsubsection{问题6.7之解}
\begin{equation}% page393 第1个
\frac{k_{wd}}{r} \frac{d}{dr}(r\frac{dT}{dr})+g_d=0
\end{equation}
\begin{equation}% page393 第2个
\frac{k_{wd}}{\rho}\frac{d}{d\rho}(\rho\frac{dT}{d\rho})+g_da_1^2=0
\end{equation}
\begin{equation}% page393 第3个
T(\rho)=-\frac{g_da_1^2}{4k_{wd}}\rho^2+A\ln\rho+B
\end{equation}
\begin{equation}% page393 第4个
T(\rho)=-\frac{g_da_1^2}{4k_{wd}}[(1-\rho^2)+(\frac{\alpha-1}{\ln\alpha})\ln\rho]+T_{op}
\end{equation}
\begin{equation}% page393 第5个
\frac{dT}{d\rho}=\frac{g_da_1^2}{4k_{wd}}[-2\rho+(\frac{\alpha^2-1}{\ln\alpha})\frac{1}{\rho}]=0
\end{equation}
\begin{equation}% page393 第6个
\rho_{mx}=\sqrt{\frac{\alpha^2-1}{2\ln\alpha}}
\end{equation}
\begin{equation}% page393 第7个
T(\rho_{mx})\equiv T_{mx}=\frac{a_1^2g_d}{4k_{wd}}[(1-\rho_{mx}^2)+(\frac{\alpha^2-1}{\ln\alpha})\ln\rho_{mx}]+T_{op}
\end{equation}
\begin{equation}% page393 第8个
\Delta T_{mx}=\frac{g_{d_c}a_1^2}{4k_{wd}}(1+\frac{\alpha^2-1}{2\ln\alpha}[\ln(\frac{\alpha^2-1}{2\ln\alpha})-1])
\end{equation}
\begin{equation}% page393 第9个
g_{d_c}=\frac{4k_{wd}\Delta T_{mx}}{a_1^2\{(1+\frac{\alpha^2-1}{2\ln\alpha}[\ln(\frac{\alpha^2-1}{2\ln\alpha})-)\}}
\end{equation}
\begin{equation}% page394 第1个
g_{d_c}=(\frac{k_{wd}\Delta T_{mx}}{a_1^2})\gamma_{d_c}(\alpha)
\end{equation}
\begin{equation}% page394 第1个
=\frac{(0.01\ \mathrm{W/cmK})(3\ \mathrm{K})}{(10\ \mathrm{cm})^2}(7.9)\simeq2.4\times10^{-3}\ \mathrm{W/cm^3}\simeq2.4\times10^3\ \mathrm{W/m^3}
\end{equation}


