\chapter{稳定性}
\section{引言}
可靠性是所有器件设备都必须满足的一个主要指标,超导磁体当然不例外。历史上看,可靠性曾是超导磁体中最困难,因而也是最具挑战性的一个方面。
如图1.5所示,超导电性存在于由三个参数(电流密度$J$,磁场$H$和温度$T$)为边界的相体积内部。

这三个参数中的电流密度和磁场,至少在正常运行条件下,设计者是可以很好的定义并控制它们的。甚至在复杂的故障模式条件下,比如含有多个螺管的混合磁体或嵌套多线圈磁体,
电流密度和磁场也是可控的。可以说:磁体设计者能够牢牢掌控这两个参数。
到了温度参数,就不如此了。温度这三个参数中最难控的。相对运行点的温度偏移,在时间上难以预测,在空间上(即在线圈内部)就更难控制了。
储存在磁体内的磁场能和机械能,很容易转化为热能,引起线圈内部某些位置的导体温度上升到超过其临界值。
实际上,最终超导磁体的所有“稳定性问题”都可以归为磁体设计者不能控制线圈温度保持在运行点。

本章我们考虑 1)超导线圈内控制温度的基本物理问题;2)线圈内不可预测温升发生可能性的稳定性评估方法。
第7和第8章同样讨论不同条件下线圈内温升:第7章讨论温升的原因或源;第8章讨论控制磁体不可预测温升的保护方法。
首先,LTS和HTS磁体的稳定性问题就截然不同。

\subsection*{LTS vs. HTS}
如图1.6给出的,稳定性的实现难度和费用随运行温度提高而降低。在下面的讨论中,我们将比较HTS和LTS的“稳定裕度”,并证明HTS磁体实际上
是非常稳定的。换句话说,任何HTS磁体都会达到其运行电流,且不会存在“提前失超(premature quench)”.

\section{稳定性理论和标准}

\subsection{公式6.1涉及的概念}
每一个概念……
\begin{description}
  \item[磁通跳跃] 如第五章……
  \item[低温稳定性] 基本概念……
  \item[动态稳定性] 第五章……
  \item[等效面积] 标准……
  \item[MPZ] 最小传播区域……
  \item[不稳定情况] 最后……
\end{description}

\subsection{热能量}

\subsection{热传导}

\subsection{焦耳热}

\subsection{扰动谱}

\subsection{稳定裕度 vs 扰动能}

\subsection{冷却}

\section{电流密度}

\subsection{横截面积}

\subsection{复合物超导体}

\subsection{绕组中的电流密度}


\section{专题}
\subsection{讨论1:低温稳定性——电路模型}

\newpage
\subsection{问题2:低温稳定性——温度依赖}

\newpage
\subsection{讨论3:Stekly低温稳定性判据}

\newpage
\subsection{讨论4:复合物超导体}

\newpage
\subsection{问题5:低温稳定性——非线性冷却曲线}

\newpage
\subsection{讨论6:等效面积判据}

\newpage
\subsection{讨论7:超导体“指数”n}

\newpage
\subsection{问题8:复合物超导体(n)——电路模型}

\newpage
\subsection{问题9:电流脉冲下的YBCO}

\newpage
\subsection{讨论10:CICC导体}

\newpage
\subsection{问题11:冷却复合物导体的伏安关系}

\newpage
\subsection{问题12:混合III SCM的稳定性分析}

\newpage
\subsection{讨论13:cryostable vs 准绝热磁体}

\newpage
\subsection{讨论14:MPZ概念}

\newpage
\subsection{问题15:绝热绕组中的耗散能密度}