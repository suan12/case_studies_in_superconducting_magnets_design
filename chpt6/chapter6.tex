\chapter{稳定性}
\section{引言}
可靠性是所有器件设备都必须满足的一个主要指标,超导磁体当然不例外。历史上看,可靠性曾是超导磁体中最困难,因而也是最具挑战性的一个方面。
如图1.5所示,超导电性存在于由三个参数(电流密度$J$,磁场$H$和温度$T$)为边界的相体积内部。

这三个参数中的电流密度和磁场,至少在正常运行条件下,设计者是可以很好的定义并控制它们的。甚至在复杂的故障模式条件下,比如含有多个螺管的混合磁体或嵌套多线圈磁体,
电流密度和磁场也是可控的。可以说:磁体设计者能够牢牢掌控这两个参数。
到了温度参数,就不如此了。温度这三个参数中最难控的。相对运行点的温度偏移,在时间上难以预测,在空间上(即在线圈内部)就更难控制了。
储存在磁体内的磁场能和机械能,很容易转化为热能,引起线圈内部某些位置的导体温度上升到超过其临界值。
实际上,最终超导磁体的所有“稳定性问题”都可以归为磁体设计者不能控制线圈温度保持在运行点。

本章我们考虑 1)超导线圈内控制温度的基本物理问题;2)线圈内不可预测温升发生可能性的稳定性评估方法。
第7和第8章同样讨论不同条件下线圈内温升:第7章讨论温升的原因或源;第8章讨论控制磁体不可预测温升的保护方法。
首先,LTS和HTS磁体的稳定性问题就截然不同。

\subsection*{LTS vs. HTS}
如图1.6给出的,稳定性的实现难度和费用随运行温度提高而降低。在下面的讨论中,我们将比较HTS和LTS的“稳定裕度”,并证明HTS磁体实际上
是非常稳定的。换句话说,任何HTS磁体都会达到其运行电流,且不会存在“提前失超(premature quench)”.

\section{稳定性理论和标准}

\subsection{公式6.1涉及的概念}
每一个概念……
\begin{description}
  \item[磁通跳跃] 如第五章……
  \item[低温稳定性] 基本概念……
  \item[动态稳定性] 第五章……
  \item[等效面积] 标准……
  \item[MPZ] 最小传播区域……
  \item[不稳定情况] 最后……
\end{description}



\begin{equation}% page362 第1个 6.10a
V_{cd}=R_mI_m\simeq R_m{(I_t-Ic)}
\end{equation}
\begin{equation}% page362 第2个 6.10b
G_{j}=V_{cd}I_t
\end{equation}
\begin{equation}% page362 第3个 6.11
{(G_j)}\simeq R_{m}I_{t}{(I_t-I_c)}
\end{equation}
\begin{equation}% page363 第1个 6.12
I_c(T)=I_{c_o}(\frac{T_c-T}{T_c-T_{o_p}}) (T_{o_p}\leq T \leq T_c)
\end{equation}
\begin{equation}% page363 第2个 6.13a
G_j(T)=0 (T_{o_p}\leq T \leq T_{cs})
\end{equation}
\begin{equation}% page363 第3个 6.13b
G_j(T)=R_mI_t^2(\frac{T-T_{cs}}{T_c-T_{cs}})(T_{cs}\leq T\leq T_c)
\end{equation}
\begin{equation}% page363 第4个 6.13c
G_j(T)=R_mI_t^2 (T\ge T_c)
\end{equation}
\begin{equation}% page364 第1个 6.13a
G_j(T)=0 (T_{o_p}\leq T \leq T_{cs})
\end{equation}
\begin{equation}% page364 第2个 S1.1
G_j(T)=R_mI_t[I_t-I_{co}(\frac{T_c-T}{T_c-T_{o_p}})] (T_{cs}\leq T\leq T_c)
\end{equation}

\begin{equation}% page364 第3个 S1.2
I_{c_o}=I_t(\frac{T_c-T_{op} }{T_c-T_{cs}})
\end{equation}
\begin{equation}% page364 第4个 6.13b
G_j(T)=R_mI_t^2(\frac{T-T_{cs}}{T_c-T_{cs}}) (T_{cs}\leq T\leq T_c)
\end{equation}
\begin{equation}% page364 第5个 6.13c
G_j(T)=R_mI_t^2 (T\ge T_c)
\end{equation}
\begin{equation}% page364 第6个 6.14a
T_{cs}\leq T\leq T_c:
\end{equation}
\begin{equation}% page364 第6个 6.14a
G_j(T)=R_m(T)I_t^2(\frac{T-T_{cs}}{T_c-T_{cs}})
\end{equation}
\begin{equation}% page364 第7个 6.14b
T\ge T_c:
\end{equation}
\begin{equation}% page364 第7个 6.14b
G_j(T)=R_m(T)I_t^2
\end{equation}
\begin{equation}% page365 第1个 6.15
C_{cd}(T)\frac {\partial T}{\partial t}=\nabla.[k_{cd}(T)\nabla T]+\rho _{cd}(T)J_{cd}^2(t)+g_d(t)-(\frac{f_p\ \\mathrm{P}_D}{A_{cd}})g_q(T)
\end{equation}
\begin{equation}% page365 第2个 6.16a
G_j(T)=R_mI_{c_o}^2(\frac{T-T_{op}}{T_c-T_{op}}) (T_{o_p}\leq T \leq T_c)
\end{equation}
\begin{equation}% page365 第2个 6.16b
\rho_{cd}J^2_{cd}(t)=\frac{\rho_{m}I_{co}^2}{A_{cd}A_m}
(\frac{T-T_{op}}{T_c-T_{op}}) (T_{o_p}\leq T \leq T_c)
\end{equation}
\begin{equation}% page365 第3个 6.17
g_q(T)=h_q(T-T_b)\simeq h_q(T-T_{op})
\end{equation}
\begin{equation}% page365 第4个 6.18
\frac{f_pP_Dh_q(T-T_{op}}{A_cd}\geq \frac{\rho_m I_{co}^2}{A_{cd}A_m}(\frac{T-T_{op}}{T_c-T_{op}})
\end{equation}
\begin{equation}% page365 第4个 6.18
\frac{\rho_m I_{co}^2}{f_pP_DA_mh_q(T_c-T_{op})}\leq 1
\end{equation}
\begin{equation}% page365 第5个 6.19
\alpha_{sk}=\frac{\rho_m I_{co}^2}{f_pP_DA_mh_q(T_c-T_{op})}
\end{equation}
\begin{equation}% page366 第1个 6.20
A_m=\frac {\rho_mI_{co}^2}{\alpha_{sk}f_pP_Dh_q(T_c-T_{op})}
\end{equation}
\begin{equation}% page366 第2个 6.6
J_{op}=\frac{I_{op}}{A_{sc}+A_ {\bar{m}}+A_m}
\end{equation}
\begin{equation}% page366 第2个 6.21a
=(\frac{\gamma_{m/s}}{\gamma_{m/s}+1})J_{m_0}
\end{equation}
\begin{equation}% page366 第3个 6.21b
\gamma_{m/s}\equiv \frac{A_m}{A_{sc}+A_m}
\end{equation}
\begin{equation}% page367 第1个 6.22
[J_{m_o}]_{sk}=\sqrt{\frac{f_pP_Dq_{fm}}{\rho_mA_m}}
\end{equation}
\begin{equation}% page368 第1个 S2.1
\frac {\rho_mI_{co}^2}{A_m}=f_pP_Dq_{fm}
\end{equation}
\begin{equation}% page368 第2个 6.22
[J_{m_o}]_{sk}=\sqrt{\frac{f_pP_Dq_{fm}}{\rho_mA_m}}
\end{equation}
\begin{equation}% page368 第3个 s1.2
I_{c_o}=I_t(\frac{T_c-T_{op} }{T_c-T_{cs}})
\end{equation}
\begin{equation}% page368 第4个
g_j(T)=(\frac{A_{cd}}{f_pP_D})R_mI_t^2=(\frac{A_{cd}}{f_pP_D})R_mI_{co}^2\frac{(T_c-T_{cs})^2(T-T_{cs})}{(T_c-T_{op})^3}
\end{equation}
\begin{equation}% page368 第5个
\frac{dg_j(T)}{dT}=(\frac{A_{cd}}{f_pP_D})R_mI_{co}^2\frac{(T_c-T_{cs})^2}{(T_c-T_{op})^3}
\end{equation}
\begin{equation}% page369 第1个 6.23
\int_{T_{op}}^{T_{eq}}[g_q(T)-(\frac{A_{cd}}{f_pP_D})g_j(T)]d(T)
=\int_{T_{op}}^{T_{eq}}[g_q(T)-\hat{g}_j(T)]dT=0
\end{equation}
\begin{equation}% page369 第2个 6.24a
g_j(T)=\rho_m(T)J_{m_o}^2(\frac{T-T_{op}}{T_c-T_{op}}) (T_{op}\leq T \leq T_c)
\end{equation}
\begin{equation}% page369 第3个 6.24b
g_j(T)=\rho_m(T)J_{m_o}^2 (T \geq T_c)
\end{equation}
\begin{equation}% page370 第1个 6.25a
V_s=V_c(\frac{I_s}{I_c})^n
\end{equation}
\begin{equation}% page370 第2个 6.25b
E_s=E_c(\frac{I_s}{I_c})^n
\end{equation}
\begin{equation}% page371 第1个 6.26a
R_s=R_c(\frac{I_s}{I_c})^{(n-1)}
\end{equation}
\begin{equation}% page371 第2个 6.26b
R_{dif}=nR_c(\frac{I_s}{I_c})^{(n-1)}
\end{equation}
\begin{equation}% page372 第1个 S3.1
R_s=\frac{V_s}{I_s}=\frac{V_c}{I_s}(\frac{I_s}{I_c})^2=\frac{V_c}{I_c}(\frac{I_s}{I_c})^{(n-1)}
\end{equation}
\begin{equation}% page372 第2个 6.26a
R_s=R_c(\frac{I_s}{I_c})^{(n-1)}
\end{equation}
\begin{equation}% page372 第3个 S3.2
R_{dif}=\frac{\partial V_s}{\partial I_s}=\frac{nV_c}{I_c}(\frac{I_s}{I_c})^{(n-1)}
\end{equation}
\begin{equation}% page372 第4个 6.26a
R_{dif}=nR_c(\frac{I_s}{I_c})^{(n-1)}
\end{equation}
\begin{equation}% page372 第5个S3.3a
I_t=I_m+I_s
\end{equation}
\begin{equation}% page372 第6个S3.3b
V_m=R_mI_m=V_s=V_c(\frac{I_s}{I_c})^{n}
\end{equation}
\begin{equation}% page372 第6个S3.3c
3\times 10^{-4}\ \mathrm{\Omega}\times I_m[\ \mathrm{A}]=10^{-5}V(\frac{90\ \mathrm{A}-I_m[\ \mathrm{A}]}{100\ \mathrm{A}})^{15}
\end{equation}
\begin{equation}% page372 第7个S3.4
P_{cd}=R_mI_mI_t=V_sI_t
\end{equation}
\begin{equation}% page375 第1个6.27a
R_m(T)=0.190+1.530(\frac{T-77}{293-77})\ [\ \mathrm{m\Omega}]
\end{equation}
\begin{equation}% page375 第1个6.27b
I_c(T)=100(\frac{93-T}{93-77})[\ \mathrm{A}]
\end{equation}
\begin{equation}% page375 第1个6.27C
V_s(T)=5[\frac{I_s(T)}{I_c(T)}]^{10} [\ \mathrm{\mu V}]
\end{equation}
\begin{equation}% page376 第1个S4.1a
I_t=I_s+I_m
\end{equation}
\begin{equation}% page376 第2个S4.1b
R_mI_m=V_s(I_s)
\end{equation}
\begin{equation}% page376 第3个S4.2a
I_s=(290\ \mathrm{A})-I_m
\end{equation}
\begin{equation}% page376 第4个S4.2b
(0.19\times10^{-3}\ \mathrm{\Omega})I_m=(5\times10^{-6}\ \mathrm{V})(\frac{290\ \mathrm {A}-I_m}{100\ \mathrm{A}})^{10}
\end{equation}
\begin{equation}% page376 第5个S4.3
(18\times10^{-3}\ \mathrm{V})=(5\times10^{-6}\ \mathrm{V})(\frac{195.26\ \mathrm{A}}{100\ \mathrm{A}})^n
\end{equation}
\begin{equation}% page376 第6个S4.4
V_{cd}=40\times10^{-3}\ \mathrm{V}=R_m(T)I_m
\end{equation}
\begin{equation}% page376 第7个S4.5a
10\times10^{-3}=[{[0.190+1.530(\frac{T-77}{293-77})]^\times10^{-3}\ \mathrm{\Omega}}]I_m(T)
\end{equation}
\begin{equation}% page376 第8个S4 .5b I_m(T)=\frac{40\times10^{-3}\ \mathrm{V}}{[0.190+1.530(\frac{T-77}{293-77})]\times10^{-3}\ \mathrm{\Omega}}I_m(T)
\end{equation}
\begin{equation}% page376 第9个S4 .5c
=\frac{40}{0.190+1.530(\frac{T-77}{293-77})}\ \mathrm{A}
\end{equation}
\begin{equation}% page377 第1个S4.6
40\times10^{-3}\ \mathrm{V}=(5\times10^{-6}V)[\frac{300A-I_m(T)}{I_c(T)}]^{12.24}
\end{equation}
\begin{equation}% page377 第2个S4.7
40\times10^{-3}\ \mathrm{V}=(5\times10^{-6}V)[\frac{300\ \mathrm{A}-\frac{40\ \mathrm{A}}{[0.190+1.530(\frac{T-77}{293-77})]}}{(100\ \mathrm{A})(\frac{93-T}{93-77})}]^{12.24}
\end{equation}
\begin{equation}% page377 第3个S4.8
8000=[\frac{16(406200T-28027799)}{(93-T)(135400T-6793895)}]^{12.24}
\end{equation}
\begin{equation}% page378 第1个6.28a
C_{cd}(T)\frac{\partial T}{\partial t}=\nabla.[k_{cd}(T)\nabla T]+\rho_{cd}(T)J_{cd_o}^2(t)+g_d(t)-(\frac{f_pP_D}{A_{cd}})h_{he}(T-T_{he})
\end{equation}
\begin{equation}% page378 第2个6.28b
C_{he}(T_{he})\frac{\partial T_{he}}{\partial t}=(\frac{f_{p}P_D}{A_cd})h_{he}(T-T_{he})
\end{equation}
\begin{equation}% page379 第1个6.29
h_{he}=0.0259(\frac{k_{he}}{D_{hy}})Re^{0.8}Pr^{0.4}(\frac{T_{he}}{T_{cd}})^{-0.716}
\end{equation}
\begin{equation}% page381 第1个6.30
I_{lim}=\sqrt{\frac{A_mf_p\ \mathrm{P}_Dh_{he}(T_c-T_{op})}{\rho_m}}
\end{equation}

\subsection{热能量}

\subsection{热传导}

\subsection{焦耳热}

\subsection{扰动谱}

\subsection{稳定裕度 vs 扰动能}

\subsection{冷却}

\section{电流密度}

\subsection{横截面积}

\subsection{复合物超导体}

\subsection{绕组中的电流密度}


\section{专题}
\subsection{讨论1:低温稳定性——电路模型}

\newpage
\subsection{问题2:低温稳定性——温度依赖}

\newpage
\subsection{讨论3:Stekly低温稳定性判据}

\newpage
\subsection{讨论4:复合物超导体}

\newpage
\subsection{问题5:低温稳定性——非线性冷却曲线}

\newpage
\subsection{讨论6:等效面积判据}

\newpage
\subsection{讨论7:超导体“指数”n}

\newpage
\subsection{问题8:复合物超导体(n)——电路模型}

\newpage
\subsection{问题9:电流脉冲下的YBCO}

\newpage
\subsection{讨论10:CICC导体}

\newpage
\subsection{问题11:冷却复合物导体的伏安关系}

\newpage
\subsection{问题12:混合III SCM的稳定性分析}

\newpage
\subsection{讨论13:cryostable vs 准绝热磁体}

\newpage
\subsection{讨论14:MPZ概念}

\newpage
\subsection{问题15:绝热绕组中的耗散能密度}