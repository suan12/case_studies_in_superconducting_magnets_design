\chapter{保护}
\section{引言}
保护是五大关键设计和运行项目之一——其他四项是稳定性、机械完整性、制冷和导体。
\subsection{热能密度 vs. 磁能密度}
除非绕组得到了保护,不然磁体绕组的一小部分,即“热点”,就要吸收掉存储于绕组中的大部分磁能。这样,该部分将过热并永久性损坏。
不过,熔化磁体中单位绕组体积的热能密度要远大于磁体存储的磁能密度。

仅考虑将磁体内部空间内存储的能量全部绝热转换为热,引起铜(绕组的一种代表性材料)的焓密度$h_{Cu}(T)$变化。如果是从4K(或者80K)加热到它的熔点1356K,
那么初始磁感应密度$B_0$将高达$~150 T$:
\begin{eqnarray}\label{eqn: 8.1}
% \nonumber % Remove numbering (before each equation)
  \frac{B_0^2}{2\mu_0}&=& h_{Cu}(1356K)-h_{Cu}(4K/80K)\approx 5.2\times 10^9 J/m^3 \nonumber\\
  B_0 &\approx
   &\sqrt{2(4\pi \times 10^-7 H/m)(5.2\times 10^9 J/m^3)}\approx 115 T
\end{eqnarray}

\subsection{热点和热点温度}

\subsection{绕组材料的温度数据}

\subsection{$T_f$的安全、风险、高度风险区间}

\subsection{温度引起的应变}


\section{绝热加热}
\subsection{恒定电流模式下的绝热加热}
\subsection{恒定放电量模式下的绝热加热}
\subsection{引线短接的磁体的绝热加热}
\subsection{恒定电压模式下的绝热加热}

\section{高电压}
\subsection{电弧环境}
\subsection{Paschen电压试验}
\subsection{失超磁体内的电压峰值}

\section{正常区传播(NZP)}

\subsection{轴向NZP速度}
\subsection{“制冷”条件下的NZP}
\subsection{横向匝间速度}
\subsection{热-流体失超恢复(THQB)}
\subsection{交流损耗诱导的NZP}

\section{计算机仿真}

\section{自保护磁体}
\subsection{尺度限制}

\section{孤立磁体的被动保护}


\section{主动保护}
\subsection{过热}
\subsection{多线圈磁体中的过压}
\subsection{主动保护技术:检测-抑制}
\subsection{主动保护技术:检测-激活加热器}
\subsection{失超电压保护技术:基本电桥}

\section{专题}
\subsection{问题1:大型超导磁体的回温}

\newpage
\subsection{问题2:6 kA气冷HTS引线的保护}

\newpage
\subsection{问题3:制冷机制冷的NbTi磁体的保护}

\newpage
\subsection{问题4:混合III SCM的“热点”温度}

\newpage
\subsection{讨论5:失超电压探测——一个变种}

\newpage
\subsection{问题6:抑制电阻的设计}

\newpage
\subsection{讨论7:磁体的“缓慢”放电模式}

\newpage
\subsection{讨论8:低阻电阻器设计}

\newpage
\subsection{讨论9:过热\& 内部电压判据}

\newpage
\subsection{讨论10:Bi2223带电流引线的保护}

\newpage
\subsection{讨论11:$MgB_2$磁体的主动保护}

\newpage
\subsection{问题12:NMR磁体的被动保护}

\newpage
\subsection{讨论13:HTS磁体到底要不要保护?}
